\documentclass[]{article}

\usepackage{anysize}
\usepackage{amsfonts}
\usepackage{amsmath}
\usepackage{amssymb}
\usepackage{enumerate}
\usepackage{cancel}
\usepackage{graphicx}
\usepackage{hyperref}
\usepackage{color}

%%% for citations. See Lucy's tutorial: https://github.com/LucyMcGowan/Tutorials/blob/master/BiblatexTutorial.md
\usepackage[backend=biber,maxnames=10,citestyle=science]{biblatex}
\addbibresource{summaryOfRef.bib}

%%% when using biber as a "backend" compile like this:
% pdflatex myFile
% biber myFile
% pdflatex myFile


\let\epsilon\varepsilon
\newcommand\SLASH{\char`\\}
\newcommand{\csch}{\text{csch}}
\marginsize{0.5in}{1in}{0.5in}{1in}
\setlength\parindent{0pt}

% Title Page: Modify as needed
\title{Summary for Previous research related to Probability Scale Residuals}
\author{Summarized by Svetlana Eden}
\date{Month Day, 2015}

\hypersetup{hidelinks}

% \usepackage[pdftex,bookmarks,pagebackref,pdfpagemode=UseOutlines,
%      colorlinks,linkcolor=\linkcol,
%      pdfauthor={Svetlana K Eden},
%      pdftitle={\titl}]{hyperref}


\begin{document}
\maketitle
\tableofcontents
\listoffigures
\listoftables
\clearpage

%%%%%%%%%%%%%%%%%%%%%%%%%%%%%%%%%%%%%%%%%%%%%%%%%%%%%%% Matrix algebra
%%%%%%%%%%%%%%%%%%%%%%%%%%%%%%%%%%%%%%%%%%%%%%%%%%%%%%% Matrix algebra
\section{Oral Exam Stuff}
Survival analysis is a well researched topic: time to event, time to event with censoring, multiple events, competing risk events, current status analysis. We are interested in association of two survival events. For example, we could be interested in time to breast cancer diagnosis in mothers and their daughters. It could also be a time to an onset of an opportunistic infection and time to viral failure in HIV infected patients. These motivating examples include paired survival, but in general, survival data doesn't have to be paired. The interest to bivariate survival to the best of our knowledge goes back to Clayton (1978)[??? check this]. Since then different methods were employed to address this question.\\
~\\
\begin{itemize}
	\item Evaluation of the bivariate survival function with two marginals and a correlation component:
	Dabrowska, Pruitt, Prentice, Van Der Laan.
	\item Martingale residuals (Sih\&Louice, [??? look for more papers on martingale residuals])
	\item sort of hazard ratio for bivariate surviaval (Clayton, Fan, Oakes, Sankaran et al)
	\item copulas (zhang2008, [??? provide more references on copulas])
	\item other regression residuals (Cuzick1982, Prentice)
	\item Kendall's tau: weighted, global, local. 
	\item distance measure [??? not sure if I want to address this]
	\item Spearman correlation (Dabrowska1986, Ding2012, Cuzick1982)
	\item Probability scale residuals, PSR (Li\&Shepard)
	\item Challanges and limitations (discrete data covariates)
	\item Qi - covariates (look at prentice and Cuzick)
	\item show that this is Spearman
	\item apply to dataset
	\item simmulations
	\item explore performance of partial and conditional Spearman.
\end{itemize}

Correlation of PSR is in fact a Spearman correlation generalized to the case with sensoring. Although, this kind of association was covered by (Cuzick1982, Dabrowska1986, Ding2012), they mentioned only a continuous case. PSR of (Li\&Shepard) are generalized to discrete case for the case of independent censoring and for the case of currents status data.\\
~\\
Future direction. We want to compare the performance of PSR to the performance of martingale residuals for the discrete case.

\section*{Introduction}
The interest in survival outcome can be traced back to year 1846, when a Hungarian doctor of the Vienna General Hospital maternity clinic, Ignaz Semmelweis, collected data on the rate of death from childbed fever. He hypothesized that a child birth is helped by a doctor or a medical student, the women's rate of death from childbed fever is higher because of a disease that is caused by cadaverous particles that medical students and doctors have on on their hands after disecting cadavras. Jumping forward about 100 years, survival data was routinely collected and used by hospital, agencies, and governments [READ YOUR epidemiology book]. The recorded data were used to estimate rates of death and other events. Hald in 1949 ["Maximum Likelihood estimation of th eparameters of a normal distribution which is trancated at a known point", \emph{Skandinavisk Aktuarielidskrift, 32 (1949)}, 119-134], usggested a more complex view of survival data and introduced a concept of sensoring. This concept elegantly taken into accout by one of the most famous work on survival, Kaplan and Meier [1958, p.459]. Cox (1972) extended Kaplan\&Meier results even further and introduced a regression model for censored suvival data. His model utilized a concept of a \emph{hazard rate}:
$$
\begin{aligned}
	\lambda(s|y) &= \frac{\partial}{\partial s}\left\{  -log \mathcal{S}(s|y)  \right\} =
	             \frac{f(s|y)}{\mathcal{S}(s|y)}
\end{aligned}
$$
Where $\mathcal{S}(s|y) = 1-F(s|y)$, and $F(s|y)$ and $f(s|y)$ are cumulative distribution function and density function respectively. In the same paper (p.195) he also generalized his model for a case when one subject can have two events.\\
Unlike Cox, Basu (1971) [get this paper from the library ??? this does not add up because Cox wrote his paper later] considered a \emph{paired} bivariate survival with the hazard rate defined as:
$$
\begin{aligned}
	\lambda(s,t) &= \frac{f(s,t)}{\mathcal{S}(s,t)} = \frac{\partial}{\partial s}\left\{  -log \mathcal{S}(s,t)  \right\}\frac{\partial}{\partial t}\left\{  -log \mathcal{S}(s,t)  \right\} - \frac{\partial^2}{\partial s\partial t}\left\{  -log \mathcal{S}(s,t)  \right\}
\end{aligned}
$$
Later in 1978, Clayton [???] addressed a question of association between paired survival times of fathers and sons and defined a measure similar to harzard ratio. He wanted this measure to be
\begin{enumerate}
	\item  easy to compute
	\item expressible as a constant ratio of age-specific rates
  \item symmetrical in two variables
\end{enumerate}
If we denote time to event for fathers to be $s$ and time to event for sons to be $t$, this measure can be written in the following way:
$$
\begin{aligned}
	\theta = \frac{\lambda_s(s_0|t=t_0)}{\lambda_s(s_0|t>t_0)} = \frac{\lambda_t(t_0|s=s_0)}{\lambda_t(t_0|s>s_0)}
\end{aligned}
$$
The requirement of symmetry is motivated by the fact than neither son nor father are causes of event, rather, there is an unobservable variable that is associated with the event for both of them.

This measure can be interpreted for sons (as much as for fathers) in the following way: how much more likely for a son to have an event at time $s_0$ given that his father had an event at time $t_0$ compared to a son whose father survived until time $t_0$ without the event.

Independence between fathers and sons would mean $\theta = 1$, positive association $\theta > 1$ and negative association $\theta < 1$. Clayton also explained how his model could be generalized to include covariates, and in estimating the model he adopts an approach similar to Cox's. He also derived likelihood for estimating $\theta$ when two marginal distributions are completely unknown.\\

Note that although Clayton developed $\theta$ in the context of a model, the model was not necessacery to check independence. Independence could be checked by evaluating $\theta$.\\
Oakes (1982 ??? reference [YOU DIDN'T COVER THIS BEFORE]) criticised Clayton's likelihood approach to estimating $\theta$ [make sure that you know speficic critisizm points] and introduced an alternative method based on \emph{Kendall}'s $\tau$, coefficient of concordance. In 1989 he also introduced a bivariate survival model based on \emph{frailties}: an unobserved variable that induces dependence between two survival times that both depend on this variable. In the context of this model, he further utilized coefficient $\theta$ introduced by Clayton. He also noted that bivariate distributions introduced by frailty models are part of a larger class of archimedian distributions studied by Genest and MacKay (1986a,b [I did not read their papers and not sure if I should]):
$$
\begin{aligned}
	S(t) = p\left[ q\left\{ S_1(t_1) \right\}  + q\left\{ S_2(t_2) \right\}  \right]
\end{aligned}
$$
Where $p(u)$ is an inverse function of $q(v)$. He shows how to estimate his model in the presence of censoring in either one or both components. He showed that in the context of his model, $\theta$ and \emph{Kendall}'s $\tau$ were related through the following equality: $\tau(v) = \frac{\theta(v)-1}{\theta(v)+1}$, where $v=S(t_1, t_2)$ [I am not sure if all this is necessary].

\textbf{Show the resemblence with archemedian copulas}

\textbf{READ: Hanley, J. A., and Parnes, M. N. (1983), "Nonparametric Estimation of a Multivariate Distribution in the Presence of Censoring," Biometrics, 39, 129-139.}


\section{Test of Association Between Two Ordinal Variables While Adjusting for Covariates, Li and Shepherd, 2010 \cite{li2010test}}
We have two ordinal variables $\pmb{X}$ and $\pmb{Y}$ and we would like to test if after adjustment for $\pmb{Z}$ they are independent. We do this in two ways:
\begin{enumerate}[1)]
	\item compare \emph{observed} distribution with \emph{expected} distribution under the null.
	\item run two regressions $\pmb{X\sim Z}$ and $\pmb{Y\sim Z}$, obtain their residuals, $r_x$ and $r_y$ and test them (???)
\end{enumerate}
\subsection{Observed vs expected distributions statistic, $T_1$}
In general, the joint distribution can be written:
$$
\begin{aligned}
	P(Y,X) = \int_z P(Y,X|\pmb{Z})dP(\pmb{Z})
\end{aligned}
$$
Under the null of conditional independence:
$$
\begin{aligned}
	P_0(Y,X) = \int_z P(Y|\pmb{Z})P(X|\pmb{Z})dP(\pmb{Z})
\end{aligned}
$$
More notations:
$$
\begin{aligned}
  p_{i}^j &= P(Y_i = j|\pmb{Z}_i = \pmb{z}_i),~~j=1,...,s\\
  q_{i}^l &= P(X_i = l|\pmb{Z}_i = \pmb{z}_i),~~l=1,...,t\\
	&\gamma_i^j=P(Y_i\leq j|\pmb{Z}=\pmb{z}_i), ~~p_{i}^j = \gamma^{j}_i - \gamma^{j-1}_i\\
	 &similar~thing~for~q_i^l
\end{aligned}
$$
Now we get estimates $\hat{p}_i^j$ and $\hat{q}_i^l$ from regressions $\pmb{X~Z}$ and $\pmb{Y~Z}$ and then estimate the null distribution:
$$
\begin{aligned}
  \hat{P}_0 = \left\{\hat{\pi}_0^{jl}\right\},~~where~~\hat{\pi}_0^{jl} = \frac{1}{n}\sum_i \hat{p}_i^j \cdot \hat{q}_i^l
\end{aligned}
$$
Now we estimate the alternative distribution:
$$
\begin{aligned}
  \hat{P}_{jl} = \frac{n_{jl}}{n}
\end{aligned}
$$
Where $n_{jl}$ is number of subjects with levels j and l respectively.\\
We now summarize the observed and expected distribution separately by calculating \emph{Goodman} and \emph{Kruskal}'s gamma \cite{goodman1979measures} (Goodman, L. A., and Kruskal, W. H. (1954), "Measures of Association for Cross Classifications," Journal of the American Statistical Association).
$$
\begin{aligned}
	C&=\sum_{j_1<j_2, ~l_1<l_2}\pi_{j_1l_1}\pi_{j_2l_2}\\
	D&=\sum_{j_1<j_2, ~l_1>l_2}\pi_{j_1l_1}\pi_{j_2l_2}\\
  \Gamma &= \frac{C-D}{C+D}\\
\end{aligned}
$$
{\scriptsize{\textbf{Note:} Strictly speaking, $C$ and $D$ from above have to be multiplied by $2$ (see section \ref{sssec:GoodmanKruskal}). But because 2 cancels out, I guess, they omit the $2$}}\\
Let $\Gamma_1 = \Gamma(\hat{P})$ and  $\Gamma_0 = \Gamma(\hat{P}_0)$ be the gamma for the observed and expected distributions for $(Y,X)$. The test statistics is $T_1=\Gamma_1 - \Gamma_0$. When the multinomial models are correct (?!), under the null:
$$
\begin{aligned}
	\hat{P}_0 \rightarrow P,~~~\hat{P} \rightarrow P\\
	T_1 = \Gamma(\hat{P}) - \Gamma(\hat{P_0}) \underset{n\rightarrow \infty}{\longrightarrow} 0
\end{aligned}
$$
\subsubsection{\emph{Goodman} and \emph{Kruskal}'s gamma}
\label{sssec:GoodmanKruskal}
We have two $iid$ subjects $1$ and $2$ with two ordered responses $\underline{a}_1,\underline{b}_1$ and $\underline{a}_2,\underline{b}_2$ that have categories $1,2,...,s$, and $1,2,...,t$. If we summarised proportions of each combination in a table we would use the following notations:\\
~\\
\begin{tabular}{ l | c | c | c | c | c}
  \hline
  ~ & $B_1$ & $B_2$ & ... & $B_t$ & Total\\
  \hline
  $A_1$ & $\rho_{11}$&$\rho_{12}$&... &$\rho_{1t}$& $\rho_{1\cdot}$\\
  $A_2$ & $\rho_{21}$&$\rho_{22}$&... &$\rho_{2t}$& $\rho_{2\cdot}$\\
  ... & ...&...&... &...& ...\\
  $A_s$ & $\rho_{s1}$&$\rho_{s2}$&... &$\rho_{st}$& $\rho_{1\cdot}$\\
  \hline
  Total & $\rho_{\cdot 1}$& $\rho_{\cdot 2}$&... & $\rho_{\cdot t}$& 1\\
\end{tabular}
~\\


Let's denote:
$$
\begin{aligned}
	\Pi_c&=Pr\left\{ \underline{a}_1 <\underline{a}_2 ~and ~\underline{b}_1<\underline{b}_2;~~or~\underline{a}_1 >\underline{a}_2 ~and ~\underline{b}_1>\underline{b}_2\right\}\\
	\Pi_d&=Pr\left\{ \underline{a}_1 <\underline{a}_2 ~and ~\underline{b}_1>\underline{b}_2;~~or~\underline{a}_1 >\underline{a}_2 ~and ~\underline{b}_1<\underline{b}_2\right\}\\
	\Pi_t&=Pr\left\{\underline{a}_1 =\underline{a}_2 ~or ~\underline{b}_1=\underline{b}_2\right\}\\
\end{aligned}
$$
From the above, the conditional probability for the \emph{like} order given no ties is $\Pi_c/(1-\Pi_t)$ and the conditional probability fo the $unlike$ order given no ties is $\Pi_d/(1-\Pi_t)$. Keeping in mind that $\Pi_c +\Pi_d +\Pi_t=1$, a possible measure of association, $\gamma$ is:
$$
\begin{aligned}
	\gamma = \frac{\Pi_c - \Pi_d}{1-\Pi_t} =  \frac{\Pi_c - \Pi_d}{\Pi_c + \Pi_d} = \frac{2\Pi_c - 1 + \Pi_t}{1- \Pi_t}
\end{aligned}
$$
This measure can be interpreted as \emph{difference} between the conditional probabilites of like and unlike orders: how much more probable it is to get like than unlike orders in the two classifications when two individuals are chosen randomly from the population.\\
The latter equality is especially usefull because we can compute it using easily verifyable relationships:
$$
\begin{aligned}
	\Pi_c&=2\sum_a \sum_b \rho_{ab}\left\{ \sum_{a'>a} \sum_{b'>b} \rho_{a'b'} \right\}\\
	\Pi_t&=\sum_a \rho_{a\cdot}^2 + \sum_b \rho_{\cdot b}^2 - \sum_{a} \sum_{b} \rho_{ab}^2\\
\end{aligned}
$$

{\scriptsize{\textbf{Verification}\\
We know that $P(A\cup B) = P(A) + P(B) - P(A\cap B)$. Let's denote $A=\{\underline{a}_1 =\underline{a}_2\}$ and  $B=\{\underline{b}_1 =\underline{b}_2\}$. We think of two subjects having the same categories is like drawing from the cells of the above table with replacement because two subjects are independent. Keeping in mind that $P(\{\underline{a}_1=A_i\cap \underline{b}_1 =B_j) =\rho_{A_i,B_j}$, the probability of two subjects having the same categories is $\rho_{A_i,B_j}\cdot \rho_{A_i,B_j}$ and therefore $P(A\cap B) = \sum_{i,j}\rho_{i,j}^2$. Similarly, and imagining that we draw two subjects independently with replacement based on marginal distributions for $A$ and $B$: $P(A) = \sum_{i}\rho_{i\cdot}^2$ and $P(B) = \sum_{j}\rho_{\cdot j}^2$
\\
Now let's denote $A=\{\underline{a}_1 <\underline{a}_2 \cap\underline{b}_1 <\underline{b}_2\}$ and  $B=\{\underline{a}_1 >\underline{a}_2 \cap \underline{b}_1 >\underline{b}_2\}$. Because these two events do not overlap, we have: $P(A\cup B) = P(A) + P(B)$. Applying the same principle of drawing with replacement, we have: $P(A) = \sum_{i,j}\left(\sum_{k<i,l<j}\rho_{k,l} \right)\rho_{i,j}$ and $P(B) = \sum_{i,j}\left(\sum_{k>i,l>j}\rho_{k,l} \right)\rho_{i,j}$. It so happends that when these two probabilities are summed up, we get the double of one of them. We show it on the example of 2 $A$ categories and three $B$ categories: $P(A\cup B) = [\rho_{11}\rho_{22} + (\rho_{12} + \rho_{11})\rho_{23}]~ + ~[(\rho_{22} + \rho_{23})\rho_{11} + \rho_{23}\rho_{12}] ~=~ 2(\rho_{22} + \rho_{23})\rho_{11} + 2\rho_{12}\rho_{23}$, so $P(A\cap B) = 2\cdot \sum_{i,j}\left(\sum_{k<i,l<j}\rho_{k,l} \right)\rho_{i,j}$
\\
\textbf{Note}(Unrelated to the verification above)\\
In the same paper, there is also an index, $\lambda$, that measures association between unordered categories. Another interesting part of the paper is where they derive the same index $\lambda$ but using a loss function.
}}\\

Properties:
\begin{enumerate}[(i)]
	\item $\gamma$ is indeterminate if the population is concentrated in a single row or column of the cross-classification table.
	\item $\gamma$ is $1$ if the population is concentrated in a upper-left to lower-right diagonal of the cross classification table. $\gamma$ is $-1$ if the population is concentrated in a lower-left to upper-right diagonal of the table. It can be diagonal or step-diagonal, because the conflinct in order is impossible:\\
	~\\
	\begin{tabular}{ c | c | c }
	  $\rho_{11}$ & $\rho_{12}$ & 0\\
		\hline
	  0 & $\rho_{22}$ & $\rho_{23}$\\
		\hline
	  0 & 0 & $\rho_{33}$\\
	\end{tabular}
	~\\
	\item $\gamma$ is $0$ in the case of independence, but the converse need not hold except in the $2\times 2$ case. Example:\\
	~\\
	\begin{tabular}{ c | c | c }
	  .2 & 0 & .2\\
		\hline
	  0 & .2 & 0\\
		\hline
	  .2 & 0 & .2\\
	\end{tabular}
	~\\
	\item In $2\times 2$ case, $\gamma$ is Yule's coefficient of association $\gamma=\frac{\rho_{11}\rho_{22} - \rho_{12}\rho_{21}}{\rho_{11}\rho_{11}+\rho_{12}\rho_{21}}$
  \item Also, in some way, it is similar to Stewart's measure.
\end{enumerate}
\subsection{Residual-based test statistic, $T_2$}
For simplicity, only notations for $Y$ are introduced, the notations for $X$ are similar.\\
Assume that $Y$ is distributed multinomially with parameter $\theta^Y$. Let $Y_i=y_i$ be the observed outcome for subject $i$. The corresponding distribution of possible outcome levels $Y_{i,fit}\sim Y_i|\pmb{z}_i$ given covariate $\pmb{z}_i$ is multinomial with probability $\{p_i^j\}$. Instead of calculating the difference between $Y_{i,fit}$ and $y_i$ we compare them and give them scores of $1$, $-1$, and $0$ if $y_i>Y_{i,fit}$, $y_i<Y_{i,fit}$, and $y_i=Y_{i,fit}$ respectively.\\
Keeping in mind that $Y_{i,fit}$ is the \emph{random variable} in the following derivations:
$$
\begin{aligned}
  p_{i,high} &= P(y_i>Y_{i,fit}) = P(Y_{i,fit}<y_i) = P(Y_i \leq y_i - 1|\pmb{Z}=\pmb{z}_i) = \gamma_i^{y_i-1}\\
  p_{i,low} &= P(y_i<Y_{i,fit}) = P(Y_{i,fit}>y_i)= 1 - P(Y_i \leq y_i|\pmb{Z}=\pmb{z}_i) = 1-\gamma_i^{y_i}\\
  p_{i,tie} &= P(y_i=Y_{i,fit}) = P(Y_{i,fit}=y_i)= p_i^{y_i}
\end{aligned}
$$
The expected score $Y_{i,res} = p_{i,high} - p_{i,low}$, which is a function of data $(Y_i, \pmb{Z}_i)$ and $\theta^Y$, and we denote $Y_{i,res}$ realization as $y_{i,res} = Y_{i, res|\hat{\theta}^Y}$. Similarly, for $X$, we define: $X_{i,res} = q_{i,high} - q_{i,low}$ and $x_{i,res}$.\\
When the models for $P(X|\pmb{Z})$ and  $P(Y|\pmb{Z})$ are correct, and keeping in mind that $p_i^j = \gamma_i^j - \gamma_i^{j-1}$ and $\gamma_i^j = \sum_{k = 1}^{j}p_i^k$, we have:
$$
\begin{aligned}
  E(p_{i,high}|\pmb{Z}_i) &= E \left(\gamma_i^{Y_i-1} |\pmb{Z}_i\right) = \sum_{j=1}^s p_i^j \gamma_i^{j-1}=\sum_{j=1}^s p_i^j \left(\sum_{k = 1}^{j-1}p_i^k\right) = \sum_{j_1<j_2} p_i^{j_1} p_i^{j_2}\\
  E(p_{i,low}|\pmb{Z}_i) &= E \left(1-\gamma_i^{Y_i} |\pmb{Z}_i\right) = \sum_{j=1}^s p_i^j (1-\gamma_i^{j}) = \sum_{j=1}^s p_i^j \left(1-\sum_{k = 1}^{j}p_i^k\right) = \sum_{j=1}^s p_i^j \left(\sum_{k = j+1}^{s}p_i^k\right) = \sum_{j_1<j_2} p_i^{j_1} p_i^{j_2}\\
\end{aligned}
$$
Therefore $E(p_{i,high} - p_{i,low}|\pmb{Z}_i)=0$ and $E(Y_{i,res})=E[E(p_{i,high} - p_{i,low}|\pmb{Z}_i)]=0$. Similarly, for $E(X_{i,res})=0$ and:
$$
\begin{aligned}
	E(Y_{i,res},X_{i,res}) &= E[E(Y_{i,res},X_{i,res}|\pmb{Z}_i)] = E[E(Y_{i,res}|\pmb{Z}_i)E(X_{i,res}|\pmb{Z}_i)] = 0\\
	&therefore\\
	cov(Y_{i,res},X_{i,res}) &= E(Y_{i,res},X_{i,res}) - E(Y_{i,res})E(X_{i,res}) = 0\\
\end{aligned}
$$
Therefore, we use the sample correlation coefficient, $T_2$, between $Y_{i,res}$ and $Y_{i,res}$ across all subjects as a test statistics. Under the null, $T_2\underset{n\rightarrow \infty}{\longrightarrow}0$
\subsection{A variation of residual statistic, $T_3$}
Alternatively, we can compute a statistic based on concordance vs discordance of pairs $(Y_i, X_i)$ and $(Y_i', X_i')$:
$$
\begin{aligned}
	Concordant ~pairs: ~Y_i>X_i ~and ~Y_i'>X_i'~~or~~ Y_i<X_i ~or~ Y_i'<X_i'\\
	Discordant ~pairs: ~Y_i>X_i ~and ~Y_i'<X_i'~~or~~ Y_i<X_i ~or~ Y_i'>X_i'\\
	Tied ~pairs: otherwise\\
\end{aligned}
$$
 Under the null the probability of concordance: $C_i = p_{i,high}q_{i,high} + p_{i,low}q_{i,low}$, and the probability of discordance $D_i = p_{i,high}q_{i,low} + p_{i,low}q_{i,high}$. 
$$
\begin{aligned}
	C_i - D_i &= p_{i,high}q_{i,high} + p_{i,low}q_{i,low} -p_{i,high}q_{i,low} - p_{i,low}q_{i,high}=\\
	 &= (p_{i,high}-p_{i,low}) (q_{i,high} - q_{i,low})=\\
	 &=Y_{i,res}X_{i,res}\\
\end{aligned}
$$
We know that $E(C_i - D_i)=E(Y_{i,res}X_{i,res}) = 0$, so we suggest a third statistic:
$$
\begin{aligned}
	T_3 = E(\hat{C}_i - \hat{D}_i) &=\frac{1}{n} \sum_i y_{i,res}x_{i,res}\\
\end{aligned}
$$
\subsection{Two method of computing the p-value}
\subsubsection{Parametric Bootstrap}
We generate new datasets from the product distribution $\{\hat{p}_i^j\hat{q}_i^l\}_{j,l}$ for each subject ($i=1,...,n$). Then we carry out the entire estimating procedure for the replicate dataset to obtain the corresponding statisitc, denoted as $T^*$. We do this $N^{emp}$ times and get an empirical distribution of $T$ under the null. The two-sided p-value is then computed as either:
$$
\begin{aligned}
	\#(|T^*|\geq |T|)/N^{emp}
\end{aligned}
$$
or
$$
\begin{aligned}
	2\times min\{\#(T^*\geq T),\#(T^*\leq T)\}/N^{emp}
\end{aligned}
$$
\subsubsection{Asymptotic Distribution}
Estimate $\hat{\theta}$ can be obtained by solving the equation $\sum_{i=1}^n \Psi_i(\theta) = 0$ where $\Psi_i(\theta) = \Psi(Y_i, X_i, \pmb{Z}_i; \theta)$ is a \emph{p}-variate function that doesn't depend on $i$ or $n$ and satisfies $E_{\theta}[\Psi_i(\theta)]=0$. From \emph{M}-estimation theory (Stefanski and Boos 2002)\cite{stefanski2002calculus}, if $\Psi$ is suitable smooth, then as $n\leftarrow \infty$,
$$
\begin{aligned}
	\sqrt{n}(\hat{\pmb{\theta}} - \pmb{\theta}) &\underset{d}{\longrightarrow}N(\pmb{0}, V(\theta)),~~where\\
	  V(\theta) &= A(\theta)^{-1} B(\theta)[A(\theta)^{-1}]'\\
		A(\theta) &= E \left[-\frac{\partial}{\partial \theta} \Psi_i(\theta)\right]\\
		B(\theta) &= E \left[\Psi_i(\theta)\Psi_i'(\theta)\right]\\
\end{aligned}
$$
 {\scriptsize{\textbf{My explanation}(might not to be correct): Because $E_{\theta}[\Psi(\theta)]=\pmb{0}$}, the variance of $\Psi(\theta)$ is $B(\theta) = E \left[\Psi_i(\theta)\Psi_i'(\theta)\right]$, therefore $\sqrt{n}(\hat{\Psi}(\theta) - \Psi(\theta)) \underset{d}{\longrightarrow}N(\pmb{0}, B(\theta))$. We can express $\theta$ as $\Psi^{-1}_i(\nu)$ (an inverse of $\Psi$, do not confuse with power $-1$) and apply the delta method keeping in mind that $\frac{\partial \Psi^{-1}_i(\theta)}{\partial \nu} = -\frac{1}{\partial \Psi_i(\nu)/ \partial \theta}$. Then, according to the delta method: $\sqrt{n}(\hat{\theta} - \theta) \underset{d}{\longrightarrow}N\left(\pmb{0}, \frac{B(\theta)}{(\partial (\Psi^{-1}_i(\theta)) /\partial \theta)^2}\right)$ (here $-1$ is used in sense of power). My $i$ indices are all messed up in this explanation.
 }\\
 ~\\
Now we apply \emph{delta} method to a statistic, based on $\theta$:

$$
\begin{aligned}
	\sqrt{n}(g(\hat{\pmb{\theta}}) - g(\pmb{\theta})) &\underset{d}{\longrightarrow}N(0, \sigma^2),~~where\\
	  \sigma^2 &= \left[\frac{\partial}{\partial \theta}g(\pmb{\theta})\right]V(\theta)\left[\frac{\partial}{\partial \theta}g(\pmb{\theta})\right]'
\end{aligned}
$$
In order to estimate $\sigma^2$ we estimate:

$$
\begin{aligned}
	\hat{A}(\pmb{\theta}) &= \frac{1}{n}\sum_{i}\left[-\frac{\partial}{\partial\pmb{\theta}}\Psi_i(\hat{\pmb{\theta}})\right]\\
	\hat{B}(\pmb{\theta}) &= \frac{1}{n}\sum_{i}\Psi_i(\hat{\pmb{\theta}})\Psi_i(\hat{\pmb{\theta}})'\\
	\widehat{\frac{\partial}{\partial\pmb{\theta}}g(\pmb{\theta})} &= \frac{\partial}{\partial\pmb{\theta}}g(\hat{\pmb{\theta}})
\end{aligned}
$$
Under the null the \emph{p-value} can be computed approximately as $2\Phi\left(\frac{-|T|}{\hat{simga}/n}\right)$, where $\Phi$ is the cumulative distribution function of the standard normal distribution.
\subsubsection{Defining $\theta$, $\Psi_i(\theta)$, and $g(\theta)$}
For all three statistics, the parameter vector will have the form: $\pmb{\theta} = (\pmb{\theta}^Y, \pmb{\theta}^X, \pmb{\theta}^T)$. The corresponding estimating function $\Psi_i(\pmb{\theta})$ will have the form:
$$
\begin{aligned}
	\Psi_i(\pmb{\theta})^T = \left(\frac{\partial}{\partial \pmb{\theta}^Y}l_Y(Y_i, \pmb{Z}_i; \pmb{\theta}^Y),~\frac{\partial}{\partial \pmb{\theta}^X} l_X(X_i, \pmb{Z}_i;\pmb{\theta}^X), \psi(Y_i, X_i, \pmb{Z}_i;\pmb{\theta})\right)
\end{aligned}
$$
Where $l_X$ and $l_Y$ are the log-likelihood functions of the multinomial models that are used to model $P(Y|\pmb{Z})$ and $P(X|\pmb{Z})$, with parameters $\pmb{\theta}^Y$ and $\pmb{\theta}^X$ respectively. They are score functions and thus $E\left[ \frac{\partial}{\partial\pmb{\theta}}l_Y(Y_i, \pmb{Z}_i; \pmb{\theta}^Y) \right]=\pmb{0}$ and $E\left[ \frac{\partial}{\partial\pmb{\theta}}l_X(Y_i, \pmb{Z}_i; \pmb{\theta}^X) \right]=\pmb{0}$. The functions $\psi(Y_i, X_i, \pmb{Z}_i;\pmb{\theta})$ will be different for each statistic.\\
~\\
\textbf{For statistic $\pmb{T}_1$}: for $T_1=\Gamma(\hat{P}) - \Gamma(\hat{P}_0)$, we define $\pmb{\theta}^T = (\pi_{11},...,\pi_{1,t-1},\pi_{1,t}, ~~~\pi_{21},...,\pi_{2,t-1},\pi_{2t}, ~~~..., ~~~\pi_{s1},...,\pi_{s,t-1})$. Note that $\pi_{s,t}$ is not included because it can be expressed from the rest of $\pi_{s,t} = 1-\sum_{j,j}\pi_{ij}$.\\
The corresponding function $\psi$ is:
$$
\begin{aligned}
	\psi^T(Y_i,X_i,\pmb{Z}_i;\pmb{\theta}^T) = (I_{\{Y_i=1,X_i=1\}}-\pi_{11}, ~~...,~~I_{\{Y_i=s,X_i=t-1\}}-\pi_{s,t-1})
\end{aligned}
$$
where $I_a$ is an indicator functon of event $a$. $E[\psi(Y_i,X_i,\pmb{Z}_i;\pmb{\theta}^T)]=\pmb{0}$ because for each element of vector $\psi(Y_i,X_i,\pmb{Z}_i;\pmb{\theta}^T)$, say $\psi_{11}$ we have: $E[\psi_{11}] = (1-\pi_{11})\cdot \pi_{11} + (0-\pi_{11})\cdot (1-\pi_{11})=0$. Let $g(\pmb{\theta}) = \Gamma(P) - \Gamma(P_0)$. Then $g(\pmb{\theta})=0$ under the null, and $T_1=g(\hat{\pmb{\theta}})$.\\
~\\
\textbf{For statistic $\pmb{T}_2$}: Let's denote $w_1=E(Y_{i,res})$, $w_2=E(X_{i,res})$, $w_3=E(Y_{i,res}, X_{i,res})$, $w_4=E(Y^2_{i,res})$, and $w_5=E(X^2_{i,res})$. Then $\pmb{\theta}^T = (w_1, w_2, w_3, w_4, w_5)$, and the corresponding $\psi$ is:
$$
\begin{aligned}
	\psi^T(Y_i,X_i,\pmb{Z}_i;\pmb{\theta}) = (Y_{i,res}-w_1, ~X_{i,res}-w_2,  ~Y_{i,res}X_{i,res}-w_3, ~Y^2_{i,res}-w_4, ~X^2_{i,res}-w_5)
\end{aligned}
$$
By definition $E[\psi(Y_i,X_i,\pmb{Z}_i;\pmb{\theta}^T)]=\pmb{0}$. Solving the equation $\sum_i\psi(Y_i,X_i,\pmb{Z}_i;\pmb{\theta}^T)]=\pmb{0}$, we have $\hat{w}_1=\frac{1}{n}\sum_i y_{i,res}$, $\hat{w}_2=\frac{1}{n}\sum_i x_{i,res}$, $\hat{w}_3=\frac{1}{n}\sum_i y_{i,res}x_{i,res}$, $\hat{w}_4=\frac{1}{n}\sum_i y^2_{i,res}$, and $\hat{w}_5=\frac{1}{n}\sum_i x^2_{i,res}$. Let $g(\pmb{\theta}) = (w_3 - w_1w_2)/\sqrt{(w_4-w_1^2)(w_5-w_2^2)}=cor(Y_{i,res},X_{i,res})$ (?!). Because it is a correlation, $g(\pmb{\theta})=0$ under the null, and $T_2=g(\hat{\pmb{\theta}})$\\
~\\
\textbf{For statistic $\pmb{T}_3$}: $T_3 = \frac{1}{n}\sum_i (\hat{C}_i - \hat{D}_i)$, parameter $\pmb{\theta}^T$ is a scalar, $\theta^T = E(C_i - D_i)$, which is $0$ under the null. The corresponding function $\psi$ is:
$$
\begin{aligned}
	\psi^T(Y_i,X_i,\pmb{Z}_i;\pmb{\theta}) = (C_i - D_i) - \theta^T
\end{aligned}
$$
By definition $E[\psi(Y_i,X_i,\pmb{Z}_i;\theta)]=\pmb{0}$. Let $g(\pmb{\theta}) = \theta^T$. Then $g(\pmb{\theta}) = 0$ under the null, and $T_3=g(\hat{\pmb{\theta}}) $. The delta method is not needed here because $\hat{\sigma}^2$ is simply the last element of the diagonal of $\hat{V}(\pmb{\theta})$ (because $g(\cdot)$ is an identity function).
\section{A new residual for ordinal outcomes, Li and Shepherd, 2012 \cite{li2012new}}
\textbf{Desirable properties of residuals for ordial variables:}
\begin{enumerate}[(a)]
	\item results in only one value per subject;
	\item reflects the overall direction of the observed value compared with the fitted value;
	\item is monotonic with respect to the observed value for those with the same covariates;
	\item has a range of possible values that is symmetric about zero;
	\item has expectation zero;
	\item should perserve order without assigning arbitrary scores to the categories.
\end{enumerate}
~\\

\textbf{Definition:}\\
Consider a set of of $s$ ordered categories, $S=\{1,...,s\}$, with order $1<...<s$. For a category $y$ in $S$ and a distribution $F$ over $S$, we define a \textbf{\textit{residual}}:
$$
\begin{aligned}
	r(y,F)=E\{sign(y,Y)\} = pr(y>Y) - pr(y<Y),
\end{aligned}
$$
where $Y$ is a random variable with distribution $F$, and $sign(a,b)$ is $-1$, $0$ and $1$ for $a<b$, $a=b$, and $a>b$, respectively. Suppose we fit an ordinal outcome variable $Y$ on covariates $Z$. For subject $i$, let $Y_i$ be the outcome and let $F_{Z_i;\theta}$ be the distribution of $Y_i$ given covariates $Z_i$ under a model with parameter $\theta$. We define:
$$
\begin{aligned}
	R_i = r(Y_i,F_{Z_i;\theta})
\end{aligned}
$$
Give data$(y_i, z_i)$ and a fitted model with parameter estimates $\hat{\theta}$, the residual for subject $i$ is $\hat{r}_i = r(Y_i,F_{Z_i;\theta})$. Notice that $\hat{r}_i$ is not a realization of $R_i$, but of the random variable $\hat{R}_i = r(Y_i,F_{Z_i;\hat{\theta}})$. If $\hat{\theta} \overset{P}{\rightarrow} \theta$, then $F_{Z_i;\hat{\theta}} \rightarrow F_{Z_i;\theta}$, and $\hat{R}_i \overset{D}{\rightarrow} R_i$. Therefore, moment properties of $R_i$ are applicable to $\hat{R}_i$ asymptotically.\\
~\\
\textbf{Properties of $r(y, F)$, $R_i$, and $\hat{r}_i$:}
\begin{enumerate}[\textit{Property} 1.]
	\item $-1\leq r(1, F) \leq ... \leq r(s,F) \leq 1$;
	\item when $s=1$, $r(1,F) = r(1,1)=0$, where $F=(1)$ is a point mass;
	\item $r(j, F) = -r(s-j+1, G)$, where $G=(q_1,...q_s) = (p_s,...,p_1)$ with $q_j=p_{s-j+1}$;
	\item as functions of $(\gamma_1, ..., \gamma_{s-1})$, $\partial r(j,F)/\partial \gamma_j = \partial r(j+1, F)/\partial \gamma_j$.
	\item \textit{Branching Property}. If two adjacent categories $t$ and $t+1$ are merged, with distribution $G=(q_1,...q_{s-1})$, where $q_j=p_j$ for $j<t$, $q_t=p_t + p_{t+1}$ and $q_j=p_{t+1}$ for $j>1$, then $r(j, G)=$
	$$
	\begin{aligned}
		r(j, G)&= r(j, F), ~~~j<t\\
		r(j, G)&= r(j+1, F), ~~~j>t\\
		r(j, G)&= \{ p_t r(t, F) + p_{t+1}(t+1, F) \}/(p_t + p_{t+1}), ~~~j=t\\
		\end{aligned}
		$$
	\item The function $r(j,F)$ satisfies Properties 2, 3(for $s=2$), 4(for $s=2$), and 5 if and only if $r(j,F) = c(\gamma_{j-1} - (1-\gamma_j))$, where $c$ is an arbitrary constant.
	\item $r(j, F)/2 = ridit_j - 1/2$, where $ridit_j = \gamma_{j-1} + p_j/2 = (\gamma_{j-1} + \gamma_j)/2$ and the mean ridit is $1/2$.
	\item $E(R) = 0$;
	\item $var(R) = (1-\sum^s_{j=1} p_j^3)/3$, or alternatively, $var(R) = \sum^s_{j=1} p_j\gamma_{j-1}\gamma_j$. This property provides alternative ways of calculating $var(R)$ and implies that $var(R)$ does not depend on the order of the probabilities.
	\item when $p_1=...=p_s = 1/s$ $var(R)$ reaches its maximum $(1-1/s^2)/3$
	\item Consider a random sample of $n$ subjects, with $n_j$ subjects in category $j$ $(j=1,...,s)$. Their empirical distribution is $\hat{F}=(\hat{p}_1, ... \hat{p}_s)$, where $\hat{p}_j = n_j/n$, with cumulative probabilities $\hat{\gamma}_j = \sum_{k\leq j} n_k/n$. With no covariates, the predictor of all subjects is a constant. Then $r_j = r(j,\hat{F}) = \hat{\gamma}_{j-1} + \hat{\gamma_j} - 1 = \left( \sum_{k<j}2n_k + n_j - n \right)/n$. If we rank these subjects, their midrank is $rank_j = \sum{k<j} n_k + (n_j + 1)/2$. The following property holds:
	$$
	\begin{aligned}
		rank_j = T(r_j) = (n/2)\cdot r_j + (n+1)/2
	\end{aligned}
	$$
The function $T(r_j)$ can be viewed as a translation from the residual scale to the rank scale. This property implies that when there are no covariates, statistics $T_2$ (see reference \cite{li2010test}) is Spearman's rank correlation coefficient between $X$ and $Y$. When covariates exist, $T(r_j)$ yields adjusted ranks of the subjects, and $T_2$ can be interpreted as an adjusted rank correlation.
	\item We now focus on models for an ordinal outcome $Y$ on covariates $Z$ with parameters $\theta$. For subject $i$ $(i=1,...,n)$ and category $j$ $(j=1,...,s)$, let $\gamma_{i,j} = pr(Y_i\leq j|Z_i;\theta)$; for convenience, we define $\gamma_{i,0} = 0$. Let $p_{i,j} = pr(Y_i=j|Z_i)=\gamma_{i,j} - \gamma_{i, j-1}$. The following two moment properties hold:
	$$
	\begin{aligned}
		E{R_i|Z_i;\theta} = 0;
	\end{aligned}
	$$
	\item $var(R_i|Z_i;\theta) = E(R_i^2|Z_i;\theta) =\sum_{j=1}^s R_{i,j}^2 p_{i,j}= \sum_{j=1}^s (\gamma_{i,j-1} + \gamma_{i,j} - 1)^2 p_{i,j}$
	
	\item Let $\hat{p}_{i,j}$ and $\hat{\gamma}_{i,j}$ be the maximum likelihood estimates of $p_{i,j}$ and $\hat{\gamma}_{i,j}$. Then $\hat{r}_i = \hat{\gamma}_{i, y_i-1} + \hat{\gamma}_{i, y_i}$ - 1. The variance of $R_i$ can be consistently estimated by inserting these estimates into Property 13, $\hat{var}(R_i) = var(R_i|z_i;\hat{\theta}) =  \sum_{j=1}^s (\hat{\gamma}_{i,j-1} + \hat{\gamma}_{i,j} - 1)^2 \hat{p}_{i,j}$. One could therefore calculate a standardized residual as $\hat{r}_i/\{\hat{var}(R_i)\}^{1/2}$, which for binary $Y$ is the Pearson residual $(I_{\{Y_i=2\}} - \hat{p}_{i,2})(\hat{p}_{i,1}\hat{p}{i,2})^{-1/2}$.
	
	\item Now we consider a proportional odds model (??? McCullagh, 1980), $logit\{pr(Y \leq j|Z)\}=\alpha_j + Z^T\beta~(j=1,...,s-1)$ with parameters $\theta = (\alpha_1, ...\alpha_{s-1}, \beta)$. Under this model, our residuals are related to score residuals (??? Therneau el al., 1990, ). Let $l_i$ be the loglikelihood for subject $i$, and $U_i=U_i(\theta) = \partial l_i/\partial \theta$ be the score function. Because $E_{\theta}(U_i) = 0$, $U_i$ is called the score residual. For proportional odds models, the following two properties hold:
	$$
	\begin{aligned}
		\hat{r}_i = - \sum_{j=1}^{s-1}(\partial l_i/\partial \alpha_j)|_{\hat{\theta}}, ~~~where ~\hat{\theta}~is~MLE
	\end{aligned}
	$$
	\item $\sum_{i=1}^n \hat{r}_i = 0$.
\textbf{Also}, under certain assumptions (see the paper), these residuals are equivalent to comparing $(\gamma_{i,j-1} + \gamma_{i,j})/2$ with $1/2$ on the probability scale.
	
\end{enumerate}
~\\

\section{Regression Models for correlated Failure Time Data. Page 302, \cite{kalbfleisch2011statistical}}
This summary is based on book of Kalbfleisch and Prentice \emph{The Statistical Analysis of Failure Time Data}

\subsection{Notations}
If $\tilde{T}_i$ is the failure time for subject $i$ and $\pmb{x}_i$ is a covariance vector, \textbf{the joint survivor function} is:
	$$
	\begin{aligned}
		F(t_1,...,t_m;x) = P(\tilde{T}_1>t_1,...,\tilde{T}_m > t_m;x)
	\end{aligned}
	$$

Marginal \textbf{first-order }hazard function is:
	$$
	\begin{aligned}
		 \lambda_j(dt_j;x) = P(y_j\leq \tilde{T}_j<t_j + dt_j|\tilde{T}_j \geq t_j,x) ~~~j=1,2,...,m
	\end{aligned}
	$$

Marginal \textbf{first-order }hazard function is:
	$$
	\begin{aligned}
		 \lambda_j(dt_j,dt_k;x) = P(y_j\leq \tilde{T}_j<t_j + dt_j,~y_k\leq \tilde{T}_k<t_k + dt_k~|~\tilde{T}_j \geq t_j,~\tilde{T}_k \geq t_k,~x) ~~~j=1,2,...,m
	\end{aligned}
	$$

\subsection{Representation and estimation of the bivariate survivor function}
Some notations:
	$$
	\begin{aligned}
		F(t_1,t_2) &= P(\tilde{T}_1>t_1,\tilde{T}_2 > t_2)\\
		\Lambda(t_1,t_2) &= \int_0^{t_1} \int_0^{t2} \lambda(du_1, du_2)
	\end{aligned}
	$$
$\Lambda$ \textbf{does not} determine the survivor function over a follow-up region $[0,t_1] \times [0,t_2]$. However, together with the marginal cumulative hazard functions $\Lambda_1$ and $\Lambda_2$ it leads to a \emph{convenient} survivor function estimate with \emph{attractive properties}.\\
First, let's introduce some notations:
\[
F(dt_1, dt_2)=
\begin{bmatrix}
    F^{(11)}(t_1, t_2)dt_1 dt_2 & \text{If F is absolutely continuous in both} \\
        & \text{components at }(dt_1, dt_2) \\
    F^{(10)}(t_1, \Delta t_2)dt_1 & \text{If F is continuous in its first, but not its }\\
        & \text{second argument at }(dt_1, dt_2) \\
    F^{(01)}(\Delta t_1, t_2)dt_2 & \text{If F is continuous in its second, but not its} \\
        & \text{first argument at }(dt_1, dt_2) \\
    F(\Delta t_1, \Delta t_2) & \text{If} (t_1, t_2)\text{ is a mass point}\\
\end{bmatrix}
\]
	$$
	\begin{aligned}
		F(\Delta t_1, \Delta t_2) &= F(t_1^-, t_2^-) - F(t_1^-, t_2) - F(t_1, t_2^-) + F(t_1, t_2)\\
		F(t_1, t_2) &= \int_{t_1}^{\infty} \int_{t_2}^{\infty} F(du_1, du_2)
	\end{aligned}
	$$
\textbf{Bivariate hazard function:}
	$$
	\begin{aligned}
		\Lambda(dt_1, dt_2) &= F(dt_1, dt_2)/F(t_1^-, t_2^-) ~~or\\
		\Lambda(\Delta t_1, \Delta t_2) &= F(\Delta t_1, \Delta t_2)/F(t_1^-, t_2^-)
	\end{aligned}
	$$
It is important to notice that $F(t_1,t_2)$ is not expressed only through $\Lambda$ (see page 309 of Kalbfleisch and Prentice) for the complicated formula. But there is also an easier expression, if we substitute $F(\Delta t_1, \Delta t_2) = \Lambda(\Delta t_1, \Delta t_2) \cdot F(t_1^-, t_2^-)$ into the expression for $F(\Delta t_1, \Delta t_2)$, we get:
	$$
	\begin{aligned}
		F(t_1, t_2) &= F(t_1^-, t_2) + F(t_1, t_2^-) - F(t_1^-, t_2^-)[1-\Lambda(\Delta_1, \Delta_2)]\\
	\end{aligned}
	$$
The above expression is a recursive procedure for estimating $F(t_1, t_2)$ at all failure time points starting with Kaplan-Meier estimates of $F(t_1, 0) = F_1(t_1)$ and $F(t_2, 0) = F_2(t_2)$ given the estimate of $\Lambda(\Delta_1, \Delta_2)$.\\
Dabrowska (1988) inserts a simple estimator $\hat{\Lambda}(\Delta_1, \Delta_2)$, which is a ratio of the number of double failures at $(t_1,t_2)$ to the number of pairs at risk at that point:
	$$
	\begin{aligned}
		\hat{\Lambda}(\Delta t_1, \Delta t_2) = \#(T_1=t_1, T_2=t_2, \delta_1=\delta_2=1)/\#(T_1\geq t_1, T_2\geq t_2)
	\end{aligned}
	$$
This estimator has a reasonably good performance in moderate sample sizes, but is inefficient because of poor correspondence between $\hat{\Lambda}$ and Kaplan-Meier estimates $\hat{F}_1$ and $\hat{F}_2$. Dabrowska (1988) and Prentice\&Cai (1992) tried to improve it by estimating double failure hazard rates in a manner that aknowledge the empirical rate (see the above $\hat{\Lambda}$) and the amount of mass assigned by the Kaplan-Meier marginals along $\tilde{T}_1=t_1$ and along $\tilde{T}_2=t_2$ that remains to be assigned at $(t_1,t_2)$. These estimators were developed using certain product integral and Peano serries representation for the ratio $F(t_1,t_2)/(F_1(t_1)F_2(t_2))$.\\
\emph{Prentice-Cai's} and \emph{Dabrowska's} estimator are given by:
	$$
	\begin{aligned}
		&\hat{\pmb{\Lambda}}_{\pmb{P\&C}}(\Delta t_1, \Delta t_2) 
		=\hat{\Lambda}(\Delta t_1, \Delta t_2) + \hat{L}_1(\Delta t_1, 0)\left[ \hat{L}_2(t_1^-, \Delta t_2)  - \hat{\Lambda}_2(t_1^-, \Delta t_2)\right] + \hat{L}_2(0, \Delta t_2)\left[ \hat{L}_1(\Delta t_1, t_2^-)  - \hat{\Lambda}_1(\Delta t_1, t_2^-)\right]\\
		&\hat{\pmb{\Lambda}}_{\pmb{D}}(\Delta t_1, \Delta t_2) 
		= \hat{L}_1(\Delta t_1, t_2^-)\hat{\Lambda}_2(t_1^-, \Delta t_2)  +  \frac{(1 - \hat{L}_1(\Delta t_1, t_2^-))(1 - \hat{\Lambda}_2(t_1^-, \Delta t_2))}{ (1-\hat{\Lambda}_1(\Delta t_1, t_2^-))  (1-\hat{\Lambda}_2(t_1^-, \Delta t_2))  } \hat{\Lambda}(\Delta t_1, \Delta t_2) - \hat{\Lambda}_1(\Delta t_1, t_2^-)\hat{\Lambda}_2(t_1^-, \Delta t_2)
	\end{aligned}
	$$
Where
	$$
	\begin{aligned}
		&\hat{\Lambda}(\Delta t_1, \Delta t_2) = \#(T_1=t_1, T_2=t_2, \delta_1=\delta_2=1)/\#(T_1\geq t_1, T_2\geq t_2)\\
		&\hat{L}_1(\Delta t_1, t_2^-) = \hat{F}_1(\Delta t_1, t_2^-)/\hat{L}_1(t_1^-, t_2^-)\\
		&\hat{\Lambda}_1(\Delta t_1, t_2^-) = \#(T_1=t_1, \delta_1=1, T_2\geq t_2)/\#(T_1\geq t_1, T_2\geq t_2)
	\end{aligned}
	$$

\section{Simulation of Survival Data}

\subsection{Simulation of bivariate Gumbel distribution using copulas}

Here we use a method described in the book of \cite{nelsen2007introduction}. The method utilizes \emph{copulas} in order to simulate bivariate data. We describe this method using example of \emph{Gumbel} distribution (p.23):
	$$
	\begin{aligned}
    &H_{\theta}(x, y) = 1-e^{-x} - e^{-y} + e^{-(x+y+\theta xy)}, ~~~ x\geq 0, ~~~y\geq 0\\
    &H(x,y) = 0, ~~~otherwise
	\end{aligned}
	$$
According to \cite{nelsen2007introduction}, the marginal distributions of this function are exponentials. Indeed, let's find the $f_X(x)$ of this distribution:

$$
\begin{aligned}
    f(x,y) & = \frac{\partial^2}{\partial x \partial y}H_{\theta}(x, y) = \frac{\partial}{\partial y}(e^{-x} - (1+\theta y)e^{-(x+y+\theta xy)}) =\\
      & = e^{-(x+y+\theta xy)}((1+\theta y)(1+\theta x) - \theta)\\
    f_X(x) & = \int_0^{\inf} f(x,y)dy = e^{-x}(1 + \theta x)\int_0^{\inf}e^{-y(1+\theta x)} (1 + \theta y) dy  - \theta e^{-x} \int_0^{\inf} e^{-y(1+\theta x)} dy = \\
    & = e^{-x}\cancel{(1 + \theta x)} \frac{1+\theta x + \theta}{(1+\theta x)^{\cancel{2}}} + e^{-x}\frac{-\theta}{1 + \theta x} = e^{-x}
\end{aligned}
$$

\textbf{Note} that in order to simulate this distribution, it is not necessary to use copulas, we do this in order to practice the technique of using copulas. We can simulate the marginal distribution for $X$ first ($X \sim f_X(x) = e^{-x}$). Then we can use conditional distribution function for $Y$ in order to sample $Y$: 


$$
\begin{aligned}
F_{Y|X}(y|X=x) &= \frac{F(X\leq x,Y=y)}{f_X(x)}= \frac{\frac{\partial F(x,y)}{\partial x}}{f_X(x)} =\frac{ e^{-x} - (1+\theta y)e^{-(x+y+\theta xy)}}{e^{-x}} =\\
&= 1 - (1+\theta y)e^{-y(1+\theta x)}.
\end{aligned}
$$


 We first simulate $x_i \sim e^{-x}$ and $t_i \sim Unif(t)$ and then for each $x_i$ and $t_i$ we find $y_i$ by numerically inverting expression: $t_i = 1 - (1+\theta y_i)e^{-y_i(1+\theta x_i)}$ \\
~\\
If we do decide to use copulas, next, we need to find the quasi-inverses (see definition in the book) are $F^{(-1)}(u) = -ln (1-u)$ and $G^{(-1)}(v) = -ln (1-v)$, hense the corresponding copula is:
  $$
  \begin{aligned}
      C_{\theta}(u, v) = u+v-1+(1-u)(1-v)e^{-\theta ln(1-u) ln(1-v)}
  \end{aligned}
  $$
The simulations can be performed based on the algorithm described in the book on pages 41-42. We first have to find $\frac{\partial C_{\theta}(u, v)}{\partial u}$:

  $$
  \begin{aligned}
     \frac{\partial C_{\theta}(u, v)}{\partial u} &= 1-(1-v)e^{-\theta ln(1-u) ln(1-v)} + \cancel{(1-u)}(1-v)e^{-\theta ln(1-u) ln(1-v)} \theta \frac{ln(1-v)}{\cancel{1-u}} =\\
     &= 1 - (1-v)e^{-\theta ln(1-u) ln(1-v)}(1 + \theta ln(1-v))
  \end{aligned}
  $$
\textbf{Note}: $\frac{\partial C_{\theta}(u, v)}{\partial u} = P(U= u, V\leq v)$, which is simply a conditional \emph{cdf} for $V$. Let's denote $t=\frac{\partial C_{\theta}(u, v)}{\partial u}$.\\
We next simulate independently $(t_1, t_2, ..., )$, and $(u_1, u_2, ...)$, which are both uniform variables. Next, we and proceed in the following manner. For each values of $t_i$  $u_i$, we numerically invert function:
  $$
  \begin{aligned}
      t_i = 1 - (1-v)e^{-\theta ln(1-u_i) ln(1-v)}(1 + \theta ln(1-v))
  \end{aligned}
  $$
and get a value of $v=v_i$. Then we use the quasi-inverses $F^{(-1)}(u) = -ln (1-u)$ and $G^{(-1)}(v) = -ln (1-v)$ in order to back transform $u_i$ and $v_i$ into $x_i$ and $y_i$: $x_i = -ln (1-u_i)$ and $y_i = -ln (1-v_i)$.

\subsection{Figuring out \emph{Spearman's} correlation of bivariate Gumbel distribution}
According to \cite{nelsen2007introduction}, for continuous distributions, \emph{Spearman's} correlation can be expressed through \emph{copula} as:
	$$
	\begin{aligned}
		 \rho_S = 12{\int\int}_{I^2}C(u, v)dudv - 3
	\end{aligned}
	$$
  
Therefore, for Gumbel distribution we have:
	$$
	\begin{aligned}
		 \rho_S &= 12\int\int_{I^2}\left[ u+v-1+(1-u)(1-v)e^{-\theta ln(1-u) ln(1-v)} \right]dudv - 3 =\\
     &= 12\int\int_{I^2}\left[ u+v-1 \right]dudv +  12\int\int_{I^2} (1-u)(1-v)e^{-\theta ln(1-u) ln(1-v)} dudv - 3\\
     &= 12\int_0^1\left[ \frac{1}{2} + (v-1) \right]dv -  12\int_0^1(1-v)\int_0^1 (1-u)^{-\theta ln(1-v) + 1} d(1-u)dv - 3 = \\
     &= 0  - 12\int_0^1(1-v)  \left[ \left. \frac{(1-u)^{-\theta ln(1-v) + 2}}{-\theta ln(1-v) + 2}\right|_0^1 \right]  dv - 3 = \\
     &= 12\int_0^1\frac{1-v}{-\theta ln(1-v) + 2}  dv - 3
	\end{aligned}
	$$
As far as I know, the integral above is \emph{non-elementary} and cannot be expressed algebraically, therefore, in order to find it, we can use numeric integration, for example R-function \texttt{integrate($\cdot$)}. We keep in mind that according to \cite{nelsen2007introduction}, $0 \leq \theta \leq 1$.

\subsection{Figuring out \emph{Kendall's} $\tau$ of bivariate Gumbel distribution}
According to \cite{nelsen2007introduction}, for continuous distributions, \emph{Spearman's} correlation can be expressed through \emph{copula} as:
	$$
	\begin{aligned}
		 \rho_S = 4{\int\int}_{I^2}C(u, v)dC(u, v) - 1
	\end{aligned}
	$$
Where $dC(u, v) = \frac{\partial^2 C(u,v)}{\partial u\partial v} = $
Therefore:
	$$
	\begin{aligned}
		 \rho_S &= 4\int\int_{I^2}\left[ u+v-1+(1-u)(1-v)e^{-\theta ln(1-u) ln(1-v)} \right]dudv - 1 =\\
     &= 12\int\int_{I^2}\left[ u+v-1 \right]dudv +  12\int\int_{I^2} (1-u)(1-v)e^{-\theta ln(1-u) ln(1-v)} dudv - 3\\
     &= 12\int_0^1\left[ \frac{1}{2} + (v-1) \right]dv -  12\int_0^1(1-v)\int_0^1 (1-u)^{-\theta ln(1-v) + 1} d(1-u)dv - 3 = \\
     &= 0  - 12\int_0^1(1-v)  \left[ \left. \frac{(1-u)^{-\theta ln(1-v) + 2}}{-\theta ln(1-v) + 2}\right|_0^1 \right]  dv - 3 = \\
     &= 12\int_0^1\frac{1-v}{-\theta ln(1-v) + 2}  dv - 3
	\end{aligned}
	$$

\section{PSR (probability scale residuals) for censored data}
% Terms for residuals:
% \begin{itemize}
% 	\item \textbf{OMER}-observed minus expected residuals: $E(y-Y^*) = y - \hat{y}$
% 	\item \textbf{PSR}-probability scale residuals: $E{sign(y,Y^*)} = pr(Y^* < y) - pr(Y^* > y)$
% \end{itemize}
PSR are defined as $r(t,F^*) = E\{sign(y,Y^*)\} = pr(Y^* < y) - pr(Y^* > y) = F^*(t-) - (1-F^*(y)) = F^*(t-) - 1 + F^*(t)$.\\
Since we do not always observe $t$, the PSR must be defined in terms of $y$ (time to event or censoring) and $\delta$ (if the event happened or not). \\
\\
\textbf{By definition}, $r(y,F^*, \delta) = r(t,F^*|\Delta=\delta)$.\\
If $\delta = 1$, then $t=y$ and $r(y,F^*, \delta=1) = F^*(y-) - 1 + F^*(y)$. \\
If $\delta = 0$, then $t$ is unknown, except that it occurs some time after the censoring time $y$, so $r(t,F^*, \delta=0) = E\{r(t,F^*)|T^*>y\}$. Let's find it:
	$$
	\begin{aligned}
		E\{r(t,F^*)|T^*>y\} &= \frac{\int_{t>y}r(t,F^*)dF^*(t)}{1-F^*(t)}= \frac{\int_{t>y}(F^*(t-) - 1 + F^*(t))dF^*(t)}{1-F^*(y)}= \frac{\int_{t>y}F^*(t-)dF^*(t) - \int_{t>y}(1 - F^*(t))dF^*(t)}{1-F^*(y)}=\\
		&= \frac{\int_{t>y}\int_{s<t}dF^*(s)dF^*(t) - \int_{t>y}\int_{s>t}dF^*(s)dF^*(t)}{1-F^*(y)}= \frac{\int_{t>y}\int_{s<t}dF^*(s)dF^*(t) - \int\int_{y<t<s}dF^*(s)dF^*(t)}{1-F^*(y)}=\\
		&= \frac{\left[\int\int_{s<y<t}dF^*(s)dF^*(t) + \int\int_{y<s<t}dF^*(s)dF^*(t)\right] - \int\int_{y<t<s}dF^*(s)dF^*(t)}{1-F^*(y)}\\
	\end{aligned}
	$$
Because $t$ and $s$ are symmetric, $\int\int_{y<s<t}dF^*(s)dF^*(t) = \int\int_{y<t<s}dF^*(s)dF^*(t)$, so we have:
	$$
	\begin{aligned}
		E\{r(t,F^*)|T^*>y\} &= \frac{\left[\int\int_{s<y<t}dF^*(s)dF^*(t) + \cancel{\int\int_{y<s<t}dF^*(s)dF^*(t)}\right] - \cancel{\int\int_{y<t<s}dF^*(s)dF^*(t)}}{1-F^*(y)}\\
		&= \frac{\int\int_{s<y<t}dF^*(s)dF^*(t)}{1-F^*(y)} = \frac{(1-F^*(y))\int_{s<y}dF^*(s)}{1-F^*(t)} = \frac{\cancel{(1-F^*(y))}F^*(y)}{\cancel{1-F^*(y)}} =\\
		&=F^*(y)
	\end{aligned}
	$$
	
Therefore for sensored data:
	$$
	\begin{aligned}
		r(y, F^*) &= F^*(y) + F^*(y-) - 1,~~~&\delta = 1 \\
		r(y, F^*) &= F^*(y) ,~~~&\delta = 0 \\
	\end{aligned}
	$$
or $r(y, F^*) = F^*(y) + \delta((F^*(y-) - 1)$.

\subsection{Spearman correlation and PSR correlation for continuous censored outcome}
Here we show that if we define Spearman correlation for continuous and censored outcomes a certain way, we then Spearman correlation and PSR correlation for ontinuous censored outcome are equivalent.\\
~\\
Spearman correlation for continuous outcome can be defined as $cor(F_S(s), F_T(t))$ (reference ???). In case of censored outcomes however, we don't always know $s$ and $t$. This means that we have to redefine Spearman correlation in a different way. In the following section we use $S$, $C$, $\delta=I(S\leq C)$, $X = min(S, C)$ for the first outcome and $T$, $D$, $\epsilon=I(T\leq D)$, $Y = min(T, D)$ for the second.\\
Let's modify the definition of the Spearman correlation in the following way. When $\delta = 1$ or $S\leq C$, we have that $X = min(S, C)=S$ and instead of $F_S(s)$ we can take $F_S(x)$. When $\delta = 0$ or $S > C$, we have that $X = min(S, C)=C$ and instead of $F_S(s)$ we can take $E_S[F_S(s)|S>x]$. Let's compute $E_S[F_S(s)|S>x]$:
	$$
	\begin{aligned}
		E_S[F_S(x)|S>x] &= \frac{1}{1-F_S(x)}\int_{(x, \infty)} F_S(s)dF_S(s) = \frac{1}{1-F_S(x)} \left.\frac{ [F_S(s)]^2}{2}\right|_{(x, \infty)} = \frac{1-[F_S(x-)]^2}{2(1-F_S(x))} \\
	\end{aligned}
	$$
Since we are focusing now on the continuous case, we can assume that $F_S(x)=F_S(x-)$ therefore we have
	$$
	\begin{aligned}
		E_S[F_S(x)|S>X] &=\frac{1-[F_S(x)]^2}{2(1-F_S(x))}=\frac{\cancel{(1-F_S(x))}(1+F_S(x))}{2\cancel{(1-F_S(x))}}  = \frac{1+F_S(x)}{2} \\
	\end{aligned}
	$$
So, instead of looking at the Spearman correlation as correlation of $F_S(s)$ and $F_T(t)$, we redefine correlation as:
	$$
	\begin{aligned}
		Spearman~correlation~&=~ cor\left(  \delta F_S(x) + (1-\delta) \frac{1+F_S(x)}{2},~\epsilon F_T(y) + (1-\epsilon) \frac{1+F_T(y)}{2}  \right)\\
		&=~ cor\left(  \frac{1+\delta}{2}F_S(x) +  \frac{1-\delta}{2},~\frac{1+\epsilon}{2}F_T(y) +  \frac{1-\epsilon}{2}  \right)\\
		&=~ cor\left(  (1+\delta)F_S(x) +  1-\delta,~(1+\epsilon)F_T(y) +  1-\epsilon  \right)\\
	\end{aligned}
	$$
It is easy to check that in the continous case, the above expression is the same as: 
	$$
	\begin{aligned}
		PSR~correlation~&=~ cor\left(  (1+\delta)F_S(x) -\delta,~(1+\epsilon)F_T(y) -\epsilon  \right)\\
	\end{aligned}
	$$

\subsection{Spearman correlation and PSR correlation for discrete censored outcome}
For discrete case, Spearman correlation can be defined as (???):
	$$
	\begin{aligned}
		cor\left( \frac{F_S(s) + F_S(s-)}{2},~\frac{F_T(t) + F_T(t-)}{2} \right)\\
	\end{aligned}
	$$
So let's find $E_S[F_S(x)|S>x]$ for the discrete case. First (just to get started) let's derive $E_S[F_S(x)]$. We denote $Pr\left\{S=s\right\}$ as $P_s$:
	$$
	\begin{aligned}
		E_S[F_S(x)] &= \sum_{i=0}^{\infty}F_S(i)P_i = \sum_{i=0}^{\infty}\left( \sum_{k=0}^{i}P_k \right)P_i =\\
		&= (P_0 P_0) + (P_0 P_1 + P_1 P_1) + (P_0 P_2 + P_1 P_2 + P_2 P_2) + ...= \\
		&= P_0^2 + P_1^2 + P_2^2 + ... + P_0 P_1 + P_0 P_2 + P_2 P_1 + ... =\frac{1}{2}(P_0 + P_1 + P_2 + ...)^2 + \frac{1}{2}(P_0^2 + P_1^2 + P_2^2 + ...) =\\
		&= \frac{1}{2} + \frac{1}{2}\sum_{i=0}^{\infty}P_i^2
	\end{aligned}
	$$
Now we are equiped to compute $E_S[F_S(x)|S>x]$:
	$$
	\begin{aligned}
		E_S[F_S(x)|S>x] &= \frac{1}{1-F_S(x)} \sum_{i=x+1}^{\infty}F_S(i)P_i = \frac{1}{1-F_S(x)}\sum_{i=x+1}^{\infty}\left( \sum_{k=0}^{i}P_k \right)P_i =\\
		 &= \frac{1}{1-F_S(x)}\left[ \sum_{i=0}^{\infty}\left( \sum_{k=0}^{i}P_k \right)P_i - \sum_{i=0}^{x}\left( \sum_{k=0}^{i}P_k \right)P_i   \right] = \\
		 &= \frac{1}{1-F_S(x)}\left[ \frac{1}{2} + \frac{1}{2}\sum_{i=0}^{\infty}P_i^2 -  \frac{1}{2} \left\{ \left(\sum_{i=0}^{x}P_i\right)^2 + \sum_{i=0}^{x}P_i^2  \right\}    \right] = \\
		 &= \frac{ 1 - F_S^2(x) + \sum_{i=x+1}^{\infty}P_i^2 }{2(1-F_S(x))}\\
	\end{aligned}
	$$

Our next challenge is to compute $E_S[F_S(x-)|S>x]$. We assume that $F_S(0-)=0$:
	$$
	\begin{aligned}
		E_S[F_S(x-)|S>x] &= \frac{1}{1-F_S(x)} \sum_{i=x+1}^{\infty}F_S(i-1)P_i = \frac{1}{1-F_S(x)}\sum_{i=x+1}^{\infty}\left( \sum_{k=0}^{i-1}P_k \right)P_i =\\
		 &= \frac{1}{1-F_S(x)}\left[ \sum_{i=0}^{\infty}\left( \sum_{k=0}^{i-1}P_k \right)P_i - \sum_{i=0}^{x}\left( \sum_{k=0}^{i-1}P_k \right)P_i   \right] = \\
		 &= \frac{1}{1-F_S(x)}\left[ \frac{1}{2} - \frac{1}{2}\sum_{i=0}^{\infty}P_i^2 -  \frac{1}{2} \left\{ \left(\sum_{i=0}^{x}P_i\right)^2 - \sum_{i=0}^{x}P_i^2  \right\}    \right] = \\
		 &= \frac{ 1 - F_S^2(x) - \sum_{i=x+1}^{\infty}P_i^2 }{2(1-F_S(x))}\\
	\end{aligned}
	$$
Now, let's recall that in the discrete case each Spearman correlation component is $\frac{F_S(s) + F_S(s-)}{2}$, so let's compute it for the case when $\delta=0$:
	$$
	\begin{aligned}
	  \frac{E_S[F_S(x)|S>x] + E_S[F_S(x-)|S>x]}{2} &= \frac{1}{2}\left[\frac{ 1 - F_S^2(x) + \sum_{i=x+1}^{\infty}P_i^2 }{2(1-F_S(x))}   +   \frac{ 1 - F_S^2(x) - \sum_{i=x+1}^{\infty}P_i^2 }{2(1-F_S(x))}\right]=\\
		&= \frac{1 - F_S^2(x)}{2(1-F_S(x))}= \frac{1 + F_S(x)}{2}\\
	\end{aligned}
	$$
As a result, we have:
	$$
	\begin{aligned}
		Spearman~correlation~&=~ cor\left(  \delta\frac{F_S(x) + F_S(x-)}{2} + (1-\delta) \frac{1 + F_S(x)}{2}, ~\epsilon\frac{F_T(y) + F_T(y-)}{2} + (1-\epsilon) \frac{1 + F_T(y)}{2} \right)=\\
		&=~ cor\left(  \delta( F_S(x) + F_S(x-)) + (1-\delta) (1 + F_S(x)), ~\epsilon (F_T(y) + F_T(y-)) + (1-\epsilon) (1 + F_T(y)) \right)=\\
		&=~ cor\left( \cancel{\delta F_S(x)} + \delta F_S(x-) +  1 + F_S(x) -\delta - \cancel{\delta F_S(x)}, ~\cancel{\epsilon F_T(y)} + \epsilon F_T(y-) + 1 + F_T(y)-\epsilon -\cancel{\epsilon F_T(y)}   \right)=\\
		&=~ cor\left( F_S(x) + \delta F_S(x-) +  1 -\delta , ~ F_T(y) + \epsilon F_T(y-) + 1 -\epsilon   \right)\\
	\end{aligned}
	$$
It can be easily shown that the above expression is the same as correlation of \emph{PSR}:
	$$
	\begin{aligned}
		PSR~correlation~&=~ cor\left(  F_S(x) + \delta F_S(x-) -\delta,~F_T(y) + \epsilon F_T(y-) -\epsilon  \right)\\
	\end{aligned}
	$$

\subsection{Spearman correlation and PSR based on a different definition.}
Up to now, in order to prove the equivalence of \emph{Spearman} corrlation and the correltation of the PSR we have been using the following defition of \emph{Spearman} correlation: $cov(F_S(s), F_T(t))$. In this section, we show that if we start with the original definition, $E(sign(S_0-S_1)sign(T_0-T_2))$, we will get the same result. Let's see why. Keeping in mind that $S_1 \perp T_2$ we can write (see \cite{lui2015covariate}):
	$$
	\begin{aligned}
		E(sign(S_0-S_1)sign(T_0-T_2)) &= E_{(S_0,T_0)}\left\{ E_{S_1,T_2|S_0,T_0}[sign(S_0-S_1)sign(T_0-T_2) |(S_0,T_0)] \right\} = \\
		  &= E_{(S_0,T_0)}\left\{ E_{S_1|S_0}[sign(S_0-S_1) |S_0]  E_{T_2|T_0}[sign(T_0-T_2) |T_0] \right\} = \\
		  &= E_{(S_0,T_0)}\left\{ [ P(S_1 < S_0) - P(S_1 > S_0) ] \cdot  [P(T_2 < T_0) - P(T_2 > T_0)] \right\}=\\
		  &= E_{(S_0,T_0)}\left\{ [ F_S(S_0^-) - 1 + F_S(S_0) ] \cdot  [F_T(T_0^-) - 1 + F_T(T_0))] \right\}=\\
		  &= E\left\{ PSR_S \cdot PSR_Y \right\}\\
	\end{aligned}
	$$
When censoring takes place, we don't observe $S_0$, but we still obsere $x_0$, so similarly to the previous section, instead of $PSR = [ F_S(S_0^-) - 1 + F_S(S_0) ]$ we can take $PSR = E[ F_S(S_0^-) - 1 + F_S(S_0) | S_0>x_0]$.

Above is the covariance of PSR. In order to scale it to $[-1, 1]$ we need to find the variance of PSR under censoring. \cite{shepherd2016probability} showed that for continuous censored data, the variance of PSR is $E[R^2] = \frac{1}{3} - \frac{1}{3}E_C[(1-F(C))^3]$, where $C$ is time to censoring. Because:

	$$
	\begin{aligned}
		E_C[(1-F(C))^3] &= E_C[1-3F(C) + 3F^2{C} - F^3(C))^3] = 1 - 3 \left. \frac{F^2(C)}{2}  \right|_{(0, \infty)} + 3 \left. \frac{F^3(C)}{3}  \right|_{(0, \infty)} -  \left. \frac{F^4(C)}{4}  \right|_{(0, \infty)} =\\
    &= \frac{1}{4}
	\end{aligned}
	$$
We have that \emph{Spearman's}, $\rho_S$, defined for censored data and the correlation of PSR are the same if \emph{Spearman's} $\rho_S$ is scaled to $[-1, 1]$, or:

	$$
	\begin{aligned}
		c^* \cdot \rho_S &= cor(PSR_X, PSR_Y)\\
    & where~c^*=\frac{1}{4}
	\end{aligned}
	$$
  
\subsection{PSR covariance of perfectly correlated variables under censoring}
\subsubsection{Positive correlation under censoring, $T\equiv S$}
First, let's introduce some notations. Times to event for sons and fathers are $S$ and $T$ respectively with the corresponding distribution functions of $F(S)$ and $G(T)$. We fix two points $s_0$ and $t_0$ and find the covariance of PRS under the assumption of perfect dependence expressed in the functional form $S =\nu( T)$, where $\nu(\cdot)$ is a strictly increasing function. For brevity, we denote $F(s)=U$, $G(t)=V$, $F(s_0)=U_0$, and $G(t_0)=V_0$.\\
~\\

Next, let's show how two distribution functions are related in case of perfect correlation. The following statements are somewhat straight forward:
	$$
	\begin{aligned}
    &Because~T = \nu(S)\\
    &We~have~S = \nu^{-1}(T)\\
    &F_T(t) = P(T \leq t) = P\left(g(S) \leq t\right) = P(S\leq g^{-1}(t)) = P(S\leq s) = F_S(s) 
	\end{aligned}
	$$
Therefore, as long as $s=h(t)$, we have: $U=F(s) = G(t) = V$
% Following the logic in the previous two section, we know that we have to take an integral on a curve. We define this curve in space of distribution functions (with support of $[0, 1] \times [0, 1]$). Because in general, $F_T(t) = F_S(s)$, we still have the same normalizing constant:
%   $$
%   \begin{aligned}
%     C_{norm} = \int_{0}^{1} \sqrt{(F_T'(t))^2 + (F_S'(s))^2}dt = \int_{0}^{1} \sqrt{2}dt = \sqrt{2}
%   \end{aligned}
%   $$
% which will cancel out with a similar term that appears because of integration on the curve, so we get the same expressions as before for any strictly decreasing or strictly increasing $g(\cdot)$.
% ~\\

Now, let's find the marginal and joint distributions of $U$ and $V$. First, let's point out that the marginal distributions of $U$ and $V$ are uniform. Next, from \cite{nelsen2007introduction} we know that the joint \emph{cdf} of these two variables is $min(U, V)$. This does not help to derive the joint \emph{pdf} though (at least it did not help me). Let's denote the joint \emph{pdf} of $(U, V)$ as $h(u, v)$. In order to figure our joint \emph{pdf}, let's use conditionality principle: $h(u, v)=h_{U|V=v}(u|v)\cdot h_V(v)$. We know that $U,V \sim Unif(0, 1)$. The conditional \emph{pdf}, $h_{U|V=v}$, equals $1$ when $u=v$ and $0$ when $u\neq v$, or, $h_{U|V=v} = C_{norm}I_{[0\leq U \leq 1]}\cdot I_{[U=v]}$, where $C_{norm}$ is a normalizing constant. This is a strange function that can be visualized as a piece of a plain, perpendicular to the $(U,V)$ plain. It stretches from $(U,V) = (0, 0)$ to $(U,V) = (1,1)$. We will call this piece of plain a \emph{segment mass}. It has a height of 1 and a length of $\sqrt{2}$, therefore, in order for it to sum up to 1, it has to be divided by $\sqrt{2}$ or $C_{norm}=\sqrt{2}$.\\
~\\
Because $h(u, v)$ is a \emph{segment mass}, in order to find covariance, we cannot just take a regular integral, we have to take an integral on a curve $U=V$, or $V(t)=t$ and $U(t) = t$. We know that the integral over a curve has to have a term $\sqrt{\left(\frac{dU(t)}{dt}\right)^2 + \left(\frac{dV(t)}{dt}\right)^2}$. This term equals to $\sqrt{1^2 + 1^2} = \sqrt{2}$, therefore two terms of $\sqrt{2}$ cancel out in the following integrals.\\
~\\
Suppose now that we need to find the covariance of PSR for a fixed $(u_0, v_0)$. Let's assume a continuous case, $U = U^-$, $V = V^-$:
	$$
	\begin{aligned}
    cov_{(u_0, v_0)}(U, V) &= \underbrace{\int_0^{v_0}\int_0^{u_0} (2U - 1)(2V - 1)dH}_{\text{A}} + \underbrace{\int_{v_0}^{1}\int_0^{u_0} (2U - 1)VdH}_{B} + \\
    &+\underbrace{\int_{0}^{v_0}\int_{u_0}^1 (2V - 1)UdH}_{C} + \underbrace{\int_{v_0}^{1}\int_{u_0}^1 VUdH}_{D}
	\end{aligned}
	$$
Now let's compute each integral separately. Because $V(t)=t$, $U(t) = t$, and $u_0 < v_0$ integral $A$ on curve $V(t)=t,~U(t) = t$ is:
	$$
	\begin{aligned}
    A &= \int_0^{u_0} (2t - 1)^2 dt = \frac{1}{6}\left\{  (2u_0 - 1)^3 - (0-1)^3\right\}=\\
     &= \frac{1}{6}\left\{  (2u_0 - 1)^3 + 1\right\}
	\end{aligned}
	$$
Because $u_0 < v_0$, set $[0, u_0]\times [v_0, 1]$ does not intersects with curve $V(t)=t$, $U(t) = t$ and therefore integral $B$ is 0.\\
Because $u_0 < v_0$,  integral $C$ becomes:
	$$
	\begin{aligned}
    C &= \int_{u_0}^{v_0} (2t - 1)u_0 dt = \frac{u_0}{2} \int_{u_0}^{v_0}  (2t - 1) d(2t - 1) = \\
    &= \frac{u_0}{4}\left\{  (2v_0 - 1)^2 -(2u_0 - 1)^2 \right\}
	\end{aligned}
	$$

Because $u_0 < v_0$,  integral $D$ becomes:
	$$
	\begin{aligned}
    D &= \int_{v_0}^{1} v_0 u_0 dt = u_0 v_0 (1 - v_0)
	\end{aligned}
	$$

Summing up all four integrals and taking into account the symmetry, we have an expression for \emph{Spearman} correlation of two identical variables, $S = \nu( T)$ conditional on the censoring point $(u_0, v_0)$:
	$$
	\begin{aligned}
    \rho_{S = \nu( T)}(u_0, v_0)/3 &= \frac{1}{6}\left\{  (2u_0 - 1)^3 +1\right\} +  \frac{u_0}{4}\left\{  (2v_0 - 1)^2 -(2u_0 - 1)^2 \right\} + u_0 v_0 (1 - v_0), ~~~~when~u_0 < v_0\\
    &= \frac{1}{6}\left\{  (2v_0 - 1)^3 +1\right\} +  \frac{v_0}{4}\left\{  (2u_0 - 1)^2 -(2v_0 - 1)^2 \right\} + v_0 u_0 (1 - u_0), ~~~~when~u_0 \geq v_0
	\end{aligned}
	$$


\subsubsection{Negative correlation under censoring, $T\equiv 1-S$}
The derivations are very similar to the previous section with some differences. When $S = \nu( T)$ and $\nu(\cdot)$ is strictly decreasing, we have: $U = 1-V^-$. Assuming a continuous case we have a \emph{segment-mass} again, but this time it stretches from $(U,V) = (1,0)$ to $(U,V) = (0,1)$. And again, in order for it to sum up to 1, it has to be divided by $\sqrt{2}$.\\
~\\
It can be shown that when $T\equiv 1-S$ the \emph{cdf}'s are related in the same way: $V = 1-U$. Because are taking an integral on a curve $U=V$ again ($U(t) = t$ and $V(t)=1-t$) and because we remember that integral on a curve has to have a term $\sqrt{\left(\frac{dU(t)}{dt}\right)^2 + \left(\frac{-dV(t)}{dt}\right)^2}$. This term again equals to $\sqrt{1^2 + (-1)^2} = \sqrt{2}$, therefore two terms of $\sqrt{2}$ cancel out in the following integrals.\\
~\\
Suppose now that we need to find the covariance of PSR for a fixed $(u_0, v_0)$. Similar to the previous case, let's compute each integral separately. First, we assume a case when $u_0 + v_0 < 1$. Let's also note that $PSR_U = 2U(t) - 1 = 2t - 1$ and $PSR_V = 2V(t) - 1 = -2t + 1$.

Because $U(t) = t$ and $V(t)=1 - U(t) = 1 - t$, and $u_0 + v_0 < 1$ integral $A$ is 0.\\
~\\
Integral $B$ is the following: 
	$$
	\begin{aligned}
    B &= \int_{1 - v_0}^1 (-2t + 1)u_0 dt = -\frac{u_0}{2}\int_{1 - v_0}^1 (2t - 1) d(2t - 1)= -\frac{u_0}{4}\left\{  (2v_0 - 1)^2 -1\right\} = \\
    &=u_0v_0(u_0 - 1)
	\end{aligned}
	$$
Integral $C$ is very similar and keeping in mind that it should be symmetric to $B$:
	$$
	\begin{aligned}
    C &=\int_{0}^{u_0} (2t - 1)u_0 dt  = u_0v_0(v_0 - 1)
	\end{aligned}
	$$
Integral $D$ is the following: 
	$$
	\begin{aligned}
    D &= \int_{u_0}^{1-v_0} u_0v_0 dt =u_0v_0(1 - u_0 - v_0)
	\end{aligned}
	$$

Now, let's assume a case when $u_0 + v_0 > 1$. 

Integral $A$ is
	$$
	\begin{aligned}
    A &= \int_{1-v_0}^{u_0} -(2t - 1)^2 dt = -\frac{1}{2} \int_{1-v_0}^{u_0} (2t - 1)^2 d(2t-1) = \\
    &= -\frac{1}{6}\left\{  (2u_0 - 1)^3 + (2v_0 - 1)^3 \right\}
	\end{aligned}
	$$

Integrals $B$ and $C$ are:
	$$
	\begin{aligned}
    B &= \int_{u_0}^1 (-2t + 1)u_0 dt =u_0^2(u_0 - 1)\\
    C &= \int_{0}^{1-v_0} (2t - 1)u_0 dt =v_0^2(v_0 - 1)\\
	\end{aligned}
	$$
Integral $D$ is $0$.

We did not consider a case when $u_0+v_0 = 1$ because it is very similar to the previous cases, except that both $A=0$ and $D=0$.\\
Summing up all four integrals, we have an expression for \emph{Spearman} for case $S\nu(T)$, where $\nu(\cdot)$ is strictly decreasing conditional on the censoring point $(u_0, v_0)$:
	$$
	\begin{aligned}
    \rho_{S\equiv T}(u_0, v_0)/3 &= -u_0v_0, &when~u_0 + v_0 < 1\\
    &= -\frac{1}{6}\left\{  (2u_0 - 1)^3 + (2v_0 - 1)^3 \right\} + u_0^2(u_0 - 1) + v_0^2(v_0 - 1), &when~u_0 + v_0 \geq 1
	\end{aligned}
	$$

%\subsubsection{Perfect correlation under censoring, general case}

\subsection{Spearman correlation and cov of PSR for uncensored data, yet another proof.}
Let's assume that $(T^a_1, T^a_2)$ is fixed and $T^b_1 \perp T^c_2$. Let's also assume that $S(\cdot)$ is a survival function and $F(\cdot)$ is a distribution function, then
	$$
	\begin{aligned}
		P(concordance) &- P(discordance) = \\
      &=P(T^a_1>T^b_1, T^a_2 > T^c_2 ~or ~T^a_1<T^b_1, T^a_2 < T^c_2) - P(T^a_1>T^b_1, T^a_2 < T^c_2 ~or ~T^a_1<T^b_1, T^a_2 > T^c_2)=\\
      &=[F(T^{a-}_1)F(T^{a-}_2) + S(T^a_1) S(T^a_2)] - [ F(T^{a-}_1) S(T^a_2) + S(T^a_1)F(T^{a-}_2)]=\\
      &=[S(T^a_1) - F(T^{a-}_1)][S(T^{a}_2) -F(T^{a-}_2)]=\\
      &=PSR(T^a_1)\cdot PSR(T^a_2)\\
	\end{aligned}
	$$

\subsubsection{Local version of \emph{Spearman's} $\rho$}
Inspired by \cite{fan2000class} we derive a local version of \emph{Spearman's} $\rho$, $\rho_S(x, y)$ using the following notations. We pick a pair $(X_{0},Y_{0})$ and two independent variables $(X_{1}$ and $Y_{2})$
  $$
  \begin{aligned}
    P\{ &X_{0}\wedge X_{1} = x,~Y_{0}\wedge Y_{2} = y \} =\\
     &=P\{ X_{0} > x, X_{1} = x, Y_{0} > y, Y_{2} = y \}
      + P\{ X_{0} > x, X_{1} = x, Y_{0}= y, Y_{2} = y \}
        + P\{ X_{0} > x, X_{1} = x, Y_{0}= y, Y_{2} > y \} + \\
     &+P\{ X_{0} = x, X_{1} = x, Y_{0} > y, Y_{2} = y \}
      + P\{ X_{0} = x, X_{1} = x, Y_{0}= y, Y_{2} = y \}
        + P\{ X_{0} = x, X_{1} = x, Y_{0}= y, Y_{2} > y \} + \\
     &+P\{ X_{0} = x, X_{1} > x, Y_{0} > y, Y_{2} = y \}
      + P\{ X_{0} = x, X_{1} > x, Y_{0}= y, Y_{2} = y \}
        + P\{ X_{0} = x, X_{1} > x, Y_{0}= y, Y_{2} > y \}\\
  \end{aligned}
  $$
Let's further denote marginal \emph{pdf} and \emph{cdf} of $X$ as $F(dx)$ and $F(x)$, and marginal \emph{pdf} and \emph{cdf} of $Y$ as $G(dy)$ and $G(y)$
Because $(X_{1}, Y_{1})$, $X_{1}$, and $Y_{2})$ are independent, we have:
  $$
  \begin{aligned}
P\{ &X_{1}\wedge X_{2} = x,~Y_{1}\wedge Y_{2} = y \} =\\
      &= S(x,y)F(dx)G(dy) +  S(x,dy)F(dx)G(dy) + S(x,dy)F(dx)G(y) + \\
      &~~~~+ S(dx,y)F(dx)G(dy) +  S(dx,dy)F(dx)G(dy) + S(dx,dy)F(dx)G(y) + \\
      &~~~~+ S(dx,y)F(x)G(dy) +  S(dx,dy)F(x)G(dy) + S(dx,dy)F(x)G(y) = \\
      &= S(x,y)F(dx)G(dy) +  S(x,dy)F(dx)(G(dy)+G(y))  + S(dx,y)G(dy)(F(dx) + F(x)) +\\ &~~~~+S(dx,y)((F(dx) + F(x))(G(dy)+G(y))) = \\
      &= S(x,y)F(dx)G(dy) +  S(x,dy)F(dx)G(y^-)  + S(dx,y)G(dy)F(x^-) +S(dx,y)F(x^-)G(y^-) = \\
      &= \{S(x^-,y^-)-S(x^-,y)-S(x,y^-) + S(x,y)\}F(dx)G(dy) +  \\
      &~~~~+\{S(x,y^-) - S(x,y)\}F(dx)G(y^-)  +\\
      &~~~~ +\{S(x^-,y) - S(x,y)\}G(dy)F(x^-) +\\
      &~~~~+S(dx,y)F(x^-)G(y^-) = \\
      &~~~~~~~~~~ skipping~some~algebra...\\
      &= S(x^-,y^-)F(dx)G(dy) +  \\
      &~~~~+S(x^-,dy)F(dx)(G(y^-) - G(dx))  +\\
      &~~~~+S(dx,y^-)(F(x^-) - F(dx))G(dy)  +\\
      &~~~~+S(dx,dy)(F(x^-) - F(dx))(G(y^-) - G(dx))\\
	\end{aligned}
	$$
Now, let's find the joint distribution of concordance and discordance and of $X_{1}\wedge X_{2} = x,~Y_{1}\wedge Y_{2} = y$:
	$$
	\begin{aligned}
		P\{ (X_0>X_1, Y_0>Y_2) &\cap (X_{1}\wedge X_{2} = x,~Y_{1}\wedge Y_{2} = y) \} =P\{ X_0>x, X_1=x, Y_0>y, Y_2=y)\} = S(x, y)F(dx)G(dy) \\
		P\{ (X_0<X_1, Y_0<Y_2) &\cap (X_{1}\wedge X_{2} = x,~Y_{1}\wedge Y_{2} = y) \} =P\{ X_0=x, X_1>x, Y_0=y, Y_2>y)\} = S(dx, dy)F(x)G(dy) \\
		P\{ (X_0>X_1, Y_0<Y_2) &\cap (X_{1}\wedge X_{2} = x,~Y_{1}\wedge Y_{2} = y) \} =P\{ X_0>x, X_1=x, Y_0=y, Y_2>y)\} = S(x, dy)F(dx)G(y) \\
		P\{ (X_0<X_1, Y_0>Y_2) &\cap (X_{1}\wedge X_{2} = x,~Y_{1}\wedge Y_{2} = y) \} =P\{ X_0=x, X_1>x, Y_0>y, Y_2=y)\} = S(dx, y)F(x)G(dy) \\
	\end{aligned}
	$$
 Therefore 
	$$
	\begin{aligned}
		P\{&[Concord. - Discord.] \cap (X_{1}\wedge X_{2} = x,~Y_{1}\wedge Y_{2} = y) \} = \\
    &~~~= \{S(dx,dy) + S(x^-,y^-)-S(x^-,dy)-S(dx,y^-)\}F(dx)G(dy) +  \\
      &~~~~+S(dx,dy)(F(x^-) - F(dx))(G(y^-) - G(dy))+\\
      &~~~~ -\{S(x^-,dy) - S(dx,dy)\}F(dy)(G(y^-) - G(dy)) +\\
      &~~~~ -\{S(dx,y^-) - S(dx,dy)\}(F(x^-) - F(dx))G(dy) +\\
      &~~~~~~~~~~ skipping~some~algebra...\\
      &= S(x^-,y^-)F(dx)G(dy) -  \\
      &~~~~-S(x^-,dy)F(dx)G(y^-)  -\\
      &~~~~-S(dx,y^-)F(x^-)G(dy)  +\\
      &~~~~+S(dx,dy)F(x^-)G(y^-)\\
	\end{aligned}
	$$
  
  
Therefore, for the discrete case, we have:
{\scriptsize{
  $$
  \begin{aligned}
    \rho(x, y) &=P\{concord| X_{1}\wedge X_{2} = x,~Y_{1}\wedge Y_{2} = y\} - P\{discord|X_{1}\wedge X_{2} = x,~Y_{1}\wedge Y_{2} = y\} =\\
    &= \frac{P\{concord, X_{1}\wedge X_{2} = x,~Y_{1}\wedge Y_{2} = y\} - P\{discord,X_{1}\wedge X_{2} = x,~Y_{1}\wedge Y_{2} = y\}}{P\{X_{1}\wedge X_{2} = x,~Y_{1}\wedge Y_{2} = y\}} = \\ 
    &=\frac{S(x^-,y^-)F(dx)G(dy) -S(x^-,dy)F(dx)G(y^-)-S(dx,y^-)F(x^-)G(dy) +S(dx,dy)F(x^-)G(y^-)}{S(x^-,y^-)F(dx)G(dy)+S(x^-,dy)F(dx)(G(y^-) - G(dx))+S(dx,y^-)(F(x^-) - F(dx))G(dy)+S(dx,dy)(F(x^-) - F(dx))(G(y^-) - G(dx))}=\\
    &~~~~~~~let's~denote: ~h(x, y) = P\{X_{0}\wedge X_{1} = x,~Y_{0}\wedge Y_{2} = y\}~~~ then:\\
    &=\frac{S(x^-,y^-)F(dx)G(dy) -S(x^-,dy)F(dx)G(y^-)-S(dx,y^-)F(x^-)G(dy) +S(dx,dy)F(x^-)G(y^-)}{h(x, y)}=\\
  \end{aligned}
  $$
}}
If we weight $\rho(x, y)$ by $\frac{h(x, y)}{\int_0^x \int_0^y h(x, y)}$ and integrate the above expression over $[0,x]\times[0,y]$, we get a \emph{finate region Spearman} $\rho$:
	$$
	\begin{aligned}
		\mathcal{R}(x, y) &=P\{concord| X_{1}\wedge X_{2} \leq x,~Y_{1}\wedge Y_{2} \leq y\} - P\{discord|X_{1}\wedge X_{2} \leq x,~Y_{1}\wedge Y_{2} \leq y\}=\\
    &=\frac{\int_0^x \int_0^y [S(x^-,y^-)F(dx)G(dy) -S(x^-,dy)F(dx)G(y^-)-S(dx,y^-)F(x^-)G(dy) +S(dx,dy)F(x^-)G(y^-)]}{\int_0^x \int_0^y h(x, y)}
	\end{aligned}
	$$

Let's write the numerator and the denominator in terms of hazards using the following notations:
	$$
	\begin{aligned}
		&\Lambda_{11}(dx, dy) = \frac{S(dx,dy)}{S(x^-,y^-)};~~~\Lambda_{10}(dx, y^-) = \frac{S(dx,y^-)}{S(x^-,y^-)};~~~\Lambda_{01}(x^-, dy) = \frac{S(x^-,dy)}{S(x^-,y^-)}\\
    &\lambda_X = \frac{F(dx)}{F(x^-)}; ~~~~~~\lambda_Y = \frac{G(dy)}{G(y^-)}
	\end{aligned}
	$$

	$$
	\begin{aligned}
		\mathcal{R}(x, y) &=P\{concord| X_{1}\wedge X_{2} \leq x,~Y_{1}\wedge Y_{2} \leq y\} - P\{discord|X_{1}\wedge X_{2} \leq x,~Y_{1}\wedge Y_{2} \leq y\}=\\
    &=\frac{\int_0^x \int_0^y S(x^-,y^-)F(x^-)G(y^-)[\lambda_X \lambda_Y -\Lambda_{01}\lambda_X-\Lambda_{10}\lambda_Y +\Lambda_{11}]}{\int_0^x \int_0^y S(x^-,y^-)F(x^-)G(y^-)[\lambda_X \lambda_Y+\Lambda_{01}\lambda_X(1 - \lambda_Y)+\Lambda_{10}(1 - \lambda_X)\lambda_Y+\Lambda_{11}(1 - \lambda_X)(1 - \lambda_Y)]}
	\end{aligned}
	$$

We can estimate this expression using bivariate cumulative hazard estimates used by \cite{dabrowska1988kaplan}:

	$$
	\begin{aligned}
    \hat{\Lambda}_{11}(x, y) &= \int_0^x \int_0^y \frac{\hat{K}_1(du, dv)}{\hat{H}(u^-, v^-)}\\
    \hat{\Lambda}_{10}(x, y) &= -\int_0^x \frac{\hat{K}_2(du, y)}{\hat{H}(u^-, y)} ~~(not~sure~about~the~minus)\\ 
    \hat{\Lambda}_{01}(x, y) &= -\int_0^y \frac{\hat{K}_3(x, dv)}{\hat{H}(x, v^-)}\\
	\end{aligned}
	$$
  In the above, the minus is in the definition of Dabrowska: $\Lambda_{10} =- \frac{S(dx,y^-)}{S(x^-,y^-)}$ (instead of what is writtent here $\Lambda_{10} = \frac{S(dx,y^-)}{S(x^-,y^-)}$). Then she adds minus also to the sample versions of $\Lambda_{01}$ and $\Lambda_{10}$, see the above. I did not use minus, and I probably should. In any case it cancels out, I think.
	$$
	\begin{aligned}
    \hat{H}(x, y)&= \frac{1}{n}\sum I(X_i > x, Y_i > y)\\
    \hat{K}_1(x, y)&= \frac{1}{n}\sum I(X_i > x, Y_i > y, \delta_X = 1, \delta_Y = 1)\\
    \hat{K}_2(x, y)&= \frac{1}{n}\sum I(X_i > x, Y_i > y, \delta_X = 1)\\
    \hat{K}_3(x, y)&= \frac{1}{n}\sum I(X_i > x, Y_i > y, \delta_Y = 1)\\
	\end{aligned}
	$$
It is easier to use notations of \cite{kalbfleisch2011statistical}:
	$$
	\begin{aligned}
    \hat{\Lambda}_{11}(\Delta x, \Delta y) &= \frac{Number~of~(X=x, Y=y, ~~\delta_X = \delta_y = 1)}{Number~of~(X\geq x, Y\geq y)}\\
    \hat{\Lambda}_{10}(\Delta x, ~~ y) &= \frac{Number~of~(X=x,~~\delta_X = 1)}{Number~of~(X\geq x, Y\geq y)} = ? =\frac{Number~of~(X=x, Y\geq y,~~\delta_X = 1)}{Number~of~(X\geq x, Y\geq y)}\\
    \hat{\Lambda}_{01}(x, ~~ \Delta y) &= \frac{Number~of~(Y=y,~~\delta_Y = 1)}{Number~of~(X\geq x, Y\geq y)} = ? = \frac{Number~of~(X\geq x, Y=y,~~\delta_Y = 1)}{Number~of~(X\geq x, Y\geq y)}\\
	\end{aligned}
	$$
I put the question marks in the above expressions because the book does not say that these are equal, but I am assuming they mean it because the denominator has these conditions ($X\geq x$, $ Y\geq y$).

\subsection{Rank tests for independence for bivariate censored data (Dabrowska) \cite{dabrowska1986rank}}
For null hypothesis $H_0:~F=F_1 F_2$, the following \emph{linear rank} statistic can be used
	$$
	\begin{aligned}
		 &\sum_{n=1}^N a_1(R_{1n}, \delta_{1n})\cdot a_2(R_{2n}, \delta_{2n}), ~i=1,2\\
		 &~~~~~ a_i(j,d) = E\mathcal{J}(U_{(j)}, d)\prod_{k=1}^{N_i} m_{ik}(1-U_{(k)}^{a_{ik}}
	\end{aligned}
	$$
Where: 
	$$
	\begin{aligned}
		 R_in &= \#\{m:~Y_{im} \leq Y_{in}, \delta_{im}=1   \}\\
		 a_{ik} &= \# \{n: ~R_{in}=k,~\delta_{in}=0\}\\
		 m_{ik} &= \# \{n: ~R_{in}\geq k\}\\
		 \delta_{im}&=1,~if~subject ~m~from~group~i~has~an~event\\
		 U_{(k)}~&independent ~ordered~sample~from~Uniform~distribution\\
		 \mathcal{J}(u,d) ~&is~such~that:~~\int_0^u \mathcal{J}_i(v,1)dv = -(1-u)\mathcal{J}_i(u,0)
	\end{aligned}
	$$
The author then says that the choice of $\mathcal{J}_i(u,d) =d-(1+d)u$ corresponds to the cencored-data version of the Spearman test. She also notes that the exact scores are hard to compute, so an approximate test statistics  can be used:
	$$
	\begin{aligned}
		 S_n = \sum_{n=1}^N \mathcal{J}_1( \hat{F}_1, \delta_{1n}) \cdot \mathcal{J}_2( \hat{F}_2, \delta_{2n})
	\end{aligned}
	$$
Substituting $\mathcal{J}_i(u,d) =d-(1+d)u$ into the above expression, we get:
	$$
	\begin{aligned}
		 S_n &= \sum_{n=1}^N (\delta_{1n} - (1+\delta_{1n})\hat{F}_1(Y_{1n}))\cdot (\delta_{2n} - (1+\delta_{2n})\hat{F}_1(Y_{2n})) =\\
		 &= \sum_{n=1}^N (-\hat{F}_1(Y_{1n}) + \delta_{1n}(1-\hat{F}_1(Y_{1n})))\cdot (-\hat{F}_2(Y_{2n}) + \delta_{2n}(1-\hat{F}_2(Y_{2n})))\\
	\end{aligned}
	$$
Where $\hat{F}_i$ are estimators close to the usual \emph{Kaplan-Meier} estimators of the marginal \emph{CDF}'s. If we look at the above expression closely:
	$$
	\begin{aligned}
		 S_n &= \sum_{n=1}^N (\hat{F}_1 - \delta_{1n}(1-\hat{F}_1))\cdot (\hat{F}_2 - \delta_{2n}(1-\hat{F}_2))\\
	\end{aligned}
	$$
it reminds us of $E[PSR_1 \cdot PSR_2]$, where $PSR(x, \delta) = F(x)-\delta(1-F(x))$ is a probability scale residual for \emph{continuous case}.

\subsection{Ding, Wang, 2012: Testing Independence for Bivariate Current Status Data, ref???}
Suggested two statistics for testing independence for bivariate \emph{current status data}. One is based on counts (how many events, non-events, and so forth). The other one is:
	$$
	\begin{aligned}
		 E[cov(\delta_1, \delta_2)|C_1,C_2] = E\left\{ [\delta_1 - F_1(C_1)][\delta_2 - F_2(C_2)]  \right\}\\
	\end{aligned}
	$$
Which is \textbf{exactly} covariance of \textbf{PSR} for \emph{current status data}.

\subsection{Zhang, 2008: Inference on the association measure for bivariate survival data with hybrind censoring and applications to an HIV study, ref???}
The author uses archimedian copulas and Kendall's $\tau$ to assess association of bivariate survival data with hybrid censoring.

\subsection{P. Gaduthol Sankaran, B. Abraham, A Alphonsa Antony, 2006: A dependence measure for bivariate failure time data, ref???}
Meaning to improve over the measures given by Fan et al. (1998) and Fan et al. (2000), the authors propose a \emph{covariance residual life function} (CVRL) as a dependence measure of bivariate failure time data:
	$$
	\begin{aligned}
		 C(t_1, t_2) &= M(t_1, t_2) - r_1(t_1, t_2) r_2(t_1, t_2) =\\
		  &=E[(T_1-t_1)(T_2-t_2)|T_1>t_1,T_2>t_2] - E[T_1-t_1|T_1>t_1,T_2>t_2]\cdot E[T_2-t_2|T_1>t_1,T_2>t_2]
	\end{aligned}
	$$
They note that this measure is in fact a weighted average of the Clayton's $\theta(t_1,t_2) = \frac{S(t_1,t_2)D_1D_2(S(t_1,t_2))}{D(S(t_1,t_2))D_2(S(t_1,t_2))}$ with a weight $\left[ \frac{D_1D_2(S(t_1,t_2))}{D_1(S(t_1,t_2))D_2(S(t_1,t_2))} \right]^{-1}$.

\subsection{My plan}
\begin{enumerate}[1)]
  \item look at the papers that have PRS and see if they use covariates
  \item look at all papers that are marked with stars
	\item Look who cited Cuzick's paper
	\item look cited Alvo and Cabilio, 1993:
	\item look into Shih \& Louis, 1996: \emph{who sites those and what this is about (Louis is apparently known)}.
	\item Read the Cuzick's paper in detail.
	\item Look who cited Betensky, Filkenstein, AIDS, 1999: paper
Tests of independence for bivariate survival data:
	\item Understand Qi's conditional, partial, and conditional-partial correlation
	\item Find out about Spearman test (unlike Spearman's correlation)
	\item fFind all papers that cite Dabrowska's paper on correlation
	\item Develop Spearman correlation for different $s$ and $t$: instead of having one number $\rho$ develop a new measure $\rho(s, t)$.
\end{enumerate}


%\section{Summary of: An Adjustment to Improve the Bivariate Survivor Function Repaired NPMLE. Moodie, Prentice, 2005 \cite{moodie2005adjustment}}

\subsection{Clayton, 1978 \cite{clayton1978model}}
Clayton suggested a \textit{a bivariate model for ordered pairs}. Suppose we are interested in onset of disease for fathers and sons. We want to know if they are associated. Following Cox, Clayton thought that similar to Cox proportional hazard idea, when $\lambda(s|y) = a(s)b(y)$, the hazard for a son to contract a disease in case of a binary covariate, would be $\lambda(s|y=1)/\lambda(s|y=0) = b(1)/b(0) = \theta$.

Let's first define the hazard ratio in case of two variables:


If $f(s,t)$ is a density of the age at which fathers ($t$) and sons ($s$) succomb to the disease, then the proposed model is:
	$$
	\begin{aligned}
		f(s,t)\int_s^{\infty}\int_t^{\infty} f(u, v)dvdu &= \theta \int_s^{\infty}f(u, t)du\int_t^{\infty}f(s, v)dv \\
	\end{aligned}
	$$
		or\\
	$$
	\begin{aligned}
		\theta &= \frac{f(s,t)\int_s^{\infty}\int_t^{\infty} f(u, v)dvdu} {\int_s^{\infty}f(u, t)du\int_t^{\infty}f(s, v)dv} = 
		\frac{ \frac{\partial^2 S(s, t)}{\partial s\partial t} \cdot S(s, t)}    {\frac{\partial S(s, t)}{\partial t} \frac{\partial S(s, t)}{\partial s}}
		&= \frac{f(s,t)/\frac{\partial S(s, t)}{\partial t}}  {\frac{\partial S(s, t)}{\partial s}/ S(s, t) } 
	\end{aligned}
	$$
We keep in mind that:
	$$
	\begin{aligned}
		f(s,t) &= &f(s|t)\cdot f(t)\\
		\frac{\partial S(s,t)}{\partial t} &= \int_s^{\infty}f(u, t)du = \int_s^{\infty}f(u|t)f_t(t)du = &S(s|t)\cdot f_t(t)\\
		\frac{\partial S(s,t)}{\partial s} &=...= S(t|s)\cdot f(s) = Pr(T>t|s)\cdot f(s) = Pr(T>t, s) = Pr(s,T>t) = Pr(s|T>t)S_t(t) = &f(s|T>t)\cdot S_t(t)\\
		S(s,t) &= Pr(S>s|T>t)\cdot Pr(T>t) = &S(s|T>t)\cdot S_t(t)
	\end{aligned}
	$$
By substituting these equations into the expression for $\theta$, we get:
	$$
	\begin{aligned}
		\theta &= \frac{f(s,t)/\frac{\partial S(s, t)}{\partial t}}  {\frac{\partial S(s, t)}{\partial s}/ S(s, t) } = \frac{  \frac{f(s|t)\cancel{f_t(t)}}{S(s|t)\cancel{f_t(t)}} } { \frac{f(s|T>t)\cdot \cancel{S_t(t)}}{S(s|T>t)\cdot \cancel{S_t(t)}}  }
		  = \frac{f(s|t)/S(s|t)}     {f(s|T>t) / S(s|T>t) }
		= \frac{\lambda_s(s|t) } {\lambda_s(s|T>t) }\\
	\end{aligned}
	$$
Because of the symmetry, we can write:
	$$
	\begin{aligned}
	  \theta = \frac{ \lambda_s(s_0|t=t_0)}{\lambda_s(s_0|t>t_0)} = \frac{ \lambda_t(t_0|s=s_0)}{\lambda_t(t_0|s>s_0)}
	\end{aligned}
	$$

The we have:
\begin{enumerate}[(i)]
	\item the joint survivor function:\\$\mathcal{F}(s, t) = \int_s^{\infty}\int_t^{\infty} f(u, v)dvdu$
	\item the hazard function for sons of fathers who survive until t, that is:\\
	$g(s;t)=\lim_{h\rightarrow 0} \frac{1}{h} P(T_1\in[s,s+h] | T_1\geq s, T_2 > t) =\frac{ \int_t^{\infty} f(s, v)dv} {\mathcal{F}(s, t)}$\\
	$g(s;t)=\frac{\partial}{\partial s}\{log\mathcal{F}(s,t)\}$
  
	\item the hazard function for fathers of sons who survive until s, that is:\\
	$h(t;s)=\lim_{h\rightarrow 0} \frac{1}{h} P(T_2\in[t,t+h] | T_1 > s, T_2 \geq t) = \frac{ \int_s^{\infty} f(u, t)du} {\mathcal{F}(s, t)}$\\
	$h(t;s)=\frac{\partial}{\partial t}\{log\mathcal{F}(s,t)\}$
  
	\item the bivariate failure rate:\\
	$l(t;s) =\lim_{(h_1,h_2)\rightarrow 0} \frac{1}{h_1 h_2} P(T_1\in[s,s+h_1], T_2\in[t,t+h_2] | T_1\geq s, T_2 \geq t) = \frac{f(s,t)}{\mathcal{F}(s,t)} $\\
  $l(t;s) = g(s;t) h(t;s) - \frac{ \partial^2 }{\partial s \partial t} \{-log\mathcal{F}(s,t)\}$
\end{enumerate}
After some complicated math, he derives:
	$$
	\begin{aligned}
		&\mathcal{F}(s,t) = [1 + (\theta - 1)\{a(s) + b(t)\}]^{-1/(\theta - 1)}\\
		&g(s;t) = \frac{a'(s)}{1 + (\theta - 1)\{a(s) + b(t)\}}\\
		&h(t;s) = \frac{b'(s)}{1 + (\theta - 1)\{a(s) + b(t)\}}\\
		&l(s,t) = \frac{\theta a'(s)b'(t)}{1+(\theta-1)(a(s) + b(t))^{2+1/(\theta-1)}}\\
		&f(s,t) = \frac{\theta a'(s)b'(t)}{1+(\theta-1)(a(s) + b(t))^{2}}
	\end{aligned}
	$$
Where $a(\cdot)$ and $b(\cdot)$ are nondecreasing (nuisance) functions with $a(0) = b(0)=0$.\\
\textbf{Interpretation:} when $\theta>1$, $g(s;t)$ is decreasing with respect to $t$ so that the age-specific rates for sons decrease with increased survival time of their fathers. Similarly, $h(t;s)$ decreases with respect to $s$. When $\theta=1$ and there is no asociation.\\
Parameter $\theta$ is estimated using maximum likelihood.

\subsection{Dabrowska, 1988 \cite{dabrowska1988kaplan}}
Dabrowska introduced defined a bivarate Kaplan-Meier estimator and proved its consistency (and probably something else). Her notations were used in the following papers:\\
Let $(\Omega, \mathcal{F}, P)$ be a probability space, and let $F(s,t) = P(T_1>s, T_2>t)$ be the corresponding joint survival function. By a bivariate cumulative hazard function, Dabrowska means a vector:
	$$
	\begin{aligned}
		\Lambda(s,t) &= (\Lambda_{10}(s,t), \Lambda_{01}(s,t), \Lambda_{11}(s,t))\\
		&where\\
		\Lambda_{11}(ds,dt) &= \frac{P(T_1 \in ds, T_2\in dt)}{P(T_1 \geq s, T_2 \geq t)} = \frac{F(ds, dt)}{F(s-, t-)}\\
		\Lambda_{10}(ds,t) &= \frac{P(T_1 \in ds, T_2 > dt)}{P(T_1 \geq ds, T_2 > t)} = \frac{-F(ds, t)}{F(s-, t)}~~~see ~the~paper~where ~it~is~F(s-, t-),~probably~a~typo\\ 
		\Lambda_{01}(s,dt) &= \frac{P(T_1 > s, T_2\in dt)}{P(T_1 > s, T_2 \geq t)} = \frac{-F(s, dt)}{F(s, t-)}\\
		&and\\
		\Lambda_{10}(0,t) &= \Lambda_{01}(s,0) = \Lambda_{11}(0,0) = 0\\
	\end{aligned}
	$$
If $F$ has a density $f(s,t)$, we have $\Lambda_{11}(ds,dt) = \lambda_{11}(s,t)ds~dt$, $\Lambda_{10}(ds,t) = \lambda_{10}(s,t)dt$, $\Lambda_{01}(s,dt) = \lambda_{01}(s,t)dt$, so

	$$
	\begin{aligned}
		\lambda_{11}(s,t) &= \lim_{(h_1,h_2)\rightarrow 0} \frac{1}{h_1 h_2} P(T_1\in[s,s+h_1], T_2\in[t,t+h_2] | T_1\geq s, T_2 \geq t) & = \frac{f(s,t)}{F(s-, t-)}\\
		\lambda_{10}(s,t) &= \lim_{h\rightarrow 0} \frac{1}{h} P(T_1\in[s,s+h] | T_1\geq s, T_2 > t) & = \int_t^{\infty} \frac{f(s,v)dv}{F(s-, t)}\\
		\lambda_{01}(s,t) &= \lim_{h\rightarrow 0} \frac{1}{h} P(T_2\in[t,t+h] | T_1 > s, T_2 \geq t) & = \int_s^{\infty} \frac{f(u,t)du}{F(s, t-)}\\
	\end{aligned}
	$$
Where $\lambda_{11}(s,t)$ is the instantaneous rate of \emph{double failure} at point $(s,t)$, given that the individuals were alive at times $T_1=s-$ and $T_2 = t-$. $\lambda_10(s, t)$ is the rate of a \emph{single failure} at time s give that the first individual was alive at time $T_1=s$ and the second survived beyond time $T_2 = 1$.\\

Long story short, he showed that a bivariate survival function can be decomposed into a product of two univariate components and some product of a function of bivariate hazards. He showed the consistency of the estimator (given independence of events and censoring), based on this decomposition:
	$$
	\begin{aligned}
		\hat{F}(s,t) = \hat{F}(s,0)\hat{F}(0,t)\prod_{\mathcal{A}}\{1 - \hat{L}(\Delta u, \Delta v)\}
	\end{aligned}
	$$
Where 
\begin{itemize}
	\item $\mathcal{A} = \{0<u\leq s\} \cap \{0 < v \leq t\} \cap \{(u, v) \in E_i\}$, where $E_i$ are four sets based on \emph{Jordan decomposition of functions with bounded variation}, and $\hat{L}(\Delta u, \Delta v)$ is a function of hazards (see the source).
  \item $\hat{F}(s,0)$ and $\hat{F}(0,t)$ are usual Kaplan-Meier estimates, for example, $\hat{F}(s,0) = \prod_{u\leq s}[1-\hat{\Lambda_{10}(\Delta u, o)}]$
	\item $\hat{L}(\Delta u, \Delta v) = \frac{\hat{\Lambda}_{10}(\Delta u,v^-)\hat{\Lambda}_{01}(u^-,\Delta v) - \hat{\Lambda}_{11}(\Delta u,\Delta v)}{\left(1-\hat{\Lambda}_{10}(\Delta u,v^-)\right)\left(1-\hat{\Lambda}_{01}(u^-,\Delta v)\right)}$
\end{itemize}

 \subsection{Pruitt, 1991 \cite{pruitt1991negative}}
 Pruitt showed that Dabrowska's estimator assigns negative mass to points of $\hat{F}(s,t)$. Although the value of negative mass decreases, the number of such points does not disappear with growing sample size, resulting in non-disappearing negative mass.\\
This problem is related to the problem of identifiability of bivariate survival functions.


 \subsection{Oakes, 1989 \cite{oakes1989bivariate}}
 Oakes studied association between two survival functions in the context of \emph{frailty} models $Pr(T>t|W=w) = \{B(t)\}^w$. It can be shown that for these models:
	$$
	\begin{aligned}
		S(t) &= Pr(T>t) = E_W[Pr(T>t,w)] = \int Pr(T>t|w)dF{w} = E_W[B^w(t)] = \\
		  &=E_W e^{-(-logB(t))w} = p(-logB(t))
	\end{aligned}
	$$
 where $p(\cdot)$ is the Laplace transform of $W$ ($p(x) = Ee^{xW}$). This representation extends for bivariate distributions, $S(t_1, t_2) = Pr(T_1>t_1, T_2>t_2)$. ''A bivariate frailty model asserts that T, and T2 are conditionally independent, given W:''
	$$
	\begin{aligned}
		S(t_1, t_2) &= \int \{B_1(t_1)B_2(t_2)\}^w dF(w)\\
		& therefore:\\
		S(t_1, t_2) &= p(-logB_1(t_1)-logB_2(t_2))
	\end{aligned}
	$$
	Oakes shows that when each $S_j(t_j)$ depends on the unobserved $W$ this induces an association between the observed times. He also notes that bivariate distributions generated by frailty models are a subclass of the archimedean distributions studied by Genest and MacKay (1986a,b) with the following general form. In the following we use the following notation $t=(t_1, t_2)$\\
	$$
	\begin{aligned}
		S(t) &= p[q(S_1\{t_1\}) + q(S_2\{t_2\})],~~~~~S_1(t_1) = S_1(t_1, 0)~~and~~S_2(t_2) = S_2(0, t_2)
	\end{aligned}
	$$
Where $p(\cdot)$ is any nonnegative decreasing function with $p(0)=1$ and nonnegative second derivative, and $q(\cdot)$ is its inverse function. Recalling Clayton's formula for $\theta$:
	$$
	\begin{aligned}
		\theta &= \frac{ \frac{\partial^2 S(s,t)}{\partial s \partial t} S(s, t)}    {\frac{\partial S(s, t)}{\partial t} \frac{\partial S(s, t)}{\partial s}}
	\end{aligned}
	$$
Oakes proves a lemma that to make a point that $\theta$ depends on $t$ only through $S(t)=v$ $\theta(v) = \theta^*(t) = \theta^*(S(t))$, (keep in mind that $t$ is a vector). He derives a general formula for $\theta(v)$:
	$$
	\begin{aligned}
		\theta(v) = -v\cdot q''(v)/q'(v)
	\end{aligned}
	$$
He also shows how to find $S(t)$ based on $\theta(v)$. He gives several examples of how to find one from the other.


	
{\tiny{\textbf{We recall Kendall's $\tau$, 1938 \cite{kendall1938new}}: If we have two independent paris of variables (with the same bivariate distribution) $(U_1, Z_1)$ and $(U_2, Z_2)$, Kendall's $\tau$ is a probability of concordance minus probability of discordance:
$\tau = P[(U_1 - U_2)(Z_1 - Z_2)>0] - P[(U_1 - U_2)(Z_1 - Z_2)<0] = E[sign((U_1 - U_2)(Z_1 - Z_2))]$}}

 
In notations of Oakes, Kendall's (1938) coefficient of concordance is:
	$$
	\begin{aligned}
		&\tau = E\{sign(T_{1}^{(a)} - T_{1}^{(b)}) (T_{2}^{(a)} - T_{2}^{(b)}) \}
	\end{aligned}
	$$
	Where $(T_1^{(a)}, T_2^{(a)})$ and $(T_1^{(b)}, T_2^{(b)})$ are two independent pairs (for example, if we study mothers (1) and daughers (2), $(T_1^{(a)}, T_2^{(a)})$ is one mother-daughter pair and $(T_1^{(b)}, T_2^{(b)})$ is the other).\\

 Oakes called $\tau$ a \emph{global} measure. He also said that parameter $\theta$ (introduced by Clayton) can be viewed as \emph{local} measure. If $T^{(a)}=(T_1^{(a)}, T_2^{(a)})$ and $T^{(b)} = (T_1^{(b)}, T_2^{(b)})$, let $\tilde{T}^{(ab)}$ denote the component-wise minimum: $\tilde{T}^{(ab)}_j = min(T^{(a)}_j, T^{(b)}_j),~j\in\{1,2\}$, then (Oakes says that \emph{it is easily seen}):
	$$
	\begin{aligned}
		\theta(v) = \frac{Pr\{(T^{(a)}, T^{(b)})~concordant|\tilde{T}^{(ab)}=t\} }{  Pr\{(T^{(a)}, T^{(b)})~discordant|\tilde{T}^{(ab)}=t\}}
 	\end{aligned}
	$$
Therefore, the ratio $\frac{\theta(v)-1}{\theta(v)+1}$ is a conditional version of Kendall's $\tau$.
	$$
	\begin{aligned}
		&\tau(v) = E\{sign(T_{1}^{(a)} - T_{1}^{(b)}) (T_{2}^{(a)} - T_{2}^{(b)}) ~|~ min(T^{(a)}_1, T^{(b)}_1) = s_1,~min(T^{(a)}_2, T^{(b)}_2) = s_2  \}
	\end{aligned}
	$$
He comes up with an estimator and proves its convergence to $\theta(v)$.

\subsection{Cuzick, 1982 citation ?????}
Considers a general model of association:
	$$
	\begin{aligned}
		Y_1 = aZ + e_1 ~~~~~~~~ Y_2 = bZ + e_2
	\end{aligned}
	$$
Where $Z$, $e_1$, and $e_2$ are independent. He considers two cases, and only the second case is relevant for us, when $b=a\lambda,~~0<|\lambda|<\infty$. In this case two variables $(Y_1, Y_2)$ are thought to be related to a third unobserved covariate.\\
It is not very clear from the paper what they denote as $f_i(x)$. I am guessing this is a density of $Y_i$ (or of $e_i$?). In any case, they author says that when this density is logistic, $f(x) = 2\pi_{-1} e^{-x}/(1+e^{-x})^2$, then the rank of each observation is $F(x)$ with no censoring, and $\frac{1+F(x)}{2}$, when censoring is present.

\subsection{Shih and Louis, 1998 citation ?????}
They take martingale residuals $T_n=\sum_i \int_0^{t_0}  \int_0^{t_0} d\hat{M}_{i1}(u_1) d\hat{M}_{i2}(u_2)$ (where $M_{ij}$ is the difference between observed and expected deaths), and create two statistics out of them:
\begin{enumerate}
	\item take supremum: $U_n= \sup_{0\leq t \leq t_0} \left|\sum_i \int_0^{t_0}  \int_0^{t_0} d\hat{M}_{i1}(u_1) d\hat{M}_{i2}(u_2)\right|$
	\item weigh it: $V_n= \sum_i \int_0^{t_0}  \int_0^{t_0} W_n(u_1,u_2) d\hat{M}_{i1}(u_1) d\hat{M}_{i2}(u_2)$ (which is analogous to the weighted logrank test)
\end{enumerate}
They choose \emph{optimal} weights by somehow using the \emph{cross ratio} defined by Oakes $\theta = \frac{ \frac{\partial^2 S(s,t)}{\partial s \partial t} S(s, t)}    {\frac{\partial S(s, t)}{\partial t} \frac{\partial S(s, t)}{\partial s}}$: 

\subsection{Alvo and Cabilio, 1995 citation ?????}
UNDER CONSTRUCTION.\\
The authors use a concept of distance applied to \emph{Spearman} and \emph{Kendall} correlation measure in order to build a test that can deal with missing data. 

\subsection{Fan, 1998 \cite{fan2000dependence}}
There is a need in non-parametric measure of association between two survival times.\\
Clayton suggested a \textit{cross ratio} (Clayton, 1978, Oaks, 1989) (see Clayton's $\theta$):
	$$
	\begin{aligned}
		&\tilde{c}(s_1, s_2) = \frac{F(ds_1, ds_2)F(ds_1^-, ds_2^-)} {F(ds_1, ds_2^-)F(ds_1^-, ds_2)} \\
		 &or\\
		 & \tilde{c}(s_1, s_2) = \frac{ \lambda_1(s_1|T_2=s_2)}{\lambda_1(s_1|T_2 \geq s_2)} = \frac{ \lambda_2(s_2|T_1=s_1)}{\lambda_2(s_2|T_1 \geq s_1)}
	\end{aligned}
	$$
Which $\left(\frac{ \lambda_2(s_2|T_1=s_1)}{\lambda_2(s_2|T_1 \geq s_1)}\right)$ can be interpreted as ''the ratio of the breast cancer risk for daughters of age $s_2$ whose mothers developed breast cancer at age $s_1$, compared to daughters of age $s_2$ whose mothers were breast cancer free to age $s_1$. This type of hazard ratio, or relative risk, is most familiar to epidemiologists, and an average relative risk over a range of ages may be readily interpreted and meaningful in epidemiologic contexts''.\\

''Following the suggestion of in Hsu and Prentice (1996) one could consider a summary dependence measure that weights the cross ratio $\tilde{c}(s_1, s_2)$ over $[0, t_1]\times[0, t_2]$ by $\frac{F(ds_1,ds_2)}{\int_0^{t_1}\int_0^{t_2} F(du_1, du_2)}$''. But there are some problems with consistency of $\tilde{c}(s_1, s_2)$ when it is weighted this way.\\
Instead, the authors propose to weight $c(s_1, s_2) = \frac{1}{\tilde{c}(s_1, s_2)}$ and get:
	$$
	\begin{aligned}
		C(t_1, t_2) &= \int_0^{t_1}\int_0^{t_2} \frac{c(s_1, s_2)F(ds_1,ds_2)}{\int_0^{t_1}\int_0^{t_2} F(du_1, du_2)} =\int_0^{t_1}\int_0^{t_2} \frac{F(ds_1, ds_2^-)F(ds_1^-, ds_2)} {\cancel{F(ds_1, ds_2)}F(ds_1^-, ds_2^-)} \frac{\cancel{F(ds_1,ds_2)}}{\int_0^{t_1}\int_0^{t_2} F(du_1, du_2)} = \\
		&= \int_0^{t_1}\int_0^{t_2} \frac{F(ds_1, ds_2^-)F(ds_1^-, ds_2)} {F(ds_1^-, ds_2^-)} \frac{1}{\int_0^{t_1}\int_0^{t_2} F(du_1, du_2)} \frac{F(ds_1^-, ds_2^-)}{F(ds_1^-, ds_2^-)}= \\
		&= \int_0^{t_1}\int_0^{t_2} \frac{F(ds_1^-, ds_2^-)\Lambda_{10}(ds_1, s^-_2)\Lambda_{01}(s^-_1, ds_2)}{\int_0^{t_1}\int_0^{t_2} F(du_1, du_2)} = \\
	\end{aligned}
	$$
	Because $F(\cdot, \cdot)$ is a survival function, our denominator is:
	$$
	\begin{aligned}
		\int_0^{t_1}\int_0^{t_2} F(du_1, du_2) &= \int_0^{t_2} [F(t_1, du_2) - F(0, du_2)]= F(t_1, t_2) - F(t_1, 0) - F(0, t_2) + F(0,0)=\\
		&=1 - F(t_1, 0) - F(0, t_2)+ F(t_1, t_2)\\
	\end{aligned}
	$$
	So
	$$
	\begin{aligned}
		C(t_1, t_2) &= \int_0^{t_1}\int_0^{t_2} \frac{F(ds_1^-, ds_2^-)\Lambda_{10}(ds_1, s^-_2)\Lambda_{01}(s^-_1, ds_2)}{1 - F(t_1, 0) - F(0, t_2)+ F(t_1, t_2)} \\
	\end{aligned}
	$$
	When $C(t_1, t_2)>1$, there is positive association. When $C(t_1, t_2)=1$, there is no association. When $C(t_1, t_2)<1$, there is negative association.  \\
	In addition to $C(t_1, t_2)$ they proposed another summary measure of dependence inspired by Oakes, 1989 \cite{oakes1989bivariate} who proposed (we are using notations of Fan, 1998 \cite{fan2000dependence}):
	$$
	\begin{aligned}
		&\tau(s_1, s_2) = E\{sign(T_{11} - T_{12}) (T_{21} - T_{22})~|~ T_{11}\wedge T_{12} = s_1,~T_{21}\wedge T_{22} = s_2  \}
		&~~~~where~\wedge~is~minimum
	\end{aligned}
	$$
\cite{oakes1989bivariate} noted that \emph{Kendall's} $\tau$ can be written as
	$$
	\begin{aligned}
		\tau(s_1, s_2) &= \frac{\theta(s_1, s_2) - 1}{\theta(s_1, s_2) + 1} =\frac{1 - c(s_1, s_2)}{1+c(s_1, s_2)}
	\end{aligned}
	$$
  
\subsubsection{Local version of \emph{Kendall's} $\tau$. Discrete case, not covered by \cite{fan2000class}}
\label{deriveWeightedTau}
Here we try to explain how \cite{oakes1989bivariate} showed that $\tau(s_1, s_2)$ can be written as:
	$$
	\begin{aligned}
		\tau(s_1, s_2) &= \frac{\theta(s_1, s_2) - 1}{\theta(s_1, s_2) + 1}
	\end{aligned}
	$$

In the process of explaining this, I realized that the equality $\tau(s_1, s_2) = \frac{\theta(s_1, s_2) - 1}{\theta(s_1, s_2) + 1}$ might not be true in the discrete case. I also show what the contionuous \emph{local version} of \emph{Kendall}'s $\tau$ derived by \cite{fan2000class} is different from the discrete case (which was not specificed in their paper).
~\\
For convenience, let's switch from notations $(T_{11},T_{12})$ and $(T_{21},T_{12})$ to $(X_{1},Y_{1})$ and $(X_{2},Y_{2})$. We denote the pdf and the survival functions as $f(x,y)$ and $S(x,y)$ respectively. Let's first find the distribution (\emph{pdf}) for $X_{1}\wedge X_{2} = x,~Y_{1}\wedge Y_{2} = y$:
  $$
  \begin{aligned}
    P\{ &X_{1}\wedge X_{2} = x,~Y_{1}\wedge Y_{2} = y \} =\\
     &=P\{ X_{1} > x, X_{2} = x, Y_{1} > y, Y_{2} = y \}
      + P\{ X_{1} > x, X_{2} = x, Y_{1}= y, Y_{2} = y \}
        + P\{ X_{1} > x, X_{2} = x, Y_{1}= y, Y_{2} > y \} + \\
     &+P\{ X_{1} = x, X_{2} = x, Y_{1} > y, Y_{2} = y \}
      + P\{ X_{1} = x, X_{2} = x, Y_{1}= y, Y_{2} = y \}
        + P\{ X_{1} = x, X_{2} = x, Y_{1}= y, Y_{2} > y \} + \\
     &+P\{ X_{1} = x, X_{2} > x, Y_{1} > y, Y_{2} = y \}
      + P\{ X_{1} = x, X_{2} > x, Y_{1}= y, Y_{2} = y \}
        + P\{ X_{1} = x, X_{2} > x, Y_{1}= y, Y_{2} > y \}\\
  \end{aligned}
  $$
Because $(X_{1}, Y_{1}) \perp (X_{2}, Y_{2})$ we have:
  $$
  \begin{aligned}
P\{ &X_{1}\wedge X_{2} = x,~Y_{1}\wedge Y_{2} = y \} 
      = S(x,y)S(dx, dy) +  S(x,dy)S(dx, dy) + S(x,dy)S(dx, y) + \\
      &~~~~~~+ S(dx,y)S(dx, dy) +  S(dx,dy)S(dx, dy) + S(dx,dy)S(dx, y) + \\
      &~~~~~~+ S(dx,y)S(x, dy) +  S(dx,dy)S(x, dy) + S(dx,dy)S(x, y) = \\
      &= 2S(x,y)S(dx, dy) +  2S(x,dy)S(dx, dy) + 2S(x,dy)S(dx, y) + 2S(dx,y)S(dx, dy) +  S^2(dx,dy) = \\
      &= 2S(dx, dy)\left\{S(x,y) +  S(x,dy) + S(dx,y) +  S(dx,dy)\right\}  + 2S(x,dy)S(dx, y)= \\
      &= 2S(dx, dy)\left\{\cancel{S(x,y)} +  S(x,y^-) - \cancel{S(x,y)} + S(x^-,y) - S(x,y) +  \frac{1}{2}S(x^-,y^-) - \frac{1}{2}S(x,y^-) - \frac{1}{2}S(x^-,y) + \frac{1}{2}S(x,y)\right\}\\
      &~~~~~~~~~~~~~~~~~~~~~~~~  + 2S(x,dy)S(dx, y)= \\
      &= 2S(dx, dy)\left\{\frac{1}{2}S(x^-,y^-)  + \frac{1}{2}S(x,y^-)  + \frac{1}{2}S(x^-,y) - \frac{1}{2}S(x,y)\right\} + 2S(x,dy)S(dx, y) =\\
      &= 2S(dx, dy)\left\{ S(x^-,y^-) -\frac{1}{2}S(dx,dy)\right\} + 2S(x,dy)S(dx, y) =\\
      &= 2S(dx, dy) S(x^-,y^-) -S^2(dx,dy) + 2S(x,dy)S(dx, y)\\
	\end{aligned}
	$$
  It can be shown that the following expressions are equivalent:
  $$
  \begin{aligned}
P\{ X_{1}\wedge X_{2} = x,~Y_{1}\wedge Y_{2} = y \}  &=  &\\
      &= 2S(dx, dy)S(x^-,y^-) + 2S(x,dy)S(dx, y) -S^2(dx,dy)= \\
      &= 2S(dx, dy)S(x,y)+ 2S(x^-,dy)S(dx, y^-)  -S^2(dx,dy)=\\
      &= 2S(dx, dy)S(x^-,y^-) + 2[S(x^-,dy)-S(dx,dy)][S(dx, y^-)-S(dx,dy)]  -S^2(dx,dy)=\\
      &= 2S(dx, dy)S(x^-,y^-) + 2S(x^-,dy)S(dx, y^-) + S^2(dx,dy) - 2S(dx,dy)[S(x^-,dy)+S(dx, y^-)]\\
	\end{aligned}
	$$

  The authors are doing something else. They probably assume continuous densities and use the fact that taking a derivative in point $x^-$ is the same as in point $x$ (????? under certain conditions ...):
	$$
	\begin{aligned}
    f_{min(X_1, X_2),min(Y_1, Y_2)}(x, y) &= \frac{\partial^2 S(x^-, y^-)}{\partial x \partial y} = \frac{\partial}{\partial y} \left(2\cdot S(x^-, y^-) \frac{\partial S(x^-, y^-)}{\partial x}  \right)  =\\
     &=2S(x^-, y^-) \frac{\partial^2 S(x^-, y^-)}{\partial x \partial y}   + 2 \frac{\partial S(x^-, y^-)}{\partial x}\frac{\partial S(x^-, y^-)}{\partial y}\\
     & or~in~terms~of~the~authors:\\
     &~\\
     &= 2 S(x^-, y^-) S(dx, dy) + 2 S(x^-, dy) S(dx, y^-)
	\end{aligned}
	$$

Now, let's find the joint distribution of concordance and discordance and of $X_{1}\wedge X_{2} = x,~Y_{1}\wedge Y_{2} = y$ by finding first the following:
	$$
	\begin{aligned}
		P\{ (X_1>X_2, Y_1>Y_2) &\cap (X_{1}\wedge X_{2} = x,~Y_{1}\wedge Y_{2} = y) \} =P\{ X_1>x, X_2=x, Y_1>y, Y_2=y)\} = \\
    &= P\{ X_1>x, Y_1>y, X_2=x, Y_2=y)\} = S(x, y)S(dx,dy) = \\
    &=S(dx,dy)\cdot[S(dx,dy) - S(x^-,dy) - S(dx,y^-) + S(x^-,y^-) ]\\
    &=S(dx,dy) S(x^-,y^-) + S(dx,dy)\cdot[S(dx,dy) - S(x^-,dy) - S(dx,y^-)]\\
    &=S(dx,dy) S(x^-,y^-) + [S(x^-,dy) - S(dx,dy)][S(dx,y^-) - S(dx,dy)] - S(x^-,dy)S(dx,y^-)\\
		P\{ (X_1>X_2, Y_1<Y_2) &\cap (X_{1}\wedge X_{2} = x,~Y_{1}\wedge Y_{2} = y) \} =P\{ X_1>x, X_2=x, Y_1=y, Y_2>y)\} = \\
    &= P\{ X_1>x, Y_1=y, X_2=x, Y_2>y)\} = S(x, dy)S(dx,y) = \\
    &=[S(x^-,dy) - S(dx,dy)][S(dx,y^-)  - S(dx,dy)]
	\end{aligned}
	$$
	$$
	\begin{aligned}
	\end{aligned}
	$$
  Because of symmetry:
	$$
	\begin{aligned}
		P\{&[Concord. - Discord.] \cap (X_{1}\wedge X_{2} = x,~Y_{1}\wedge Y_{2} = y) \} &= \\
      &=2S(dx,dy) S(x^-,y^-) + \cancel{2[S(x^-,dy) - S(dx,dy)][S(dx,y^-) - S(dx,dy)]} - 2S(x^-,dy)S(dx,y^-) - \\
      &~~~~~~~~-\cancel{2[S(x^-,dy) - S(dx,dy)][S(dx,y^-)  - S(dx,dy)]}=\\
      &~~~~~~~~~~~~~~~~~~~~~~~~~~~~~~~~~~~~~~~~~~~~~=~~2S(dx,dy) S(x^-,y^-) - 2S(x^-,dy)S(dx,y^-)
	\end{aligned}
	$$
  
  
Therefore, for the discrete case, we have:
  $$
  \begin{aligned}
    \tau(x, y) &= \frac{P\{concord| X_{1}\wedge X_{2} = x,~Y_{1}\wedge Y_{2} = y\} - P\{discord|X_{1}\wedge X_{2} = x,~Y_{1}\wedge Y_{2} = y\}}{P\{X_{1}\wedge X_{2} = x,~Y_{1}\wedge Y_{2} = y\}} = \\ 
      &= \frac{2S(dx,dy) S(x^-,y^-) - 2S(x^-,dy)S(dx,y^-)}{2S(dx, dy)S(x^-,y^-) + 2S(x,dy)S(dx, y) -S^2(dx,dy)}= \\
      &= \frac{2S(dx,dy) S(x^-,y^-) - 2S(x^-,dy)S(dx,y^-)}{2S(dx, dy)S(x,y)+ 2S(x^-,dy)S(dx, y^-)  -S^2(dx,dy)}=\\
      &= \frac{2S(dx,dy) S(x^-,y^-) - 2S(x^-,dy)S(dx,y^-)}{2S(dx, dy)S(x^-,y^-) + 2[S(x^-,dy)-S(dx,dy)][S(dx, y^-)-S(dx,dy)]  -S^2(dx,dy)}=\\
      &= \frac{2S(dx,dy) S(x^-,y^-) - 2S(x^-,dy)S(dx,y^-)}{2S(dx, dy)S(x^-,y^-) + 2S(x^-,dy)S(dx, y^-) + S^2(dx,dy) - 2S(dx,dy)[S(x^-,dy)+S(dx, y^-)]}\\
  \end{aligned}
  $$

The above expression shows that in discrete case, $\tau(s_1, s_2) = \frac{\theta(s_1, s_2) - 1}{\theta(s_1, s_2) + 1}$ is not true (as far as I can tell).
~\\
For the continuous case, however, everything kind of works out, if we don't care about the difference between $x$ and $x^-$:

  $$
  \begin{aligned}
    P\{concord| &X_{1}\wedge X_{2} = x,~Y_{1}\wedge Y_{2} = y\} - P\{discord|X_{1}\wedge X_{2} = x,~Y_{1}\wedge Y_{2} = y\} =\\
    &= 2S(dx,dy) S(x^-,y^-) - 2S(x^-,dy)S(dx,y^-)\\
    \tau(x, y) &= \frac{P\{concord| X_{1}\wedge X_{2} = x,~Y_{1}\wedge Y_{2} = y\} - P\{discord|X_{1}\wedge X_{2} = x,~Y_{1}\wedge Y_{2} = y\}}{P\{X_{1}\wedge X_{2} = x,~Y_{1}\wedge Y_{2} = y\}} = \\
     &= \frac{2S(x,y)S(dx,dy) - 2 S(x, dy)S(dx, y) }{ 2 S(x^-, y^-) S(dx, dy) + 2 S(x^-, dy) S(dx, y^-) } =\\
     &= \frac{\theta(x, y) - 1}{\theta(x, y) + 1}
  \end{aligned}
  $$
  ~\\

Since I want to get something that looks like the authors derive, I'll continue the derivations in the context of continuous case. Let's introduce the following notation:
	$$
	\begin{aligned}
		h(x, y) = P\{X_{1}\wedge X_{2} = x,~Y_{1}\wedge Y_{2} = y\}
	\end{aligned}
	$$
\cite{fan2000class} proposed to weight $\tau(x, y)$ using the following weight: $w(x, y) = \frac{h(x, y)}{\int_0^x \int_0^y h(u, v) dudv}$: 
	$$
	\begin{aligned}
		\mathcal{T}(t_x, t_y) &= \int_0^{t_x} \int_0^{t_y}  \tau(x, y)\frac{h(x, y)}{\int_0^{t_x} \int_0^{t_y} h(x, y) dxdy} dxdy = \\
    &= \int_0^{t_x} \int_0^{t_y}  \frac{S(x^-,y^-)f(x, y) - \frac{\partial S(x^-,y)}{\partial y}\frac{\partial S(x,y^-)}{\partial x} }{ \cancel{S(x^-,y^-)f(x, y) + \frac{\partial S(x^-,y)}{\partial y}\frac{\partial S(x,y^-)}{\partial x} }} \frac{\cancel{h(x, y)}}{\int_0^{t_x} \int_0^{t_y} h(x, y) dxdy} dxdy = \\
    &= \frac{ \int_0^{t_x} \int_0^{t_y} S(x^-,y^-)f(x, y)dxdy -  \int_0^{t_x} \int_0^{t_y} \frac{\partial S(x^-,y)}{\partial y}\frac{\partial S(x,y^-)}{\partial x}dxdy }{\int_0^{t_x} \int_0^{t_y} S(x^-,y^-)f(x, y)dxdy  +   \int_0^{t_x} \int_0^{t_y} \frac{\partial S(x^-,y)}{\partial y}\frac{\partial S(x,y^-)}{\partial x}dxdy }\\
	\end{aligned}
	$$

They say that the weighted $\tau$ (denoted as $\mathcal{T}(t_1, t_2)$) can be written as:
	$$
	\begin{aligned}
		&\mathcal{T}(t_1, t_2) = E\{sign(T_{11} - T_{12}) (T_{21} - T_{22})~|~ T_{11}\wedge T_{12} \leq t_1,~T_{21}\wedge T_{22} \leq t_2  \}
	\end{aligned}
	$$
	so that $\mathcal{T}(t_1, t_2) \in [-1,1]$
And in general, their weighting technique works because:
	$$
	\begin{aligned}
		P(X=x|Y\leq y) &= \frac{P(X=x, Y\leq y)}{P(Y\leq y)} 
    =\frac{\int_0^y P(X=x, Y = t)dF_Y}{\int_0^y P(Y= u)dF_Y}  = \\
    &=\frac{\int_0^y \frac{P(X=x, Y = t)}{P(Y = t)}P(Y = t)dF_Y}{\int_0^y P(Y= u)dF_Y} =\frac{\int_0^y P(X=x | Y = t)P(Y = t)dF_Y}{\int_0^y P(Y= u)dF_Y}=\\
&=\int_0^y P(X=x | Y = t)w(y)dF_Y\\
&~~~~~~~~~~~~~~~~~~where~~~~w(y) = \frac{P(Y = t)}{\int_0^y P(Y= u)dF_Y}
	\end{aligned}
	$$

\subsection{Prentice, 1992 \cite{prentice1992covariance}}
???
% If I understand correctly, the authors model bivariate survival as two marginal survivor functions and a covariance function (instead of joint survivor function).

\subsection{Fan, 2000 \cite{fan2000class}}
The authors suggest a class of estimators similar to what was described in \emph{Fan, 1998} \cite{fan2000dependence}, which were weighted averages of local measures. The motivation for new estimators was that ''for certain choices of weights, the estimation of the bivariate survivor function (e.g. Dabrowska (1998)\cite{dabrowska1988kaplan} and Prentice and Cai (1992) \cite{prentice1992covariance}), can be avoided and explicit variance formulae can be obtained.''\\
The authors utilize the fact that both $C(\cdot, \cdot)$ and $\mathcal{T}(\cdot, \cdot)$ depend on $F(\cdot, \cdot)$ only through the hazard function and therefore can be estimated using Nelson-Aalan-type estimator of the hazard:

	$$
	\begin{aligned}
    C(t_1, t_2) &= \frac{\int_0^{t_1}\int_0^{t_1} H_{11}(ds_1, ds_2)}{\int_0^{t_1}\int_0^{t_1} H_{10}(ds_1, s_2^-)H_{01}(s_1^-, ds_2)}\\
    \mathcal{T}(t_1, t_2) &= \frac{\int_0^{t_1}\int_0^{t_1} H_{11}(ds_1, ds_2) - \int_0^{t_1}\int_0^{t_1} H_{10}(ds_1, s_2^-)H_{01}(s_1^-, ds_2) }{\int_0^{t_1}\int_0^{t_1} H_{11}(ds_1, ds_2) + \int_0^{t_1}\int_0^{t_1} H_{10}(ds_1, s_2^-)H_{01}(s_1^-, ds_2) }\\
	\end{aligned}
	$$

\subsection{van der Laan, 1997 \cite{van1997nonparametric}}
The estimators of Dabrowska and Prentice and Cai:

	$$
	\begin{aligned}
    S(t_1, t_2) &= S_1(t_1)S_2(t_2)R(t_1, t_2);\\
		R(t_1, t_2) &= \frac{S(t_1, t_2)S(0, 0)}{S(t_1)S(t_2)}
 	\end{aligned}
	$$
where $R(t_1, t_2)$ is called \emph{a cross-ratio}(I think) and is a dependence measure between events $T_1>t_1$ and $T_2>t_2$, and $S_j(t_j)$ is a marginal survivor function that is estimated using Kaplan-Meier estimator. The cross-ratio has ''multiplicative property: the cross-ratio over the union of two adjacent rectangles equals the product of the cross-ratios of the rectangles.'' It can be computed iteratively, which leads to the Dbrowska estimator.\\
\\
If we represent $R(t_1, t_2) = e^{log(R(t_1, t_2))}$, then $log(R)$ is  an additive measure over the rectangle $(0,s_1]\times(0,s_2]$ and we can compute:
	
	$$
	\begin{aligned}
		log(R)(t_1, t_2) &= \int_{(0,t_1]\times(0,t_2]} d~log(R)(s_1,s_2)
 	\end{aligned}
	$$

There is a problem (with consistency) with finding these estimators for continuous using EM algorithm because we need to redistribute the mass and there is a problem with it for singly censored observations. This problem was corrected by Pruitt(1991) by using kernel-density estimators. Van der Laan proposed NPMLE based on interval censored singly-censored observations.\\
\\
Under complete independence of events and censoring events, Dabrowska's and Prentice and Cai's estimators did better. With increased dependence, Laan's estimator did better.

%%%%%%%%%%%%%%%%%%%%%%%%%%%%%%%%%%%%%%%%%%%%%%%%%%%%%%%%%%%%%%%%%%%%%%%%%%%%%
%%%%%%%%%%%  M-Estimation
%%%%%%%%%%%%%%%%%%%%%%%%%%%%%%%%%%%%%%%%%%%%%%%%%%%%%%%%%%%%%%%%%%%%%%%%%%%%%


\section{M-estimation for partial correlation of PSR}
In this section we are trying to understand how to compute variance of PSR correlation after adjusting for other covariates. We use \emph{M-estimation}. First we need the following vector:

\[
\pmb{\psi}_i(X_i, Y_i, \pmb{Z}_i; \pmb{\theta}) = \begin{bmatrix}
    \pmb{\psi}_X(X_i,\pmb{Z}_i; \pmb{\theta}_X) \\
    \pmb{\psi}_Y(Y_i,\pmb{Z}_i; \pmb{\theta}_Y) \\
    PSR_{X_{i}}(\pmb{\theta}_X) -\theta_1 \\
    PSR_{Y_{i}}(\pmb{\theta}_Y) -\theta_2 \\
    PSR_{X_{i}}(\pmb{\theta}_X) \cdot PSR_{Y_i}(\pmb{\theta}_Y) -\theta_3 \\
    (PSR_{X_{i}}(\pmb{\theta}_X))^2 -\theta_4 \\
    (PSR_{Y_{i}}(\pmb{\theta}_Y))^2 -\theta_5 \\
    \frac{\theta_3 - \theta_1\theta_2}{\sqrt{\theta_4 - \theta_1^2}\sqrt{\theta_5 - \theta_2^2}} - \theta_6
\end{bmatrix}
\]

Where $\pmb{\psi}_X(X_i,Z_i; \pmb{\theta}_X)$ and $\pmb{\psi}_Y(Y_i,Z_i; \pmb{\theta}_Y)$ are log-likelihood functions for models of association of $X\sim Z $ and $Y \sim Z$ respectively.\\
According to \cite{stefanski2002calculus}, we need to construct the following two matrices:

	
	$$
	\begin{aligned}
		\pmb{A}^*_n(\pmb{X}, \pmb{Y}, \pmb{Z}_i, \hat{\pmb{\theta}}) &= \frac{1}{n} \sum_{i = 1}^{n}\left[ -\frac{\partial}{\partial \pmb{\theta}^T}\psi(X_i, Y_i, \pmb{Z}_i, \hat{\pmb{\theta}}) \right]\\
		\pmb{B}^*_n(\pmb{X}, \pmb{Y}, \pmb{Z}_i, \hat{\pmb{\theta}}) &= \frac{1}{n} \sum_{i = 1}^{n} \psi(X_i, Y_i, \pmb{Z}_i, \hat{\pmb{\theta}}) \psi(X_i, Y_i, \pmb{Z}_i, \hat{\pmb{\theta}})^T\\
 	\end{aligned}
	$$

So the resulting variance is:
	$$
	\begin{aligned}
		\pmb{V}_n(\pmb{X}, \pmb{Y}, \pmb{Z}, \hat{\pmb{\theta}}) &= \frac{\pmb{V}^*_n(\pmb{X}, \pmb{Y}, \pmb{Z}, \hat{\pmb{\theta}})}{n} = \frac{(\pmb{A}_n^*)^{-1}(\pmb{X}, \pmb{Y}, \pmb{Z}, \hat{\pmb{\theta}}) \pmb{B}^*_n(\pmb{X}, \pmb{Y}, \pmb{Z}, \hat{\pmb{\theta}})  \left\{(\pmb{A}_n^*)^{-1}(\pmb{X}, \pmb{Y}, \pmb{Z}, \hat{\pmb{\theta}}) \right\}^T}{n} = \\
    &=\frac{\cancel{n}(\pmb{A}_n)^{-1}(\pmb{X}, \pmb{Y}, \pmb{Z}, \hat{\pmb{\theta}})\frac{1}{\cancel{n}} \pmb{B}_n(\pmb{X}, \pmb{Y}, \pmb{Z}, \hat{\pmb{\theta}})  \cancel{n}\left\{(\pmb{A}_n)^{-1}(\pmb{X}, \pmb{Y}, \pmb{Z}, \hat{\pmb{\theta}}) \right\}^T}{n} = \\
    &=(\pmb{A}_n)^{-1}(\pmb{X}, \pmb{Y}, \pmb{Z}, \hat{\pmb{\theta}}) \pmb{B}_n(\pmb{X}, \pmb{Y}, \pmb{Z}, \hat{\pmb{\theta}})  \left\{(\pmb{A}_n)^{-1}(\pmb{X}, \pmb{Y}, \pmb{Z}, \hat{\pmb{\theta}}) \right\}^T\\
    &~~~~where\\
		\pmb{A}_n(\pmb{X}, \pmb{Y}, \pmb{Z}_i, \hat{\pmb{\theta}}) &=  \sum_{i = 1}^{n}\left[ -\frac{\partial}{\partial \pmb{\theta}^T}\psi(X_i, Y_i, \pmb{Z}_i, \hat{\pmb{\theta}}) \right]\\
		\pmb{B}_n(\pmb{X}, \pmb{Y}, \pmb{Z}_i, \hat{\pmb{\theta}}) &=  \sum_{i = 1}^{n} \psi(X_i, Y_i, \pmb{Z}_i, \hat{\pmb{\theta}}) \psi(X_i, Y_i, \pmb{Z}_i, \hat{\pmb{\theta}})^T\\
 	\end{aligned}
	$$

\subsection{Defining the matrices}
\label{Defining:Matr}

The main challenge is to accurately define and code $\pmb{A}_n(\pmb{X}, \pmb{Y}, \pmb{Z}, \hat{\pmb{\theta}})$ because it involves a lot of derivatives. Let's show what the matrix would look like if $\pmb{\theta}_X = (\theta_{X, 1}, \theta_{X, 2})$ and $\pmb{\theta}_Y= (\theta_{Y, 1}, \theta_{Y, 2})$. Also, let's denote $\rho_S = \frac{\theta_3 - \theta_1\theta_2}{\sqrt{\theta_4 - \theta_1^2}\sqrt{\theta_5 - \theta_2^2}}$:

\[
\pmb{A}_n = -\sum_{i = 1}^{n} \begin{bmatrix}
    \frac{\partial \pmb{\psi}_X}{\partial \theta_{X,1}} & \frac{\partial \pmb{\psi}_X}{\partial \theta_{X,2}}  &  \frac{\partial \pmb{\psi}_X}{\partial \theta_{Y,1}}  &  \frac{\partial \pmb{\psi}_X}{\partial \theta_{Y,2}}     &      \frac{\partial \pmb{\psi}_X}{\partial \theta_1}     &     \frac{\partial \pmb{\psi}_X}{\partial \theta_2}     & ...   &     \frac{\partial \pmb{\psi}_X}{\partial \theta_6} \\
%%%%%%%%%%%%%%%%%%%%%%%%%%%%%%
   \frac{\partial \pmb{\psi}_Y}{\partial \theta_{X,1}} & \frac{\partial \pmb{\psi}_Y}{\partial \theta_{X,2}}  &  \frac{\partial \pmb{\psi}_Y}{\partial \theta_{Y,1}}  &  \frac{\partial \pmb{\psi}_Y}{\partial \theta_{Y,2}}     &      \frac{\partial \pmb{\psi}_Y}{\partial \theta_1}     &     \frac{\partial \pmb{\psi}_Y}{\partial \theta_2}     &    ...   &     \frac{\partial \pmb{\psi}_Y}{\partial \theta_6} \\
%%%%%%%%%%%%%%%%%%%%%%%%%%%%%%
    \frac{\partial (PSR_X - \theta_1)}{\partial \theta_{X,1}} & \frac{\partial (PSR_X - \theta_1)}{\partial \theta_{X,2}}  &  \frac{\partial (PSR_X - \theta_1)}{\partial \theta_{Y,1}}  &  \frac{\partial (PSR_X - \theta_1)}{\partial \theta_{Y,2}}      &      \frac{\partial (PSR_X - \theta_1)}{\partial \theta_1}     &     \frac{\partial (PSR_X - \theta_1)}{\partial \theta_2}     &    ...  &     \frac{\partial (PSR_X - \theta_1)}{\partial \theta_6} \\
%%%%%%%%%%%%%%%%%%%%%%%%%%%%%%
   \frac{\partial (PSR_Y - \theta_2)}{\partial \theta_{X,1}} & \frac{\partial (PSR_Y - \theta_2)}{\partial \theta_{X,2}}  &  \frac{\partial (PSR_Y - \theta_2)}{\partial \theta_{Y,1}}  &  \frac{\partial (PSR_Y - \theta_2)}{\partial \theta_{Y,2}}     &      \frac{\partial  (PSR_Y - \theta_2)}{\partial \theta_1}     &     \frac{\partial  (PSR_Y - \theta_2)}{\partial \theta_2}     &     ...   &     \frac{\partial  (PSR_Y - \theta_2)}{\partial \theta_6} \\
%%%%%%%%%%%%%%%%%%%%%%%%%%%%%%
  \hdotsfor{5} \\
%%%%%%%%%%%%%%%%%%%%%%%%%%%%%%
  \frac{\partial \rho_S}{\partial \theta_{X,1}} & \frac{\partial \rho_S}{\partial \theta_{X,2}}  &  \frac{\partial \rho_S}{\partial \theta_{Y,1}}  &  \frac{\partial \rho_S}{\partial \theta_{Y,2}}     &      \frac{\partial \rho_S}{\partial \theta_1}     &     \frac{\partial \rho_S}{\partial \theta_2}      &     ...    &     \frac{\partial \rho_S}{\partial \theta_6} \\
\end{bmatrix}
\]
Or:
\[
\pmb{A}_n =  \sum_{i = 1}^{n} \begin{bmatrix}
   -\frac{\partial \pmb{\psi}_X}{\partial \theta_{X, 1}}    &  -\frac{\partial \pmb{\psi}_X}{\partial \theta_{X, 2}}    &   0   &   0   &     0     &     0    &    0    &    0    &    0    &   0 \\
%%%%%%%%%%%%%%%%%%%%%%%%%%%%%%
    0   &   0   &   -\frac{\partial \pmb{\psi}_Y}{\partial \theta_{Y, 1}} &   -\frac{\partial \pmb{\psi}_Y}{\partial \theta_{Y,2}}      &     0     &     0    &    0    &    0    &    0    &   0  \\
%%%%%%%%%%%%%%%%%%%%%%%%%%%%%%
    -\frac{\partial PSR_X}{\partial \theta_{X,1}}   &  -\frac{\partial PSR_X}{\partial \theta_{X,2}}   & 0   &   0   &   1   &   0  & 0 & 0 & 0 & 0\\
%%%%%%%%%%%%%%%%%%%%%%%%%%%%%%
    0 & 0 &    -\frac{\partial PSR_Y}{\partial \theta_{Y,1}}   &  -\frac{\partial PSR_Y}{\partial \theta_{Y,2}}   & 0 & 1 & 0 & 0 & 0 & 0 \\
%%%%%%%%%%%%%%%%%%%%%%%%%%%%%%
     -\frac{\partial PSR_X}{\partial \theta_{X,1}} PSR_Y  &  -\frac{\partial PSR_X}{\partial \theta_{X,2}}PSR_Y & -\frac{\partial PSR_Y}{\partial \theta_{Y,1}} PSR_X  &  -\frac{\partial PSR_Y}{\partial \theta_{Y,2}}PSR_X & 0 & 0 &  1 & 0 & 0 & 0 \\
%%%%%%%%%%%%%%%%%%%%%%%%%%%%%%
    -2\frac{\partial PSR_X}{\partial \theta_{X,1}} PSR_X  &  -2\frac{\partial PSR_X}{\partial \theta_{X,2}}PSR_X & 0 & 0 & 0 & 0 &  0 & 1 & 0 & 0 \\
%%%%%%%%%%%%%%%%%%%%%%%%%%%%%%
    0 & 0 & -2\frac{\partial PSR_Y}{\partial \theta_{Y,1}}PSR_Y  &  -2\frac{\partial PSR_Y}{\partial \theta_{Y,2}}PSR_Y & 0 & 0 &  0 & 0 & 1 & 0 \\
%%%%%%%%%%%%%%%%%%%%%%%%%%%%%%
    0 & 0 & 0 & 0 & a_1 & a_2 &  a_3 & a_4 & a_5 & 1 \\
\end{bmatrix}
\]

Where:
  $$
  \begin{aligned}
    a_1 &= -\frac{\partial \rho_S}{\partial \theta_1} = -\frac{\theta_1\theta_3 - \theta_2\theta_4}{\sqrt{\theta_5 - \theta_2^2}(\theta_4 - \theta_1^2)^{3/2}}\\
    a_2 &= -\frac{\partial \rho_S}{\partial \theta_2} = -\frac{\theta_2\theta_3 - \theta_1\theta_5}{(\theta_5 - \theta_2^2)^{3/2} \sqrt{\theta_4 - \theta_1^2}}\\
    a_3 &= -\frac{\partial \rho_S}{\partial \theta_3} = -\frac{1}{\sqrt{\theta_5 - \theta_2^2}\sqrt{\theta_4 - \theta_1^2}}\\
    a_4 &= -\frac{\partial \rho_S}{\partial \theta_4} = \frac{1}{2}\frac{\theta_3 - \theta_1\theta_2}{\sqrt{\theta_5 - \theta_2^2}(\theta_4 - \theta_1^2)^{3/2}}\\
    a_5 &= -\frac{\partial \rho_S}{\partial \theta_5} = \frac{1}{2}\frac{\theta_3 - \theta_1\theta_2}{(\theta_5 - \theta_2^2)^{3/2} \sqrt{\theta_4 - \theta_1^2}}\\
   \end{aligned}
  $$
Note that $\frac{\partial \rho_S}{\partial \theta_6} = -1$

\[
\pmb{A}_n = \begin{bmatrix}
    \pmb{A}_{X,1,1}   &    \pmb{0}     &    0     &     0 & 0 & 0 & 0 & 0\\
            \pmb{0}   &     \pmb{A}_{Y,2,2}     &    0     &     0 & 0 & 0 & 0 & 0\\
          \pmb{A}_{X, 3,1}   &     \pmb{0}     &    n     &     0 & 0 & 0 & 0 & 0\\
            \pmb{0}   & \pmb{A}_{Y, 4,2} &    0     &     n & 0 & 0 & 0 & 0\\
            \pmb{A}_{X,5,1}   &     \pmb{A}_{Y,5,2}     & 0 &    0 & n & 0 & 0 & 0\\
          \pmb{A}_{X, 6,1}   &     \pmb{0}     &    0     &     0 & 0 & n & 0 & 0\\
            \pmb{0}   & \pmb{A}_{Y, 7,2} &    0     &     0 & 0 & 0 & n & 0\\
            \pmb{0}   & \pmb{0} &    a_1n & a_2n &  a_3n & a_4n & a_5n & n\\
\end{bmatrix}
\]

% \subsubsection{Alternative (wrong) approach of defining the matrices}
% This is something I wanted to try quickly, and it did not work. I don't understand yet why this is wrong, so without much explanation I am going to cut to the chase. Let's define vector $\pmb{\psi}_i$ in a different way:
% \[
% \pmb{\psi}_i(X_i, Y_i, \pmb{Z}_i; \pmb{\theta}) = \begin{bmatrix}
%     \pmb{\psi}_X(X_i,\pmb{Z}_i; \pmb{\theta}_X) \\
%     \pmb{\psi}_Y(Y_i,\pmb{Z}_i; \pmb{\theta}_Y) \\
%     PSR_{X_{i}}(\pmb{\theta}_X) -\theta_1 \\
%     PSR_{Y_{i}}(\pmb{\theta}_Y) -\theta_2 \\
%     \theta_1 \cdot \theta_2 -\theta_3 \\
%     \theta_1^2 -\theta_4 \\
%     \theta_2^2 -\theta_5 \\
%     \frac{\theta_3 - \theta_1\theta_2}{\sqrt{\theta_4 - \theta_1^2}\sqrt{\theta_5 - \theta_2^2}} - \theta_6
% \end{bmatrix}
% \]
%
%
% Then:
% \[
% \pmb{A}_n =  \sum_{i = 1}^{n} \begin{bmatrix}
%    -\frac{\partial \pmb{\psi}_X}{\partial \theta_{X, 1}}    &  -\frac{\partial \pmb{\psi}_X}{\partial \theta_{X, 2}}    &   0   &   0   &     0     &     0    &    0    &    0    &    0    &   0 \\
% %%%%%%%%%%%%%%%%%%%%%%%%%%%%%%
%     0   &   0   &   -\frac{\partial \pmb{\psi}_Y}{\partial \theta_{Y, 1}} &   -\frac{\partial \pmb{\psi}_Y}{\partial \theta_{Y,2}}      &     0     &     0    &    0    &    0    &    0    &   0  \\
% %%%%%%%%%%%%%%%%%%%%%%%%%%%%%%
%     -\frac{\partial PSR_X}{\partial \theta_{X,1}}   &  -\frac{\partial PSR_X}{\partial \theta_{X,2}}   & 0   &   0   &   1   &   0  & 0 & 0 & 0 & 0\\
% %%%%%%%%%%%%%%%%%%%%%%%%%%%%%%
%     0 & 0 &    -\frac{\partial PSR_Y}{\partial \theta_{Y,1}}   &  -\frac{\partial PSR_Y}{\partial \theta_{Y,2}}   & 0 & 1 & 0 & 0 & 0 & 0 \\
% %%%%%%%%%%%%%%%%%%%%%%%%%%%%%%
%      0  &  0  &  0  &  0  & -\theta_2 & -\theta_1 &  1 & 0 & 0 & 0 \\
% %%%%%%%%%%%%%%%%%%%%%%%%%%%%%%
%     0 & 0 & 0 & 0 & -2\theta_1 & 0 &  0 & 1 & 0 & 0 \\
% %%%%%%%%%%%%%%%%%%%%%%%%%%%%%%
%     0 & 0 & 0 & 0 & 0 & -2\theta_2 &  0 & 0 & 1 & 0 \\
% %%%%%%%%%%%%%%%%%%%%%%%%%%%%%%
%     0 & 0 & 0 & 0 & a_1 & a_2 &  a_3 & a_4 & a_5 & 1 \\
% \end{bmatrix}
% \]
%
% Where:
%   $$
%   \begin{aligned}
%     a_1 &= -\frac{\partial \rho_S}{\partial \theta_1} = -\frac{\theta_1\theta_3 - \theta_2\theta_4}{\sqrt{\theta_5 - \theta_2^2}(\theta_4 - \theta_1^2)^{3/2}}\\
%     a_2 &= -\frac{\partial \rho_S}{\partial \theta_2} = -\frac{\theta_2\theta_3 - \theta_1\theta_5}{(\theta_5 - \theta_2^2)^{3/2} \sqrt{\theta_4 - \theta_1^2}}\\
%     a_3 &= -\frac{\partial \rho_S}{\partial \theta_3} = -\frac{1}{\sqrt{\theta_5 - \theta_2^2}\sqrt{\theta_4 - \theta_1^2}}\\
%     a_4 &= -\frac{\partial \rho_S}{\partial \theta_4} = \frac{1}{2}\frac{\theta_3 - \theta_1\theta_2}{\sqrt{\theta_5 - \theta_2^2}(\theta_4 - \theta_1^2)^{3/2}}\\
%     a_5 &= -\frac{\partial \rho_S}{\partial \theta_5} = \frac{1}{2}\frac{\theta_3 - \theta_1\theta_2}{(\theta_5 - \theta_2^2)^{3/2} \sqrt{\theta_4 - \theta_1^2}}\\
%    \end{aligned}
%   $$
% \[
% \pmb{A}_n = \begin{bmatrix}
%     \pmb{A}_{X,1,1}   &    \pmb{0}     &    0     &     0 & 0 & 0 & 0 & 0\\
%             \pmb{0}   &     \pmb{A}_{Y,2,2}     &    0     &     0 & 0 & 0 & 0 & 0\\
%           \pmb{A}_{X, 3,1}   &     \pmb{0}     &    n     &     0 & 0 & 0 & 0 & 0\\
%             \pmb{0}   & \pmb{A}_{Y, 4,2} &    0     &     n & 0 & 0 & 0 & 0\\
%             \pmb{0}   & \pmb{0}     & -\theta_2n  & -\theta_1n & n & 0 & 0 & 0\\
%             \pmb{0}   & \pmb{0}     & -2\theta_1n &     0 & 0 & n & 0 & 0\\
%             \pmb{0}   & \pmb{0} &    0     &     2\theta_2n & 0 & 0 & n & 0\\
%             \pmb{0}   & \pmb{0} &    a_1n & a_2n &  a_3n & a_4n & a_5n & n\\
% \end{bmatrix}
% \]

\subsubsection{Important Notes}

\begin{enumerate}[1)]
\item
When \textbf{checking} your derivations using \textbf{simulations}, remember that the diagonal of matrix $B^*_n(\pmb{X}, \pmb{Y}, \pmb{Z}_i, \hat{\pmb{\theta}}) = \frac{1}{n} \sum_{i = 1}^{n} \psi(X_i, Y_i, \pmb{Z}_i, \hat{\pmb{\theta}}) \psi(X_i, Y_i, \pmb{Z}_i, \hat{\pmb{\theta}})^T$ is basically an estimated variance of each component of $\pmb{\psi}_i(X_i, Y_i, \pmb{Z}_i; \pmb{\theta})$. It is \textbf{not} a variance of the estimator of each parameter. For example, the $6^{th}$ element from the bottom of the diagonal is the variance of $PSR_{X_{i}}(\pmb{\theta}_X)$, and \textbf{not} the variance of $\hat{\theta}_1 = \frac{1}{n}\sum_{i=1}^{n} PSR_{X_{i}}(\pmb{\theta}_X)$.\\
The variance of the mean of $PSR$ (variance of $\hat{\theta}_1$) is the $6^{th}$ (from the bottom) diagonal element of the resulting sandwich estimator.

\item In Qi's code, instead of $\pmb{A}^{-1}\pmb{B} (\pmb{A}^{-1})^T$, she has $\pmb{A}^{-1}\psi (\pmb{A}^{-1}\psi)^T$, which is essentially the same.\\

\item In this comment we use notations of section \ref{Defining:Matr}. In this section we showed that matrices $\pmb{A}_{X,1,1}$ and $\pmb{A}_{Y,2,2}$ are the double derivatives of the log-likelihood multiplied by $-1$, and  $\pmb{A}_{X,3,1}$, $\pmb{A}_{Y, 4,2}$,  $\pmb{A}_{X,5,1}$, $\pmb{A}_{Y,5,2}$, $\pmb{A}_{X,6,1}$, $\pmb{A}_{Y, 7,2}$ are combinations of $\frac{\partial PSR}{\partial \theta}$  multiplied by $-1$.\\
The function that we are using to perform M-estimation (function \texttt{corTS()}) does not multiply the double derivatives by $-1$, and as a result, we are observing a missmatch between the sandwich variance estimator and the simulated variance estimator of the mean of $PSR$s (the variance of $\hat{\theta}_1$ and $\hat{\theta}_2$).\\
The fact that the sandwich variance estimator of $\hat{\theta}_1$ and $\hat{\theta}_2$ depends on the sign of these matrices might not make sense at first. Indeed, the derivative of log-likelihood should equal to zero and $PSR$ is supposed to equal to zero, so what does it matter what sign these matrices have. It turns out that the variance estimator (\textbf{for parametric survival models considered in this document}) is correct if we either multiply all these matrices by $-1$ or if we don't multiply them by $-1$. Here is why.\\

Suppose, we are interested in the sandwich estimator of $\hat{\theta}_1$. For simplicity, let's assume that we are modeling survival time using exponential model with an intercept only, $\theta_X$. In this paragraph, we remember that $\psi_X = \frac{\partial l_X}{\partial \theta_X}$ (where $l_X$ is the log-likelihood for $\theta_X$). Let's denote the variance of $PSR_X$, $E\left[PSR_X^2 \right]$,  as $\sigma^2_X$. Let's also use the following notations: $\gamma_{31} = -E\left[ \frac{\partial PSR_X}{\partial \theta_X} \right]$ (this is essentially matrix $\pmb{A}_{X,3,1}$ ) and $\gamma_{11} = -E\left[ \frac{\partial^2 l_X}{\partial \theta_X^2} \right]$ (matrix $\pmb{A}_{X,1,1}$). Then, the sandwich estimator for variance of $\hat{\theta}_1$ is:
  $$
  \begin{aligned}
    \widehat{var(\hat{\theta}_1)} = \left(  \frac{\gamma_{31}}{\gamma_{11}} \right)^2    E\left[ \left(\frac{\partial l_X}{\partial \theta_X} \right)^2  \right]   + 2\cdot  \left( \frac{\gamma_{31}}{\gamma_{11}}  \right)   E\left[\frac{\partial l_X}{\partial \theta_X}  PSR_X  \right]  + \sigma^2_X
  \end{aligned}
  $$

We are interested in term $2\cdot  \left( \frac{\gamma_{31}}{\gamma_{11}}  \right)   E\left[\frac{\partial l_X}{\partial \theta_X}  PSR_X  \right]$. Note that the sign of this term depends on the signs of $\gamma_{31}$, $\gamma_{11}$, and $E\left[\frac{\partial l_X}{\partial \theta_X}  PSR_X  \right]$. For exponential model with an intercept, $\gamma_{31}<0$, $\gamma_{11}>0$, and $E\left[\frac{\partial l_X}{\partial \theta_X}  PSR_X  \right]>0$, which means that $E\left[ \frac{\partial PSR_X}{\partial \theta_X} \right]>0$ and $E\left[ \frac{\partial^2 l_X}{\partial \theta_X^2} \right] < 0$. Because we are looking for maximum likelihood, the fact that $E\left[ \frac{\partial^2 l_X}{\partial \theta_X^2} \right] < 0$ is a relief and should hold across all models. On the other hand, it seems to me that whether $E\left[ \frac{\partial PSR_X}{\partial \theta_X} \right]$ or $E\left[\frac{\partial l_X}{\partial \theta_X}  PSR_X  \right]$ are greater or less than zero might depend on parametrization of the model.\\
Let me know what you think.\\
My next to-do item is to check if PSR computation for other models takes parametrization into account.

%In Qi's code (function \texttt{corTS()}), matrices $\pmb{A}_{X,3,1}$, $\pmb{A}_{Y, 4,2}$,  $\pmb{A}_{X,5,1}$, $\pmb{A}_{Y,5,2}$, $\pmb{A}_{X,6,1}$, $\pmb{A}_{Y, 7,2}$ (see section \ref{Defining:Matr}) were multiplied by $-1$. Simulations showed that this caused the variance of $\hat{\theta}_1$ (see section \ref{Defining:Matr}) to be heavily overestimated. I am planning to check if she was using function \texttt{corTS()} in a way that this did not create a problem.

\item Derivations in section \ref{norm:QiCode} were not verified.

\item In \textbf{all} parametric survival regressions considered in this document, their \emph{parametrization} (not provided by \texttt{R} documentation) was verified. In order to do so I used papers, online resources, and my own guesses, which were verified by simulations, the value of the likelihood, and similar techniques.

\item In section \ref{MEst:loglogistic} were only partially verified. I only verified the derivatives of $PSR$ by regression parameters (coefficients and scale).

\item Derivations in sections \ref{MEst:exponential}, \ref{MEst:weibull}, and \ref{MEst:lognormal}  were thoroughly verified using simulation, numeric derivatives, and other techniques.


\end{enumerate}

\subsubsection{Example: M-estimation in case of normal distribution and its implementation using Qi's code.}
\label{norm:QiCode}

In this section, we refer to two functions in Qi's code: \texttt{corTS()} and \texttt{lm.scores()}. Function  \texttt{corTS()} takes score equations and computes partial correlation. The mapping from the above notations to its parameters is the following:\\
\texttt{xresid} and \texttt{yresid} are $PSR_X$ $PSR_Y$.\\
\texttt{xz.dl.dtheta} and \texttt{yz.dl.dtheta} are $\pmb{\psi}_{X}$ and $\pmb{\psi}_{Y}$.\\
\texttt{xz.d2l.dtheta.dtheta} and \texttt{yz.d2l.dtheta.dtheta} are $A_{X,1,1}$ and $A_{Y,2,2}$.\\
\texttt{dxresid.dthetax} and \texttt{dxresid.dthetax} are $\frac{PSR_X}{\partial \pmb{\theta}_X}$ and $\frac{PSR_Y}{\partial \pmb{\theta}_Y}$\\

The above is all that is needed for M-estimation. Now we consider linear regression as an example and map our notations to Qi's function \texttt{lm.scores()} return variables. Let's figure out $\pmb{A}_{Y,2,2}$, $\pmb{A}_{Y, 4,2}$, $\pmb{A}_{Y,5,2}$, $\pmb{A}_{Y, 7,2}$, for normally distributed $Y$'s (the corresponding matrices for $X$, $\pmb{A}_{X,1,1}$, $\pmb{A}_{X, 3,1}$, $\pmb{A}_{X,5,1}$, $\pmb{A}_{X, 6,1}$, can be found in a similar manner). For linear regression $Y\sim Z$, the score equations, $\psi_Y$, are obtained straight from the likelihood. Let's denote $\pmb{res}_i=\pmb{Y}_i - \pmb{Z}_i\pmb{\theta}_Y$, then for $\pmb{\theta} = (\beta_0, \beta_1, \beta_2)$, for example, we have:
\[
\pmb{\psi}_{Y_i} = 
	\begin{bmatrix}
		y_i - \beta_0 - \beta_1z_{i1} - \beta_2z_{i2}\\
		z_{1i}(y_i - \beta_0 - \beta_1z_{i1} - \beta_2z_{i2})\\
		z_{2i}(y_i - \beta_0 - \beta_1z_{i1} - \beta_2z_{i2})\\
		-\sigma^2 + (y_i - \beta_0 - \beta_1z_{i1} - \beta_2z_{i2})^2\\
 	\end{bmatrix} = 
	\begin{bmatrix}
		res_i\\
		z_{1i}res_i\\
		z_{2i}res_i\\
		-\sigma^2 + res_i^2\\
 	\end{bmatrix} 
\]
In order to find $B_n$ using matrix multiplication, we actually have:
\[
\pmb{\psi}_{Y} = 
	\begin{bmatrix}
		res_1 & res_2 & ... & res_N\\
		z_{11}res_1 & z_{12}res_1 & ... & z_{1N}res_N\\
		z_{21}res_1 & z_{22}res_1 & ... & z_{2N}res_N\\
		-\sigma^2 + res_1^2 & -\sigma^2 + res_2^2 & ... & -\sigma^2 + res_N^2\\
 	\end{bmatrix} 
\]

I think Qi was not including the derivative by $\sigma^2$, and her $\pmb{\psi}_{Y}$ is transposed, probably because she solves matrixes in a little bit different way. With this in mind, we map (somewhat loosely) the return variable \texttt{dl.dtheta} to $\pmb{\psi}_{Y}$.\\

Let's denote $res_i = y_i - \beta_0 - \beta_1z_{1i} - \beta_2z_{2i}$ and use variable name $l$ for the likelihood:  

\[
\pmb{A}_{Y,2,2} = -\sum_i^n\begin{bmatrix}
    \frac{\partial \pmb{\psi}_{Y_i}}{ \partial \beta_0},  &
    \frac{\partial \pmb{\psi}_{Y_i}}{ \partial \beta_1}, &
    \frac{\partial \pmb{\psi}_{Y_i}}{ \partial \beta_2}, &
    \frac{\partial \pmb{\psi}_{Y_i}}{ \partial (\sigma^2)} &
\end{bmatrix}
 =-\sum_i^n\begin{bmatrix}
    \frac{\partial^2 l_{Y_i}}{ \partial \beta_0\partial \beta_0}  & \frac{\partial^2 l_{Y_i}}{ \partial \beta_0\partial \beta_1} & \frac{\partial^2 l_{Y_i}}{ \partial \beta_0\partial \beta_2} & \frac{\partial^2 l_{Y_i}}{ \partial \beta_0 \partial (\sigma^2)}\\
    \frac{\partial^2 l_{Y_i}}{ \partial \beta_1\partial \beta_o}  & \frac{\partial^2 l_{Y_i}}{ \partial \beta_1\partial \beta_1} & \frac{\partial^2 l_{Y_i}}{ \partial \beta_1\partial \beta_2} & \frac{\partial^2 l_{Y_i}}{ \partial \beta_1 \partial (\sigma^2)}\\
    \frac{\partial^2 l_{Y_i}}{ \partial \beta_2\partial \beta_0}  & \frac{\partial^2 l_{Y_i}}{ \partial \beta_2\partial \beta_1} & \frac{\partial^2 l_{Y_i}}{ \partial \beta_2\partial \beta_2} & \frac{\partial^2 l_{Y_i}}{ \partial \beta_2 \partial (\sigma^2)}\\
    \frac{\partial^2 l_{Y_i}}{ \partial (\sigma^2) \partial \beta_0}  & \frac{\partial^2 l_{Y_i}}{ \partial (\sigma^2) \partial \beta_1} & \frac{\partial^2 l_{Y_i}}{ \partial (\sigma^2)\partial \beta_2} & \frac{\partial^2 l_{Y_i}}{ \partial (\sigma^2) \partial (\sigma^2)}\\
\end{bmatrix} =
\]
\[
  =\sum_i^n\begin{bmatrix}
    1   &  z_{i1}   & z_{i2}  &  2 \cdot res_i\\
    z_{i1}   &  z_{i1}^2   & z_{i1}z_{i2}  &  2 \cdot res_iz_{i1} \\
    z_{i2}   &  z_{i1}z_{i2}   & z_{i2}^2  &  2 \cdot res_iz_{i2}  \\
    2 \cdot res_i   & 2 \cdot res_iz_{i1}   & 2 \cdot res_iz_{i2}  &  1 \\
\end{bmatrix}=
\begin{bmatrix}
  \pmb{Z}^T\pmb{Z} & 2 \cdot \pmb{Z}^T \pmb{res}\\
  2 \cdot \pmb{res}^T\pmb{Z} & 1\\
  \end{bmatrix}
\]

Because Qi was not including the derivative by $\sigma^2$, the return variable \texttt{d2l.dtheta.dtheta} can be mapped to  $\pmb{Z}^T\pmb{Z}$, or $\pmb{A}_{X,1,1}$ ($\pmb{A}_{Y,2,2}$), again loosely.\\
I am not sure why the function returns variable \texttt{resid} (linear regression residuals $Y_i - \pmb{Z}_i\pmb{\theta}_Y$), but I haven't learned the code well yet.\\

In Qi's notations, return variable \texttt{presid} is $PSR_X$ (or $PSR_Y$). Now let's see what $\pmb{A}_{Y, 4,2}$ is. Because we are in the context of linear model, $Y \sim Normal(\pmb{Z} \cdot \pmb{\theta}_Y , \sigma^2)$. Then the $PSR_Y(\pmb{\theta}_Y)$
  $$
  \begin{aligned}
    PSR_Y(\pmb{\theta}_Y) = 2F_{\pmb{\theta}_Y}(\pmb{Y}) - 1 = 2 \Phi\left(\frac{\pmb{Y} - \pmb{Z}\pmb{\theta}_Y}{\sigma}\right) - 1
  \end{aligned}
  $$
Where $F_{\pmb{\theta}_Y}(\cdot)$ is a \emph{cdf} of $Normal(\pmb{Z} \cdot \pmb{\theta}_Y , \sigma^2)$ and $\Phi(\cdot)$ is a \emph{cdf} of a standard normal, $Normal(\pmb{0}, 1)$.\\

In order to find $\pmb{A}_{Y, 4,2}$, we have to find  $\frac{\partial PSR_Y(\pmb{\theta}_Y)}{\partial \beta_0}$, $\frac{\partial PSR_Y(\pmb{\theta}_Y)}{\partial \beta_1}$, $\frac{\partial PSR_Y(\pmb{\theta}_Y)}{\partial \beta_2}$, and $ \frac{\partial PSR_Y(\pmb{\theta}_Y)}{\partial \sigma}$. In Qi's notations, they are mapped to return variable \texttt{dpresid.dtheta}:

  $$
  \begin{aligned}
    \frac{\partial PSR_{Y_i}(\pmb{\theta}_Y)}{\partial \beta_0} &= -2f\left(\frac{res_{Y_i}}{\sigma}\right)\frac{1}{\sigma}\\
    \frac{\partial PSR_{Y_i}(\pmb{\theta}_Y)}{\partial \beta_1} &= -2f\left(\frac{res_{Y_i}}{\sigma}\right)\frac{z_{i,1}}{\sigma}\\
    \frac{\partial PSR_{Y_i}(\pmb{\theta}_Y)}{\partial \beta_2} &= -2f\left(\frac{res_{Y_i}}{\sigma}\right)\frac{z_{i,2}}{\sigma}\\
    \frac{\partial PSR_{Y_i}(\pmb{\theta}_Y)}{\partial \sigma} &= +2f\left(\frac{res_{Y_i}}{\sigma}\right)\frac{res_{Y_i}}{\sigma^2}\\
  \end{aligned}
  $$

But without the last line. The remaining work is done by function \texttt{corTS()}, that I mentioned before. But, for the sake of completeness, we'll continue:
\[
\pmb{A}_{Y,4,2} = \sum_i^n\begin{bmatrix}
   2f\left(\frac{res_{Y_i}}{\sigma}\right)\frac{1}{\sigma}   &  2f\left(\frac{res_{Y_i}}{\sigma}\right)\frac{z_{i,1}}{\sigma}   & 2f\left(\frac{res_{Y_i}}{\sigma}\right)\frac{z_{i,2}}{\sigma}  &  -2f\left(\frac{res_{Y_i}}{\sigma}\right)\frac{res_{Y_i}}{\sigma^2}\\
\end{bmatrix}\]

Or
\[
\pmb{A}_{Y,4,2} = \frac{2}{\sigma}\begin{bmatrix}
   \left[f\left(\frac{\pmb{res}_Y}{\sigma}\right)\right]^T \pmb{Z}, &
   -\left[f\left(\frac{\pmb{res}_Y}{\sigma}\right)\right]^T \frac{\pmb{res}_Y}{\sigma} \\
\end{bmatrix}
\]
$A_{Y,5,2}$ and  $A_{Y,5,2}$ are computed in similar manner. To summarize, we have:

%%%%%%%%%%% A_{Y,2,2}
\[
\pmb{A}_{Y,2,2} = 
\begin{bmatrix}
  \pmb{Z}^T\pmb{Z} & \pmb{0}\\
  2 \cdot \pmb{res}_Y^T\pmb{Z} & 1\\
  \end{bmatrix}
\]
%%%%%%%%%%% A_{Y,4,2}
\[
\pmb{A}_{Y,4,2} = \frac{2}{\sigma}\begin{bmatrix}
   \left[f\left(\frac{\pmb{res}_Y}{\sigma}\right)\right]^T \pmb{Z}, &
   -\left[f\left(\frac{\pmb{res}_Y}{\sigma}\right)\right]^T \frac{\pmb{res}_Y}{\sigma} \\
\end{bmatrix}
\]
%%%%%%%%%%% A_{Y,5,2}
\[
\pmb{A}_{Y,5,2} = \frac{2}{\sigma}\begin{bmatrix}
   \left[f\left(\frac{\pmb{res}_Y}{\sigma}\right)\pmb{PSR}_X\right]^T \pmb{Z}, &
   -\left[f\left(\frac{\pmb{res}_Y}{\sigma}\right)\pmb{PSR}_X\right]^T \frac{\pmb{res}_Y}{\sigma} \\
\end{bmatrix}
\]
%%%%%%%%%%% A_{Y,7,2}
\[
\pmb{A}_{Y,7,2} = \frac{4}{\sigma}\begin{bmatrix}
   \left[f\left(\frac{\pmb{res}_Y}{\sigma}\right)\pmb{PSR}_Y\right]^T \pmb{Z}, &
   -\left[f\left(\frac{\pmb{res}_Y}{\sigma}\right)\pmb{PSR}_Y\right]^T \frac{\pmb{res}_Y}{\sigma} \\
\end{bmatrix}
\]

The above matrices are combined to get matrix $A$.

\subsection{M-estimation for parametric survival models}
In the following section, I use the on-line source written by Daowen Zhang(ST 745,  \url{http://www4.stat.ncsu.edu/~dzhang2/st745/chap5.pdf}) in order to define parametric survival models. According to Zhang, in parametric survival models we model time as the following:
  $$
  \begin{aligned}
    log(T_i) &= \beta_0 + \beta_1z_{i1} + ... + \beta_p z_{ip} + \sigma \epsilon_{i}\\
    &or\\
    T_i &= e^{\beta_0 + \beta_1z_{i1} + ... + \beta_p z_{ip} + \sigma \epsilon_{i}}\\
    &therefore\\
    S_{T_i}(t) &= P(T_i>t) = P\left( e^{\beta_0 + \beta_1z_{i1} + ... + \beta_p z_{ip} + \sigma \epsilon_{i}} >t \right ) = P\left( e^{\sigma \epsilon_{i}} > e^{-\beta_0 - \beta_1z_{i1} - ... - \beta_p z_{ip}}t \right )=\\
    &=S_{e^{\sigma \epsilon_{i}}}(e^{-\pmb{z}_i \pmb{\beta}}t)\\
  \end{aligned}
  $$
Where $e^{\sigma \epsilon_{i}}$ can have variety of distributions including Weibull, exponential, gamma, log-logistic, and log-normal.\\
In the following sections we denote random variable, time to outcome, as $Y_i$ and its values as $y_i$. 

\subsubsection{M-estimation for \emph{exponential} survival model}
\label{MEst:exponential}

For exponential survival model, we assume that $\sigma = 1$, and:
  $$
  \begin{aligned}
    e^{\epsilon_{i}} &\sim Exponential(e^{-\pmb{z}_i \pmb{\beta}})\\
    F_{e^{\epsilon_{i}}}(y_i) &= 1 - e^{-(e^{-\pmb{z}_i \pmb{\beta}})y_i}\\
    S_{e^{\epsilon_{i}}}(y_i) &= e^{-(e^{-\pmb{z}_i \pmb{\beta}})y_i}\\
    f_{e^{\epsilon_{i}}}(y_i) &= e^{-\left(\pmb{z}_i \pmb{\beta}\right)}e^{-(e^{-\pmb{z}_i \pmb{\beta}})y_i}\\
  \end{aligned}
  $$
The likelihood and the log-likelihood are the following:
  $$
  \begin{aligned}
    L &= \prod_{i=1}^N  \left[ e^{-\left(\pmb{z}_i \pmb{\beta}\right)}e^{-(e^{-\pmb{z}_i \pmb{\beta}})y_i} \right]^{\delta_i} \left[e^{-(e^{-\pmb{z}_i \pmb{\beta}})y_i}\right]^{1-\delta_i} =\prod_{i=1}^N   \left[e^{-\left(\pmb{z}_i \pmb{\beta}\right)}\right]^{\delta_i}   e^{-(e^{-\pmb{z}_i \pmb{\beta}})y_i} = \prod_{i=1}^N   e^{-\pmb{z}_i \pmb{\beta}\delta_i-(e^{-\pmb{z}_i \pmb{\beta}})y_i}=\prod_{i=1}^N L_{Y_i}\\
    l_i &=\sum_{i=1}^N log(L_{Y_i}) =\sum_{i=1}^N  \left[  -\pmb{z}_i \pmb{\beta}\delta_i-(e^{-\pmb{z}_i \pmb{\beta}})y_i \right]
  \end{aligned}\\
  $$
If $\pmb{\beta} = (\beta_0, \beta_1, \beta_2)$ then:
  $$
  \begin{aligned}
    l_{Y_i} &=   -\beta_0\delta_i-\beta_1 z_{i1}\delta_i-\beta_2z_{i2}\delta_i-(e^{-\beta_0-\beta_1 z_{i1}-\beta_2z_{i2} })y_i \\
    \psi_{Yi,\beta_0} &= \frac{\partial l_i}{\partial \beta_0} =  -\delta_i+\left(e^{-\pmb{z}_i \pmb{\beta}}\right)y_i\\
    \psi_{Yi,\beta_1} &= \frac{\partial l_i}{\partial \beta_1} =  \left\{-\delta_i+\left(e^{-\pmb{z}_i \pmb{\beta}}\right)y_i\right\}z_{i1} \\
    \psi_{Yi,\beta_2} &= \frac{\partial l_i}{\partial \beta_2} =  \left\{-\delta_i+\left(e^{-\pmb{z}_i \pmb{\beta}}\right)y_i\right\}z_{i2} \\
  \end{aligned}
  $$
  
Then:
  \[
  \pmb{A}_{Y,2,2} = -\sum_i^n\begin{bmatrix}
      \frac{\partial \pmb{\psi}_{Yi,\beta_0}}{ \partial \beta_0} &
      \frac{\partial \pmb{\psi}_{Yi,\beta_0}}{ \partial \beta_1} &
      \frac{\partial \pmb{\psi}_{Yi,\beta_0}}{ \partial \beta_2} \\
      %%%%%%%%%
      \frac{\partial \pmb{\psi}_{Yi,\beta_1}}{ \partial \beta_0} &
      \frac{\partial \pmb{\psi}_{Yi,\beta_1}}{ \partial \beta_1} &
      \frac{\partial \pmb{\psi}_{Yi,\beta_1}}{ \partial \beta_2} \\
      %%%%%%%%%
      \frac{\partial \pmb{\psi}_{Yi,\beta_2}}{ \partial \beta_0} &
      \frac{\partial \pmb{\psi}_{Yi,\beta_2}}{ \partial \beta_1} &
      \frac{\partial \pmb{\psi}_{Yi,\beta_2}}{ \partial \beta_2}\\
  \end{bmatrix}
   =-\sum_i^n\begin{bmatrix}
      \frac{\partial^2 l_{Y_i}}{ \partial \beta_0\partial \beta_0}  & \frac{\partial^2 l_{Y_i}}{ \partial \beta_0\partial \beta_1} & \frac{\partial^2 l_{Y_i}}{ \partial \beta_0\partial \beta_2} \\
      \frac{\partial^2 l_{Y_i}}{ \partial \beta_1\partial \beta_o}  & \frac{\partial^2 l_{Y_i}}{ \partial \beta_1\partial \beta_1} & \frac{\partial^2 l_{Y_i}}{ \partial \beta_1\partial \beta_2}\\
      \frac{\partial^2 l_{Y_i}}{ \partial \beta_2\partial \beta_0}  & \frac{\partial^2 l_{Y_i}}{ \partial \beta_2\partial \beta_1} & \frac{\partial^2 l_{Y_i}}{ \partial \beta_2\partial \beta_2}
  \end{bmatrix} =
  \]
  \[
    =\sum_i^n  \left(e^{-\pmb{z}_i \pmb{\beta}}\right)y_i
  \begin{bmatrix}
      1   &  z_{i1}   & z_{i2}   \\
      z_{i1}   &  z_{i1}^2   & z_{i1}z_{i2} \\
      z_{i2}   &  z_{i1}z_{i2}   & z_{i2}^2  \\
  \end{bmatrix}=\]\\
  \[
  =
  \begin{bmatrix}
      \pmb{C}^T\pmb{Z} \\
  \end{bmatrix} 
  \]
  
  Where
  \[
  \pmb{C} = 
  \begin{bmatrix}
      \left(e^{-\pmb{z}_1 \pmb{\beta}}\right)y_1   &  \left(e^{-\pmb{z}_1 \pmb{\beta}}\right)y_1  & \left(e^{-\pmb{z}_1 \pmb{\beta}}\right)y_1 \\
      \left(e^{-\pmb{z}_2 \pmb{\beta}}\right)y_2   &  \left(e^{-\pmb{z}_2 \pmb{\beta}}\right)y_2  & \left(e^{-\pmb{z}_2 \pmb{\beta}}\right)y_2 \\
       .......................... \\
      \left(e^{-\pmb{z}_N \pmb{\beta}}\right)y_N   &  \left(e^{-\pmb{z}_N \pmb{\beta}}\right)y_N  & \left(e^{-\pmb{z}_N \pmb{\beta}}\right)y_N \\
  \end{bmatrix} 
    \circ \pmb{Z} \\
  \]

Where $\circ$ is a \emph{Hadamard} product or element-wise multiplication.

We only need to figure out the following derivatives: $\frac{\partial PSR_Y}{\partial \beta_0}$, $\frac{\partial PSR_Y}{\partial \beta_1}$, $\frac{\partial PSR_Y}{\partial \beta_2}$. Because $PSR_{Y_i} = (1+\delta_i)F_Y(y_i) - \delta_i = (1+\delta_i)(1 - S_Y(y_i)) - \delta_i = 1 - (1+\delta_i)S_Y(y_i)$, we have:
  $$
  \begin{aligned}
    PSR_{Y_i} &=  1 - (1+\delta_i)e^{-(e^{-\pmb{z}_i \pmb{\beta}})y_i} \\
    \frac{\partial PSR_{Y_i}}{\partial \beta_0} &= - (1+\delta_i)e^{-(e^{-\pmb{z}_i \pmb{\beta}})y_i}  \left[ -(e^{-\pmb{z}_i \pmb{\beta}})y_i \right] (-1)=\\
    &= - (1+\delta_i)y_i e^{-(e^{-\pmb{z}_i \pmb{\beta}})y_i-\pmb{z}_i \pmb{\beta}}\\
    \frac{\partial PSR_{Y_i}}{\partial \beta_1} &= - (1+\delta_i)~y_i ~z_{i1}~ e^{-(e^{-\pmb{z}_i \pmb{\beta}})y_i -\pmb{z}_i \pmb{\beta}}\\
    \frac{\partial PSR_{Y_i}}{\partial \beta_2} &= - (1+\delta_i)~y_i ~z_{i2}~ e^{-(e^{-\pmb{z}_i \pmb{\beta}})y_i -\pmb{z}_i \pmb{\beta}}\\
  \end{aligned}
  $$

Now we can feed all this into function \texttt{corTS} and check the result using simulations. One of the ways to make sure we are doing the right thing is to compare our matrix $A_n$ to the variance-covariance matrix of the regression with exponential distribution. They should be equal. Next, we can use library \texttt{sandwich}, in order to see if matrices $n\cdot(A_n)^{-1}$ and $B_n/n$ are \emph{bread} and \emph{meat} that library  \texttt{sandwich} computes. This should be done for each model separately because otherwise the results won't be exactly the same.

\subsubsection{M-estimation for \emph{log-logistic} survival model}
\label{MEst:loglogistic}

Language $R$ does not have \emph{log-logistic} distribution, but we still need to simulate it for our work. We first simulate $\sigma\epsilon_i = log(Y_i)$ and then get the time as $Y_i = e^{\sigma\epsilon}$. I am \textbf{guessing} that $\epsilon$ is distributed as logistic random variable. Let's write it down:
  $$
  \begin{aligned}
    P(log(Y_i) \leq x) &= P(\pmb{z}_i\pmb{\beta} + \sigma\epsilon \leq x) = P\left(\epsilon \leq \frac{ x-\pmb{z}_i\pmb{\beta} }{\sigma}\right) = \frac{  1  }{  1 + e^{-\frac{ x-\pmb{z}_i\pmb{\beta} }{\sigma}}} \\
  \end{aligned}
  $$
Which can be rewritten as:
  $$
  \begin{aligned}
    P(log(Y_i) \leq x) &= \frac{  1  }{  1 + y_i^{-\frac{1}{\sigma}} e^{-\frac{-\pmb{z}_i\pmb{\beta}}{\sigma}}   } \\
  \end{aligned}
  $$


But this did not work because of (I am \textbf{guessing}) different parametrization. So I tried a different parametrization where $\sigma$ is \emph{absorbed} into $\beta$ coefficients, so we have according to \cite{bennett1983log}:

 $$
 \begin{aligned}
   F_{Y_i}(y_i) &= \frac{  1  }{  1 + y_i^{-\gamma} e^{-\pmb{z}_i\pmb{\beta}}   } \\
  \end{aligned}
  $$
  
And it did not work either. So I found a presentation of \emph{Christoph Dätwyler} and \emph{Timon Stucki}(\url{https://stat.ethz.ch/education/semesters/ss2011/seminar/contents/handout_9.pdf}), and according to them, for log-logistic distribution, \texttt{R} uses parametrization of \emph{accelerated failure time} AFT model, which is:
 $$
 \begin{aligned}
   F_{Y_i}(y_i) &= \frac{  1  }{  1 + y_i^{-\gamma} \left(e^{-\pmb{z}_i\pmb{\beta}} \right)^{-\gamma}   } \\
   S_{Y_i}(y_i) &= \frac{  y_i^{-\gamma}\left(e^{-\pmb{z}_i\pmb{\beta}} \right)^{-\gamma}   }{  1 + y_i^{-\gamma} \left(e^{-\pmb{z}_i\pmb{\beta}} \right)^{-\gamma}   } \\
  f_{Y_i}(y_i) &=   \frac{  \gamma y_i^{-\gamma-1} \left(e^{-\pmb{z}_i\pmb{\beta}} \right)^{-\gamma}  }{  \left(1 + y_i^{-\gamma} \left(e^{-\pmb{z}_i\pmb{\beta}} \right)^{-\gamma} \right)^2   } \\
  \end{aligned}
  $$


The likelihood and the log-likelihood are the following:
  $$
  \begin{aligned}
    L &= \prod_{i=1}^N  \left[    \frac{  \gamma y_i^{-\gamma-1} \left(e^{-\pmb{z}_i\pmb{\beta}} \right)^{-\gamma}  }{  \left(1 + y_i^{-\gamma} \left(e^{-\pmb{z}_i\pmb{\beta}} \right)^{-\gamma} \right)^2   } \right]^{\delta_i} \left[  \frac{  y_i^{-\gamma}\left(e^{-\pmb{z}_i\pmb{\beta}} \right)^{-\gamma}   }{  1 + y_i^{-\gamma} \left(e^{-\pmb{z}_i\pmb{\beta}} \right)^{-\gamma}   }  \right]^{1-\delta_i} = \prod_{i=1}^N      \frac{  (\gamma)^{\delta_i}   (y_i)^{-\delta_i-\gamma}   \left(e^{-\pmb{z}_i\pmb{\beta}} \right)^{-\gamma}   }{  \left(1 + y_i^{-\gamma} \left(e^{-\pmb{z}_i\pmb{\beta}} \right)^{-\gamma}  \right)^{\delta_i + 1}  } \\
    l_{Y_i} &=\sum_{i=1}^N log(L_{Y_i}) =\sum_{i=1}^N  \left[    \delta_i log(\gamma)   - (\delta_i + \gamma) log(y_i) +\pmb{z}_i\pmb{\beta}\gamma   - (\delta_i + 1)log\left(1 + y_i^{-\gamma} \left(e^{-\pmb{z}_i\pmb{\beta}} \right)^{-\gamma} \right)\right]
  \end{aligned}\\
  $$
If $\theta = (\pmb{\beta}, \gamma) = (\beta_0, \beta_1, \gamma)$ then:

  $$
  \begin{aligned}
    l_{Y_i} &=    \delta_i log(\gamma)   - (\delta_i + \gamma) log(y_i) +\pmb{z}_i\pmb{\beta}\gamma   - (\delta_i + 1)log\left(1 + y_i^{-\gamma} \left(e^{-\pmb{z}_i\pmb{\beta}} \right)^{-\gamma} \right)\\
    \psi_{Yi,\beta_0} &= \frac{\partial l_i}{\partial \beta_0} =  \gamma \left\{1 - \frac{  (1+\delta_i)y_i^{-\gamma} \left(e^{\pmb{z}_i\pmb{\beta}\gamma}\right)  }{1 +  y_i^{-\gamma} \left(e^{\pmb{z}_i\pmb{\beta}\gamma}\right)}    \right\}   =    \gamma \left\{1 - \frac{  (1+\delta_i)  }{1 +  y_i^{\gamma} \left(e^{-\pmb{z}_i\pmb{\beta}\gamma}\right)}    \right\}\\
    \psi_{Yi,\beta_1} &= \frac{\partial l_i}{\partial \beta_1} =   z_{i1}\gamma \left\{1 - \frac{  (1+\delta_i)  }{1 +  y_i^{\gamma} \left(e^{-\pmb{z}_i\pmb{\beta}\gamma}\right)}    \right\}\\
    \psi_{Yi,\gamma} &= \frac{\partial l_i}{\partial \gamma} =  \frac{\delta_i}{\gamma}  - log(y_i) + \pmb{z}_i\pmb{\beta}  - \frac{  (1+\delta_i)   \left(  -log(y_i) + \pmb{z}_i\pmb{\beta} \right)  }{1 +  y_i^{\gamma} e^{-\pmb{z}_i\pmb{\beta}\gamma}} \\
  \end{aligned}
  $$
When computing score equations, language \texttt{R} uses $\gamma^* = -log(\gamma)$. If we take a derivative according to $\gamma^*$, the last score equation is:
  $$
  \begin{aligned}
    \psi_{Yi,\gamma^*} &= \frac{\partial l_i}{\partial \gamma^*} =  -\delta_i  + e^{-\gamma^*}\left\{log(y_i)  - \pmb{z}_i\pmb{\beta} +  \frac{  (1+\delta_i)\left(  -log(y_i) + \pmb{z}_i\pmb{\beta} \right)  }{1 +  \left[y_i e^{-\pmb{z}_i\pmb{\beta}}\right]^{ e^{-\gamma^*}}    }   \right\}\\
  \end{aligned}
  $$
When we actually compute it we use $\gamma$ again:
  $$
  \begin{aligned}
    \psi_{Yi,\gamma^*} &=\left.  \frac{\partial l_i}{\partial \gamma^*} \right|_{\gamma^* = -log(\gamma)} =  -\delta_i  + \gamma\left\{log(y_i)  - \pmb{z}_i\pmb{\beta} +  \frac{  (1+\delta_i)\left(  -log(y_i) + \pmb{z}_i\pmb{\beta} \right)  }{1 +  \left[y_i e^{-\pmb{z}_i\pmb{\beta}}\right]^{ \gamma}    }   \right\}\\
  \end{aligned}
  $$

Let's move onto the second derivatives:
  \[
  \pmb{A}_{Y,2,2} = -\sum_i^n\begin{bmatrix}
      \frac{\partial \pmb{\psi}_{Yi,\beta_0}}{ \partial \beta_0} &
      \frac{\partial \pmb{\psi}_{Yi,\beta_0}}{ \partial \beta_1} &
      \frac{\partial \pmb{\psi}_{Yi,\beta_0}}{ \partial \gamma^*} \\
      %%%%%%%%%
      \frac{\partial \pmb{\psi}_{Yi,\beta_1}}{ \partial \beta_0} &
      \frac{\partial \pmb{\psi}_{Yi,\beta_1}}{ \partial \beta_1} &
      \frac{\partial \pmb{\psi}_{Yi,\beta_1}}{ \partial \gamma^*} \\
      %%%%%%%%%
      \frac{\partial \pmb{\psi}_{Yi,\gamma^*}}{ \partial \beta_0} &
      \frac{\partial \pmb{\psi}_{Yi,\gamma^*}}{ \partial \beta_1} &
      \frac{\partial \pmb{\psi}_{Yi,\gamma^*}}{ \partial \gamma^*}\\
  \end{bmatrix}
   =-\sum_i^n\begin{bmatrix}
      \frac{\partial^2 l_{Y_i}}{ \partial \beta_0\partial \beta_0}  & \frac{\partial^2 l_{Y_i}}{ \partial \beta_0\partial \beta_1} & \frac{\partial^2 l_{Y_i}}{ \partial \beta_0\partial \gamma^*} \\
      \frac{\partial^2 l_{Y_i}}{ \partial \beta_1\partial \beta_o}  & \frac{\partial^2 l_{Y_i}}{ \partial \beta_1\partial \beta_1} & \frac{\partial^2 l_{Y_i}}{ \partial \beta_1\partial \gamma^*}\\
      \frac{\partial^2 l_{Y_i}}{ \partial \gamma^*\partial \beta_0}  & \frac{\partial^2 l_{Y_i}}{ \partial \gamma^*\partial \beta_1} & \frac{\partial^2 l_{Y_i}}{ \partial \gamma^*\partial \gamma^*}
  \end{bmatrix}
  \]
  
Now, this is a handful, so we should go slowly and figure out each of these: 
  $$
  \begin{aligned}
    \left. \frac{\partial^2 l_{Y_i}}{ \partial \beta_0\partial \beta_0}  \right|_{\gamma^* = -log(\gamma)} &= - \gamma^2(1+\delta_i)  \frac{  \left[y_i e^{-\pmb{z}_i\pmb{\beta}}\right]^{ \gamma}   }{  \left( 1 +  \left[y_i e^{-\pmb{z}_i\pmb{\beta}}\right]^{ \gamma} \right)^2  } \\
    \left. \frac{\partial^2 l_{Y_i}}{ \partial \beta_0\partial \beta_1}  \right|_{\gamma^* = -log(\gamma)} &= - \gamma^2 z_{i1} (1+\delta_i)  \frac{  \left[y_i e^{-\pmb{z}_i\pmb{\beta}}\right]^{ \gamma}   }{   \left( 1 + \left[y_i e^{-\pmb{z}_i\pmb{\beta}}\right]^{ \gamma} \right)^2  } \\
    \left. \frac{\partial^2 l_{Y_i}}{ \partial \beta_1\partial \beta_1}  \right|_{\gamma^* = -log(\gamma)} &= - \gamma^2  z_{i1}^2 (1+\delta_i)  \frac{  \left[y_i e^{-\pmb{z}_i\pmb{\beta}}\right]^{ \gamma}   }{   \left(1 +  \left[y_i e^{-\pmb{z}_i\pmb{\beta}}\right]^{ \gamma} \right)^2  } \\
    %%%%%%%%%%%% derive by gamma:
    \left. \frac{\partial^2 l_{Y_i}}{ \partial \beta_0\partial \gamma^*}  \right|_{\gamma^* = -log(\gamma)} &= - \gamma   \left[   1 -   (1+\delta_i)  \frac{   1 +   \left[y_i e^{-\pmb{z}_i\pmb{\beta}}\right]^{ \gamma}     -\gamma \left( log(y_i) - \pmb{z}_i\pmb{\beta} \right)\left[y_i e^{-\pmb{z}_i\pmb{\beta}}\right]^{ \gamma} }{  \left( 1+\left[y_i e^{-\pmb{z}_i\pmb{\beta}}\right]^{ \gamma} \right)^2  }  \right]\\
    \left. \frac{\partial^2 l_{Y_i}}{ \partial \beta_1\partial \gamma^*}  \right|_{\gamma^* = -log(\gamma)} &= - \gamma  z_{i1}  \left[   1 -   (1+\delta_i)  \frac{   1 +   \left[y_i e^{-\pmb{z}_i\pmb{\beta}}\right]^{ \gamma}     -\gamma \left( log(y_i) - \pmb{z}_i\pmb{\beta} \right)\left[y_i e^{-\pmb{z}_i\pmb{\beta}}\right]^{ \gamma} }{  \left( 1+\left[y_i e^{-\pmb{z}_i\pmb{\beta}}\right]^{ \gamma} \right)^2  }  \right]\\
    \left. \frac{\partial^2 l_{Y_i}}{ \partial  \gamma^*  \partial \gamma^*}  \right|_{\gamma^* = -log(\gamma)} &= - \gamma   \left[   log(y_i)  -  \pmb{z}_i\pmb{\beta} +  (1+\delta_i)  \frac{   \left( log(y_i) - \pmb{z}_i\pmb{\beta} \right)^2  \left[y_i e^{-\pmb{z}_i\pmb{\beta}}\right]^{ \gamma} }{  \left( 1+\left[y_i e^{-\pmb{z}_i\pmb{\beta}}\right]^{ \gamma} \right)^2  }  \right]\\
  \end{aligned}
  $$
    
  \[
    \pmb{A}_{Y,2,2} =
  \begin{bmatrix}
      c_i   &  c_i z_{i1}   & d_i   \\
      c_i z_{i1}   &  c_i z_{i1}^2   & d_i z_{i1} \\
      d_i   &  d_i z_{i1}   & d_{\gamma,\gamma}  \\
  \end{bmatrix}\]\\

  Where
  $$
  \begin{aligned}
    c_i &=  \gamma^2(1+\delta_i)  \frac{  \left[y_i e^{-\pmb{z}_i\pmb{\beta}}\right]^{ \gamma}   }{  \left( 1 +  \left[y_i e^{-\pmb{z}_i\pmb{\beta}}\right]^{ \gamma} \right)^2  } \\
    %%%%%%%%%%%% derive by gamma:
    d_i &=  \gamma   \left[   1 -   (1+\delta_i)  \frac{   1 +   \left[y_i e^{-\pmb{z}_i\pmb{\beta}}\right]^{ \gamma}     -\gamma \left( log(y_i) - \pmb{z}_i\pmb{\beta} \right)\left[y_i e^{-\pmb{z}_i\pmb{\beta}}\right]^{ \gamma} }{  \left( 1+\left[y_i e^{-\pmb{z}_i\pmb{\beta}}\right]^{ \gamma} \right)^2  }  \right]\\
    d_{\gamma,\gamma} &=  \gamma   \left[   log(y_i)  -  \pmb{z}_i\pmb{\beta} +  (1+\delta_i)  \frac{   \left( log(y_i) - \pmb{z}_i\pmb{\beta} \right)^2  \left[y_i e^{-\pmb{z}_i\pmb{\beta}}\right]^{ \gamma} }{  \left( 1+\left[y_i e^{-\pmb{z}_i\pmb{\beta}}\right]^{ \gamma} \right)^2  }  \right]\\
  \end{aligned}
  $$
  
  
  
  
  \[
  \pmb{C} = 
  \begin{bmatrix}
      c_1   &  c_1  & c_1 \\
      c_2   &  c_2  & c_2 \\
       ... \\
      c_N   &  c_N  & c_N \\
  \end{bmatrix} 
    \circ \pmb{Z} \\
  \]

Where $\circ$ is a \emph{Hadamard} product or element-wise multiplication, and:
  $$
  \begin{aligned}
    c_i &=  \frac{\left(e^{-\pmb{z}_i \pmb{\beta}}\right)y_i }{\sigma}   \left\{  1 + \frac{(\delta_i + 1)\left( \frac{\left(e^{-\pmb{z}_i \pmb{\beta}}\right)y_i }{\sigma}-1 \right)  }{ 1 + e^{\frac{\left(e^{-\pmb{z}_i \pmb{\beta}}\right)y_i }{\sigma}} }  \right\}\\
  \end{aligned}
  $$


We only need to figure out the following derivatives: $\frac{\partial PSR_Y}{\partial \beta_0}$, $\frac{\partial PSR_Y}{\partial \beta_1}$, $\frac{\partial PSR_Y}{\partial \beta_2}$. Because $PSR_{Y_i} = (1+\delta_i)F_Y(y_i) - \delta_i = (1+\delta_i)(1 - S_Y(y_i)) - \delta_i = 1 - (1+\delta_i)S_Y(y_i)$, we have:
  $$
  \begin{aligned}
    PSR_{Y_i} &=  1 - (1+\delta_i) \frac{1}{1+\left[y_i e^{-\pmb{z}_i\pmb{\beta}}\right]^{ \gamma}} \\
    \frac{\partial PSR_{Y_i}}{\partial \beta_0} &= - \gamma(1+\delta_i) \frac{  \left[y_i e^{-\pmb{z}_i\pmb{\beta}}    \right]^{ \gamma}  }{    \left(1+\left[y_i e^{-\pmb{z}_i\pmb{\beta}}    \right]^{ \gamma}\right)^2} \\
    \frac{\partial PSR_{Y_i}}{\partial \beta_1} &= - \gamma(1+\delta_i) \frac{  \left[y_i e^{-\pmb{z}_i\pmb{\beta}}    \right]^{ \gamma}  }{    \left(1+\left[y_i e^{-\pmb{z}_i\pmb{\beta}}    \right]^{ \gamma}\right)^2}  z_{i1}\\
    \left. \frac{\partial PSR_{Y_i}}{\partial \gamma^*} \right|_{\gamma^* = -log(\gamma)} &= ~~~ \gamma(1+\delta_i)\frac{    \left[y_i e^{-\pmb{z}_i\pmb{\beta}}    \right]^{ \gamma}  }{    \left(1+\left[y_i e^{-\pmb{z}_i\pmb{\beta}}    \right]^{ \gamma}\right)^2}(log(y_i) - \pmb{z}_i\pmb{\beta}) \\
  \end{aligned}
  $$

I coded the above derivations and compared the results with the \texttt{sandwich} package. The \emph{meat} part matched exactly, but the \emph{bread} part almost matched. I did not pursue this further because both \emph{meat} and \emph{bread} can be extracted from the model. The only thing that we need to make sure is that when a survival model has $\sigma$,  the score equations are according to $\sigma$ or $log(\sigma)$.\\
Qi has already written a function that computes \emph{PSR} for each distribution of \emph{survreg}. The only expressions that need to be coded are:  $\frac{\partial PSR_{Y_i}}{\partial \beta_i}$ and $\left. \frac{\partial PSR_{Y_i}}{\partial \gamma^*} \right|_{\gamma^* = -log(\gamma)}$. For log-logistic distribution these are defined above. For the rest of the distributions, we define them in the next section.

\subsubsection{M-estimation for \emph{Weibull} survival model}
\label{MEst:weibull}

In \textbf{R}, Weibull distribution (functions like \texttt{rweibull()}) is parameterized in the following way:

  $$
  \begin{aligned}
    F(y)  = 1-exp\left(-\left(\frac{y}{scale}\right)^{shape}\right)
  \end{aligned}
  $$
But we are going to use different parameterization because this is what \texttt{survreg()} function is using:
  $$
  \begin{aligned}
    scale &= e^{\pmb{z}_i\pmb{\beta}};~~~~~~~~~~shape =\frac{1}{\sigma}\\
    F(y_i)  &= 1-exp\left(-\left(\frac{y}{e^{\pmb{z}_i\pmb{\beta}}}\right)^{\frac{1}{\sigma}}\right)\\
  \end{aligned}
  $$
So the PSR and their derivatives by $\beta_j$ can be written as:
  $$
  \begin{aligned}
    PSR_{Y_i} &=  (1+\delta_i) \left[  1-exp\left(-\left(\frac{y}{e^{\pmb{z}_i\pmb{\beta}}}\right)^{\frac{1}{\sigma}}\right)    \right]  -  \delta_i\\
    \frac{\partial PSR_{Y_i}}{\partial \beta_j} &= -(1+\delta_i)\frac{z_{ji}}{\sigma}  y_i^{\frac{1}{\sigma}} e^{ \left( - y_i\cdot e^{-\pmb{z}_i\pmb{\beta}} \right)^{\frac{1}{\sigma}}  - \frac{\pmb{z}_i\pmb{\beta}}{\sigma} } \\
  \end{aligned}
  $$

My \textbf{guess} is that R-code takes a derivative of $log(\sigma)$, so let's re-parameterize the distribution accordingly: 

  $$
  \begin{aligned}
    \gamma &= ln(\sigma);~~~~~~~ \sigma = e^{\gamma};~~~~~~~ \frac{1}{\sigma} = e^{-\gamma}\\
    %%%%%%%%%%%%%%%%%%
    F(y_i)  &= 1-exp\left(-\left(\frac{y}{e^{\pmb{z}_i\pmb{\beta}}}\right)^{e^{-\gamma}}\right)\\
    PSR_{Y_i} &=  (1+\delta_i) \left[  1-exp\left(-\left(\frac{y_i}{e^{\pmb{z}_i\pmb{\beta}}}\right)^{e^{-\gamma}}\right)    \right]  -  \delta_i\\
    %%%%%%%%%%%%%%%%%%
    \frac{\partial PSR_{Y_i}}{\partial \gamma} &= (1+\delta_i)   \left[   -exp\left(-\left(y_i e^{-\pmb{z}_i\pmb{\beta}}\right)^{e^{-\gamma}}\right)   \frac{\partial}{\partial \gamma}\left(    -\left(   y_i e^{-\pmb{z}_i\pmb{\beta}}  \right) ^{e^{-\gamma}} \right)   \right] =\\
    & = (1+\delta_i)      exp\left(-\left(y_i e^{-\pmb{z}_i\pmb{\beta}}\right)^{e^{-\gamma}}\right)   ln\left( y_i e^{-\pmb{z}_i\pmb{\beta}}  \right)        \left(   y_i e^{-\pmb{z}_i\pmb{\beta}}  \right) ^{e^{-\gamma}} \left( -e^{-\gamma} \right)   = \\
    %%%%%%%%%%%%%%%%%%
    & = - (1+\delta_i)   \left(  ln(y_i) -\pmb{z}_i\pmb{\beta}  \right)     y_i ^{e^{-\gamma}} exp \left\{    -\left(y_i e^{-\pmb{z}_i\pmb{\beta}}\right)^{e^{-\gamma}}      -\pmb{z}_i\pmb{\beta}e^{-\gamma}      -\gamma \right\}   = \\
    %%%%%%%%%%%%%%%%%%
    \left. \frac{\partial PSR_{Y_i}}{\partial \gamma} \right|_{\gamma = log(\sigma)} &= - (1+\delta_i)   \left(  ln(y_i) -\pmb{z}_i\pmb{\beta}  \right)     y_i ^{\frac{1}{\sigma}} exp \left\{    -\left(y_i e^{-\pmb{z}_i\pmb{\beta}}\right)^{\frac{1}{\sigma}}      -\frac{\pmb{z}_i\pmb{\beta}}{\sigma}     -ln(\sigma) \right\}\\
  \end{aligned}
  $$

\subsubsection{M-estimation for \emph{Lognormal} survival model}
\label{MEst:lognormal}

% \textbf{Gaussian} distribution:\\
% \textbf{Logistic} distribution:\\
  $$
  \begin{aligned}
    f(y) &= \frac{1}{y \sigma\sqrt{2 \pi }} e^{-\frac{(log(y) - \pmb{z}_i\pmb{\beta})^2}{2\sigma^2}}\\
    F(y) &= \Phi\left( \frac{ln(y) - \pmb{z}_i\pmb{\beta}}{\sigma}  \right)  % \frac{1}{2} + \frac{1}{\sqrt{\pi}} \int_0^y  e^{-\frac{(log(t) - \pmb{z}_i\pmb{\beta})^2}{2\sigma^2}} dt = \\
  \end{aligned}
  $$
Where $\Phi\left( \frac{ln(y) - \pmb{z}_i\pmb{\beta}}{\sigma}  \right)$ is a \emph{cdf} of \emph{normal} distribution.
So the PSR and their derivatives by $\beta_j$ can be written as:
  $$
  \begin{aligned}
    PSR_{Y_i} &=  (1+\delta_i) F(y)  -  \delta_i = (1+\delta_i)\Phi\left( \frac{ln(y_i) - \pmb{z}_i\pmb{\beta}}{\sigma}  \right)  -  \delta_i\\
  \end{aligned}
  $$
  
We know that if we take a derivative by $y$ we'll get $\frac{1}{y \sigma\sqrt{2 \pi }} e^{-\frac{(log(y) - \pmb{z}_i\pmb{\beta})^2}{2\sigma^2}}$. $\frac{1}{y}$ appears because following the chain rule we multiply by $\frac{d ln(y)}{d y}$ so if we take a derivative by $\beta_i$ we get about the same thing except that, due to the chain rule, we have to multiple it by $  \frac{\partial \left(log(y_i) - \pmb{z}_i\pmb{\beta} \right)}{\partial \beta_i}  $, therefore, we get:
  $$
  \begin{aligned}
    \frac{\partial PSR_{Y_i}}{\partial \beta_j} &= -(1+\delta_i)z_{ji}   \frac{1}{\sqrt{2\pi} \sigma}    e^{ - \frac{\left( ln(y_i) -\pmb{z}_i\pmb{\beta} \right)^2}{2\sigma^2} } \\
  \end{aligned}
  $$

Now, in order to check ourselves, let's do the same thing, but using the \emph{Leibniz} rule of taking a derivative under the integral (see \url{https://en.wikipedia.org/wiki/Leibniz_integral_rule}):
  $$
  \begin{aligned}
    \frac{d\phi}{d\alpha} = \int_a^b   \frac{\partial}{\partial\alpha}f(x, \alpha)dx +  f(b, \alpha)\frac{\partial b}{\partial\alpha}  - f(a, \alpha)\frac{\partial a}{\partial\alpha} \\
  \end{aligned}
  $$
In our case $\frac{\partial a}{\partial\beta_i} = \frac{\partial b}{\partial\beta_i} = 0$, so we have:
  $$
  \begin{aligned}
    \frac{\partial PSR_{Y_i}}{\partial \beta_j} &=(1+\delta_i) \int_0^y  \frac{\partial}{\partial \beta_j}  \frac{1}{t \sigma\sqrt{2 \pi }} e^{-\frac{(log(t) - \pmb{z}_i\pmb{\beta})^2}{2\sigma^2}} dt = (1+\delta_i) \int_0^y   \frac{1}{t \sigma\sqrt{2 \pi }} e^{-\frac{(log(t) - \pmb{z}_i\pmb{\beta})^2}{2\sigma^2}}  \left(   -\frac{1}{2\sigma^2}   \right)  2(ln(t) - \pmb{z}_i\pmb{\beta})(-z_{ji})dt  =\\
    &= -(1+\delta_i)z_{ji} \int_0^y   \frac{1}{\sigma\sqrt{2 \pi }} e^{-\frac{(log(t) - \pmb{z}_i\pmb{\beta})^2}{2\sigma^2}}  d\left(   -\frac{(log(t) - \pmb{z}_i\pmb{\beta})^2}{2\sigma^2}   \right) =  -(1+\delta_i)z_{ji}  \frac{1}{\sqrt{2\pi} \sigma}    e^{ - \frac{\left( ln(y_i) -\pmb{z}_i\pmb{\beta} \right)^2}{2\sigma^2} }\\
  \end{aligned}
  $$
Which is the same thing. Now, let's do $\frac{\partial PSR_{Y_i}}{\partial \sigma}$ using the \emph{Leibniz} rule. \texttt{R} language probably uses $\gamma = log(\sigma)$, so we have: 
  $$
  \begin{aligned}
    \gamma &= ln(\sigma);~~~~~  \sigma= e^{\gamma}\\
    \frac{\partial PSR_{Y_i}}{\partial \gamma} &=(1+\delta_i) \int_0^{y_i}  \frac{\partial}{\partial \gamma}  \frac{e^{-\gamma}}{t \sqrt{2 \pi }} e^{-\frac{(ln(t) - \pmb{z}_i\pmb{\beta})^2}{2}e^{-2\gamma}} dt = \\
    &=(1+\delta_i) \int_0^{y_i}   \frac{1}{t \sqrt{2 \pi }} e^{-\frac{(ln(t) - \pmb{z}_i\pmb{\beta})^2}{2}e^{-2\gamma} - \gamma} \left[   \cancel{-\frac{1}{2}} \left(  ln(t) -  \pmb{z}_i\pmb{\beta}  \right)^2  e^{-2\gamma}\cancel{(-2)} - 1  \right] dt  =\\
    &=(1+\delta_i) \left[       \int_0^{y_i}   \frac{1}{t \sqrt{2 \pi }}  \left(  ln(t) -  \pmb{z}_i\pmb{\beta}  \right)^2 e^{-\frac{(ln(t) - \pmb{z}_i\pmb{\beta})^2}{2}e^{-2\gamma} - 3\gamma} dt     -     \int_0^{y_i}   \frac{1}{t \sqrt{2 \pi }}  e^{-\frac{(ln(t) - \pmb{z}_i\pmb{\beta})^2}{2}e^{-2\gamma} - \gamma}  dt  \right] = \\
    & =(1+\delta_i) \left[       A - B  \right]
  \end{aligned}
  $$

It is straight forward to show (using $\gamma = ln(\sigma)$) that $B = F_Y(y_i)$ (the \emph{cdf} of the corresponding \emph{log-normal} distribution). Let's focus on $A$ now. In the following we replace $\gamma$ with $ln(\sigma)$ and remember that $\frac{dt}{t} = d\left(ln(t)\right)$:

  $$
  \begin{aligned}
    A &=   \int_0^{y_i}   \frac{1}{t \sqrt{2 \pi } \sigma^3}  \left(  ln(t) -  \pmb{z}_i\pmb{\beta}  \right)^2 e^{-\frac{(ln(t) - \pmb{z}_i\pmb{\beta})^2}{2\sigma^2}} dt  = 
    \int_0^{y_i}   \frac{1}{\sqrt{2 \pi } \sigma^3}  \left(  ln(t) -  \pmb{z}_i\pmb{\beta}  \right)^2 e^{-\frac{(ln(t) - \pmb{z}_i\pmb{\beta})^2}{2\sigma^2}} d(ln(t) - \pmb{z}_i\pmb{\beta})
  \end{aligned}
  $$

Now, let's do the following variable substitution:
  $$
  \begin{aligned}
    x &=  \frac{ln(t) - \pmb{z}_i\pmb{\beta}}{\sigma}\\
    e^{-\frac{(ln(t) - \pmb{z}_i\pmb{\beta})^2}{2\sigma^2}}   &=   e^{-\frac{x^2}{2}}\\
    \left(  ln(t) -  \pmb{z}_i\pmb{\beta}  \right)^2   &=  x^2 \sigma^2\\
    d(ln(t) - \pmb{z}_i\pmb{\beta}) &= \sigma dx\\
    t=0~then~x &= -\infty\\
    t = y_i ~then~x  &=\frac{ln(y_i) - \pmb{z}_i\pmb{\beta}}{\sigma} = x(y_i)
  \end{aligned}
  $$
So we have:
  $$
  \begin{aligned}
    A &=   \int_{-\infty}^{x(y_i)}   \frac{1}{ \sqrt{2 \pi } \cancel{\sigma^3}}   x^2 \cancel{\sigma^2} e^{-\frac{x^2}{2}} \cancel{\sigma} dx  =  \frac{1}{ \sqrt{2 \pi } }  \int_{-\infty}^{x(y_i)}     x^2  e^{-\frac{x^2}{2}} dx
  \end{aligned}
  $$
We use integration by parts:
  $$
  \begin{aligned}
    dV &= e^{-\frac{x^2}{2}} x dx = -e^{-\frac{x^2}{2}} d\left(-\frac{x^2}{2}\right);~~~ &V = - e^{-\frac{x^2}{2}}\\
    U &= x;~~~ &dU = dx
  \end{aligned}
  $$
Therefore:
  $$
  \begin{aligned}
    A & =  \frac{1}{ \sqrt{2 \pi } }      \left\{    \left.-x\cdot e^{-\frac{x^2}{2}}\right|_{-\infty}^{x(y_i)}  + \int_{-\infty}^{x(y_i)} e^{-\frac{x^2}{2}} dx   \right\}      dx =  \frac{1}{ \sqrt{2 \pi } }   \left[-x(y_i)\cdot e^{-\frac{x(y_i)^2}{2}}  + 0  \right]  +  \frac{1}{ \sqrt{2 \pi } }  \int_{-\infty}^{x(y_i)} e^{-\frac{x^2}{2}} dx  =\\
    & =  \frac{-(ln(y_i) - \pmb{z}_i\pmb{\beta})}{ \sigma\sqrt{2 \pi } }  \cdot e^{-\frac{(ln(y_i) - \pmb{z}_i\pmb{\beta})^2}{2\sigma^2}}   +  \Phi_{St.N.}\left(  \frac{ln(y_i) - \pmb{z}_i\pmb{\beta}}{\sigma}  \right)\\
  \end{aligned}
  $$
Where $\Phi_{St.N.}(\cdot)$ is the \emph{cdf} of the \emph{standard normal} distribution.\\
Let's summarize:

  $$
  \begin{aligned}
    %%%%%%%%%%%% deriv by beta_i
    \frac{\partial PSR_{Y_i}}{\partial \beta_j} & =  -(1+\delta_i) \frac{z_{ji} }{\sigma\sqrt{2\pi} }    e^{ - \frac{\left( ln(y_i) -\pmb{z}_i\pmb{\beta} \right)^2}{2\sigma^2} }\\
    %%%%%%%%%%%% deriv by gamma
    \left.\frac{\partial PSR_{Y_i}}{\partial \gamma}\right|_{\gamma = ln(\sigma)} &=~~(1+\delta_i)\left[    -\frac{ln(y_i) - \pmb{z}_i\pmb{\beta}}{\sigma}\cdot\phi_{St.N.}\left(  \frac{ln(y_i) - \pmb{z}_i\pmb{\beta}}{\sigma}  \right)   +  \Phi_{St.N.}\left(  \frac{ln(y_i) - \pmb{z}_i\pmb{\beta}}{\sigma}  \right) - F_Y(y_i)    \right]
  \end{aligned}
  $$
Where $\phi_{St.N.}$ and $\Phi_{St.N.}$ are the \emph{pdf} and the \emph{cdf} of the \emph{standard normal} distribution, and $F_Y(y_i)$ is the \emph{cdf} of the \emph{log-normal} distribution with $\mu = \pmb{z}_i\pmb{\beta}$ and $\sigma$.

\subsubsection{M-estimation for \emph{Cox} survival model}

We recall again the definition of PSR, $PSR_{Y_i} =  (1+\delta_i) F(y)  -  \delta_i$. Package \texttt{PResiduals} provides us with PSR through martingale residuals. We only need to get $\frac{\partial PSR}{\partial \beta_i}$:   $$
  \begin{aligned}
    F(y_i) &= 1 - S(y_i) = 1 - e^{-\Lambda(t) e^{\pmb{z}_i\pmb{\beta}}} \\
    \frac{\partial F(y_i, \pmb{\beta})}{\partial \beta_j} &=  -\frac{\partial S(y_i, \pmb{\beta})}{\partial \beta_j} = -\frac{\partial}{\partial \beta_j}\left[  e^{-\Lambda(t) e^{\pmb{z}_i\pmb{\beta}}}  \right] =   e^{-\Lambda(t) e^{\pmb{z}_i\pmb{\beta}}}  \cdot  e^{\pmb{z}_i\pmb{\beta}}   \cdot z_{ij} \\
    \frac{\partial PSR_{Y_i}}{\partial \beta_j} &=  (1+\delta_i)  e^{-\Lambda(t) e^{\pmb{z}_i\pmb{\beta}}}  \cdot  e^{\pmb{z}_i\pmb{\beta}}   \cdot z_{ij}=  (1+\delta_i)  S(y_i, \pmb{\beta})  \cdot  e^{\pmb{z}_i\pmb{\beta}}   \cdot z_{ij}
  \end{aligned}
  $$
We compute $\frac{\partial PSR_{Y_i}}{\partial \beta_j}$ using martingale residuals provided by \texttt{coxph}: 
  $$
  \begin{aligned}
    \hat{m}_i &= \delta_i + log\left(\hat{S}(y_i) \right) \\
    \hat{S}(y_i) &= e^{\hat{m}_i - \delta_i} \\
    \frac{\partial PSR_{Y_i}}{\partial \beta_j} &=  (1+\delta_i)  \hat{S}(y_i)  \cdot  e^{\pmb{z}_i \hat{\pmb{\beta}}}   \cdot z_{ij} =  (1+\delta_i)  e^{\hat{m}_i - \delta_i+\pmb{z}_i  \hat{\pmb{\beta}}}   \cdot z_{ij}
  \end{aligned}
  $$
  
This is not the end of the story though. The survival function is estimated by something like this (needs to be verified for \texttt{R}):
  $$
  \begin{aligned}
    \hat{S}(y_i) = e^{-\sum_{j\leq i} \hat{\lambda}_0(y_i)}
  \end{aligned}
  $$
Where $\hat{\lambda}_0(y_i)$ is a parameter of a piece-wise baseline hazard for Cox model. Because this is a parameter, therefore its estimating equation needs to be including in M-estimation matrices. I need to talk to Cindy about this, but as an approximation, we can probably use the following equation:

  $$
  \begin{aligned}
    \hat{\lambda}_0(y_i) &= \frac{\delta_i}{\sum_{j\in \mathcal{R}(y_i)} e^{\pmb{z_j}\pmb{\beta}}}~,~~~~
    where~\mathcal{R}(y_i)~is~a~risk~set
  \end{aligned}
  $$


Bryan has suggested that we need to take into account also the $\lambda_i$, 



\clearpage

\subsubsection{Miscellaneous}
Scores in \texttt{coxph} can be extracted as \texttt{mod\$score}
Questions to Bryan:
\begin{enumerate}[1)]
  \item should we include the score equation for scale, not only for $\beta$ coefficients?
\end{enumerate}

\subsection{M-estimation for conditional and partial-conditional correlation}
Another way of approaching this problem was suggested by \cite{lui2015covariate}. Instead of taking derivatives like $\frac{\partial (PSR_Y^2(\pmb{\theta}_Y))}{\partial \beta_2}$, one can run a regression: $?$.\\
Under construction ...


\section{To-Do List}
\begin{enumerate}
  \item Cox regression score equations for $\lambda_i$ !!!
  \item Paper in the following order: section 2-3 (PS method), section 4 (simplier simulations), section 5 (CASSANet), section 6 (Discussion).
  \item Chun: what if we integrate outside the covariance
  \item Look at the papers produced by CASSANet data: look up the definitions of baseline RNA, and baseline CD4.
  % \item Add age and site to the covariate list
  % \item Add splines after transforming RNA as $log$ and transforming CD4 as $\sqrt{\cdot}$.
  \item Check the definition of baseline RNA: it should be within -180 to 0 days.
  \item Check virologic failure that should start 6 months after the starting point: Yours start right away.
  \item
\end{enumerate}


% \subsection{Linear regression scores}
% In this section we are trying to understand how Qi is computing \emph{p-value} for \emph{Spearman correlation} (or correlation of PSR). We first try to understand function \texttt{lm.scores()}.
%

\printbibliography


\end{document}          

